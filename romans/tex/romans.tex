% ==== Document Class & Packages =====
\documentclass[12pt,hidelinks]{article}
	\usepackage[explicit]{titlesec}
	\usepackage{titletoc}
	\usepackage{tocloft}
	\usepackage{charter}
	\usepackage[many]{tcolorbox}
	\usepackage{amsmath}
	\usepackage{graphicx}
	\usepackage{xcolor}
	\usepackage{tikz,lipsum,lmodern}
	\usetikzlibrary{calc}
	\usepackage[english]{babel}
	\usepackage{fancyhdr}
	\usepackage{mathrsfs}
	\usepackage{empheq}
	\usepackage{fourier}% change to lmodern if fourier is no available
	\usepackage{wrapfig}
	\usepackage{fancyref}
	\usepackage{hyperref}
	\usepackage{cleveref}
	\usepackage{listings}
	\usepackage{varwidth}
	\usepackage{longfbox}
	\usepackage{geometry}
	\usepackage{marginnote}
	\tcbuselibrary{theorems}
	\tcbuselibrary{breakable, skins}
	\tcbuselibrary{listings, documentation}
	\geometry{
		a4paper,
		left=33mm,
		right=33mm,
		top=20mm}
% ========= Path to images ============
%   - Direct the computer on the path 
% 	  to the folder containg the images
% =====================================
\graphicspath{{./images/}}
% ============= Macros ================
\newcommand{\fillin}{\underline{\hspace{.75in}}{\;}}
\newcommand{\solution}{\textcolor{mordantred19}{Solution:}}
\setlength{\parindent}{0pt}
\addto{\captionsenglish}{\renewcommand*{\contentsname}{Table of Contents}}
\linespread{1.2}
% ======== Footers & Headers ==========
\cfoot{\thepage}
\chead{}\rhead{}\lhead{}
% =====================================
\renewcommand{\thesection}{\arabic{section}}
\newcommand\sectionnumfont{% font specification for the number
	\fontsize{380}{130}\color{myblueii}\selectfont}
\newcommand\sectionnamefont{% font specification for the name "PART"
	\normalfont\color{white}\scshape\small\bfseries }
% ============= Colors ================
% ----- Red -----
\definecolor{mordantred19}{rgb}{0.68, 0.05, 0.0}
% ----- Blue -----
\definecolor{st.patrick\'sblue}{rgb}{0.14, 0.16, 0.48}
\definecolor{teal}{rgb}{0.0, 0.5, 0.5}
\definecolor{beaublue}{rgb}{0.74, 0.83, 0.9}
\definecolor{mybluei}{RGB}{0,173,239}
\definecolor{myblueii}{RGB}{63,200,244}
\definecolor{myblueiii}{RGB}{199,234,253}
% ---- Yellow ----
\definecolor{blond}{rgb}{0.98, 0.94, 0.75}
\definecolor{cream}{rgb}{1.0, 0.99, 0.82}
% ----- Green ------
\definecolor{emerald}{rgb}{0.31, 0.78, 0.47}
\definecolor{darkspringgreen}{rgb}{0.09, 0.45, 0.27}
% ---- White -----
\definecolor{ghostwhite}{rgb}{0.97, 0.97, 1.0}
\definecolor{splashedwhite}{rgb}{1.0, 0.99, 1.0}
% ---- Grey -----
\definecolor{whitesmoke}{rgb}{0.96, 0.96, 0.96}
\definecolor{lightgray}{rgb}{0.92, 0.92, 0.92}
\definecolor{floralwhite}{rgb}{1.0, 0.98, 0.94}
% ========= Part Format ==========
\titleformat{\section}
{\normalfont\huge\filleft}
{}
{20pt}
{\begin{tikzpicture}[remember picture,overlay]
	\fill[myblueiii] 
	(current page.north west) rectangle ([yshift=-13cm]current page.north east);   
\node[
	fill=mybluei,
	text width=2\paperwidth,
	rounded corners=6cm,
	text depth=18cm,
	anchor=center,
	inner sep=0pt] at (current page.north east) (parttop)
	{\thepart};%
\node[
	anchor=south east,
	inner sep=0pt,
	outer sep=0pt] (partnum) at ([xshift=-20pt]parttop.south) 
	{\sectionnumfont\thesection};
\node[
	anchor=south,
	inner sep=0pt] (partname) at ([yshift=2pt]partnum.south)   
	{\sectionnamefont SECTION};
\node[
	anchor=north east,
	align=right,
	inner xsep=0pt] at ([yshift=-0.5cm]partname.east|-partnum.south) 
	{\parbox{.7\textwidth}{\raggedleft#1}};
\end{tikzpicture}%
}
% ========= Hyper Ref ===========
\hypersetup{
	colorlinks,
	linkcolor={red!50!black},
	citecolor={blue!50!black},
	urlcolor={blue!80!black}
}
% ========= Example Boxes =============
\tcbset{
	defstyle/.style={
		fonttitle=\bfseries\upshape, 
		fontupper=\slshape,
		arc=0mm, 
		beamer,
		colback=blue!5!white,
		colframe=blue!75!black},
	theostyle/.style={
		fonttitle=\bfseries\upshape, 
		fontupper=\slshape,
		colback=red!10!white,
		colframe=red!75!black},
	visualstyle/.style={
		height=6.5cm,
		breakable,
		enhanced,
		leftrule=0pt,
		rightrule=0pt,
		bottomrule=0pt,
		outer arc=0pt,
		arc=0pt,
		colframe=mordantred19,
		colback=lightgray,
		attach boxed title to top left,
		boxed title style={
			colback=mordantred19,
			outer arc=0pt,
			arc=0pt,
			top=3pt,
			bottom=3pt,
		},
		fonttitle=\sffamily,},
	discussionstyle/.style={
		height=6.5cm,
		breakable,
		enhanced,
		rightrule=0pt,
		toprule=0pt,
		outer arc=0pt,
		arc=0pt,
		colframe=mordantred19,
		colback=lightgray,
		attach boxed title to top left,
		boxed title style={
			colback=mordantred19,
			outer arc=0pt,
			arc=0pt,
			top=3pt,
			bottom=3pt,
		},
		fonttitle=\sffamily},
	mystyle/.style={
		height=6.5cm,
		breakable,
		enhanced,
		rightrule=0pt,
		leftrule=0pt,
		bottomrule=0pt,
		outer arc=0pt,
		arc=0pt,
		colframe=mordantred19,
		colback=lightgray,
		attach boxed title to top left,
		boxed title style={
			colback=mordantred19,
			outer arc=0pt,
			arc=0pt,
			top=3pt,
			bottom=3pt,
		},
		fonttitle=\sffamily},
	aastyle/.style={
			height=3.5cm,
			enhanced,
			colframe=teal,
			colback=lightgray,
			colbacktitle=floralwhite,
			fonttitle=\bfseries,
			coltitle=black,
		attach boxed title to top center={
	  		yshift=-0.25mm-\tcboxedtitleheight/2,
	   		yshifttext=2mm-\tcboxedtitleheight/2}, 
		boxed title style={boxrule=0.5mm,
			frame code={ \path[tcb fill frame] ([xshift=-4mm]frame.west)
				-- (frame.north west) -- (frame.north east) -- ([xshift=4mm]frame.east)
				-- (frame.south east) -- (frame.south west) -- cycle; },
			interior code={ 
				\path[tcb fill interior] ([xshift=-2mm]interior.west)
				-- (interior.north west) -- (interior.north east)
				-- ([xshift=2mm]interior.east) -- (interior.south east) -- (interior.south west)
				-- cycle;} }
				},
	examstyle/.style={
		height=9.5cm,
		breakable,
		enhanced,
		rightrule=0pt,
		leftrule=0pt,
		bottomrule=0pt,
		outer arc=0pt,
		arc=0pt,
		colframe=mordantred19,
		colback=lightgray,
		attach boxed title to top left,
		boxed title style={
			colback=mordantred19,
			outer arc=0pt,
			arc=0pt,
			top=3pt,
			bottom=3pt,
		},
		fonttitle=\sffamily},
	doc head command={
		interior style={
			fill,
			left color=yellow!20!white, 
			right color=white}},
	doc head environment={
		boxsep=4pt,
		arc=2pt,
		colback=yellow!30!white,
		},
	doclang/environment content=text
}
% ============= Boxes ================
\newtcolorbox[auto counter,number within=section]{example}[1][]{
	mystyle,
	title=Example~\thetcbcounter,
	overlay unbroken and first={
		\path
		let
		\p1=(title.north east),
		\p2=(frame.north east)
		in
		node[anchor=
			west,
			font=\sffamily,
			color=st.patrick\'sblue,
			text width=\x2-\x1] 
		at (title.east) {#1};
	}
}
\newtcolorbox[auto counter,number within=section]{longexample}[1][]{
	examstyle,
	title=Example~\thetcbcounter,
	overlay unbroken and first={
		\path
		let
		\p1=(title.north east),
		\p2=(frame.north east)
		in
		node[anchor=
		west,
		font=\sffamily,
		color=st.patrick\'sblue,
		text width=\x2-\x1] 
		at (title.east) {#1};
	}
}
\newtcolorbox[auto counter,number within=section]{example2}[1][]{
	aastyle,
	title=Example~\thetcbcounter,{}
}
\newtcolorbox[auto counter,number within=section]{discussion}[1][]{
	discussionstyle,
	title=Discussion~\thetcbcounter,
	overlay unbroken and first={
		\path
		let
		\p1=(title.north east),
		\p2=(frame.north east)
		in
		node[anchor=
		west,
		font=\sffamily,
		color=st.patrick\'sblue,
		text width=\x2-\x1] 
		at (title.east) {#1};
	}
}
\newtcolorbox[auto counter,number within=section]{visualization}[1][]{
	visualstyle,
	title=Visualization~\thetcbcounter,
	overlay unbroken and first={
		\path
		let
		\p1=(title.north east),
		\p2=(frame.north east)
		in
		node[anchor=
		west,
		font=\sffamily,
		color=st.patrick\'sblue,
		text width=\x2-\x1] 
		at (title.east) {#1};
	}
}
% --------- Theorems ---------
\newtcbtheorem[number within=subsection,crefname={definition}{definitions}]%
	{Definition}{Definition}{defstyle}{def}%
\newtcbtheorem[use counter from=Definition,crefname={theorem}{theorems}]%
	{Theorem}{Theorem}{theostyle}{theo}
	%
\newtcbtheorem[use counter from=Definition]{theo}{Theorem}%
{
	theorem style=plain,
	enhanced,
	colframe=blue!50!black,
	colback=yellow!20!white,
	coltitle=red!50!black,
	fonttitle=\upshape\bfseries,
	fontupper=\itshape,
	drop fuzzy shadow=blue!50!black!50!white,
	boxrule=0.4pt}{theo}
\newtcbtheorem[use counter from=Definition]{DashedDefinition}{Definition}%
 {
 	enhanced,
 	frame empty,
 	interior empty,
 	colframe=darkspringgreen!50!white,
	coltitle=darkspringgreen!50!black,
	fonttitle=\bfseries,
	colbacktitle=darkspringgreen!15!white,
	borderline={0.5mm}{0mm}{darkspringgreen!15!white},
	borderline={0.5mm}{0mm}{darkspringgreen!50!white,dashed},
	attach boxed title to top center={yshift=-2mm},
	boxed title style={boxrule=0.4pt},
	varwidth boxed title}{theo}
%%%%%%%%%%%%%%%%%%%%%%%%%%%%%%%%%%%%%%%%
\newtcblisting[auto counter,number within=section]{disexam}{
	skin=bicolor,
	colback=white!30!beaublue,
	colbacklower=white,
	colframe=black,
	before skip=\medskipamount,
	after skip=\medskipamount,
	fontlower=\footnotesize,
	listing options={style=tcblatex,texcsstyle=*\color{red!70!black}},}
%%%%%%%%%%%%%%%%%%%%%%%%%%%%%%%%%%%%%%%

\begin{document}
\begin{titlepage}
	\centering % Center everything on the title page
	\scshape % Use small caps for all text on the title page
	\vspace*{1.5\baselineskip} % White space at the top of the page
% ===================
%	Title Section 	
% ===================

	\rule{13cm}{1.6pt}\vspace*{-\baselineskip}\vspace*{2pt} % Thick horizontal rule
	\rule{13cm}{0.4pt} % Thin horizontal rule
	
		\vspace{0.75\baselineskip} % Whitespace above the title
% ========== Title ===============	
	{	\Huge The Epistle to the Romans\\ 
			\vspace{4mm}
		by Paul the Apostle \\	}
% ======================================
		\vspace{0.75\baselineskip} % Whitespace below the title
	\rule{13cm}{0.4pt}\vspace*{-\baselineskip}\vspace{3.2pt} % Thin horizontal rule
	\rule{13cm}{1.6pt} % Thick horizontal rule
	
		\vspace{1.75\baselineskip} % Whitespace after the title block
% =================
%	Information	
% =================
	{\large Andrew Yong \\
		\vspace*{1.2\baselineskip}
	University of Family Bible Church} \\
	\vfill
If you come across any problems, contact me at \url{yartow@gmail.com}\\ \vspace{1mm}.
\end{titlepage}
%%%%%%%%%%%%%%%%%%%%%%%%%%%%%%%%%%%%%%%%%%%%%%%%%%%%%%%%%%%
\tableofcontents
\vfill
\small{\noindent \textbf{About This File} \vspace{-3mm}\\
\noindent \rule{3.3cm}{0.5pt} \\
This file was created for the benefit of all teachers and students wanting to know more about God and His holy Word.\\
The entirety of the contents within this file, and folder, are free for public use.}
\newpage
\newgeometry{
	left=29mm, 
	right=29mm, 
	top=20mm, 
	bottom=15mm}
%%%%%%%%%%%%%%%%%%%%%%%%%%%%%%%%%%%%%%%%%%%%%%%%%%%%%%%%%%%

%%%%%%%%%%%%%%%%%%%%%%%%%%%%%%%%%%%%%%%%%%%%%%%%%%%%%%%%%%%

\hypertarget{romans-1-esv}{%
\section{Romans 1 (ESV)}\label{romans-1-esv}}

\textbf{Greeting} \emph{1 Paul, a servant of Christ Jesus, called to be
an apostle, set apart for the gospel of God, 2 which he promised
beforehand through his prophets in the holy Scriptures, 3 concerning his
Son, who was descended from David according to the flesh 4 and was
declared to be the Son of God in power according to the Spirit of
holiness by his resurrection from the dead, Jesus Christ our Lord, 5
through whom we have received grace and apostleship to bring about the
obedience of faith for the sake of his name among all the nations, 6
including you who are called to belong to Jesus Christ,} \emph{7 To all
those in Rome who are loved by God and called to be saints:} \emph{Grace
to you and peace from God our Father and the Lord Jesus Christ.}

The gospel of God is clearly explained in here and consists of (1) the
prophecy, (2) Jesus human and divinity nature, (3) His death and (4) His
resurrection. However, it is possible to summarize and organize this
text even further. * What is the Gospel? It is these four points
mentioned above. * What is the goal of the Gospel? To bring about the
obedience of faith (v.50). * Why is there a Gospel? For the sake of His
Name among all the nations. * For whom is the Gospel? This is not
answered here---although we know it is meant for everyone because God
did not limit us to exclude certain people, although He is the One Who
decides who will hear it---but it includes those who are in Jesus Christ
(v.6). So even Christians need to hear the Gospel \emph{again}.

\textbf{Longing to Go to Rome} \emph{8 First, I thank my God through
Jesus Christ for all of you, because your faith is proclaimed in all the
world. 9 For God is my witness, whom I serve with my spirit in the
gospel of his Son, that without ceasing I mention you 10 always in my
prayers, asking that somehow by God's will I may now at last succeed in
coming to you.}

The notes on verses 11-13 may indicate as if Paul was not doing any of
his out of love, but purely out of obedience to God. So perhaps there
was love involved for God, but not for man. However, these verses
dispute this claim as Paul indicates that God is his witness of that he
\emph{always} mentions the addressees (i.e.~the Romans) in his prayers.
This in itself is proof enough. Could he be praying out of obedience?
Yes, but it would be far-fetched to think that Paul prayed because he
had to and included in his prayer those he had to pray for. Even though
prayer is an ``obligatory'' part for believers---any believer prays, if
he does not, he cannot be deemed as one---the contents of these prayers
are to be freely filled in by the prayer. So Paul could have prayed for
anything he wanted, such as the saving of the city of Jerusalem, his
fellow Jews in the diaspora or his own wellbeing, but he chose to
include them \emph{always}, which to me sounds much more than merely a
mention of them in his prayers ``every day''.

\emph{11 For I long to see you, that I may impart to you some spiritual
gift to strengthen you--- 12 that is, that we may be mutually encouraged
by each other's faith, both yours and mine. 13 I do not want you to be
unaware, brothers, that I have often intended to come to you (but thus
far have been prevented), in order that I may reap some harvest among
you as well as among the rest of the Gentiles.}

Paul is clear here that he comes to get results. Our work for the
Kingdom of God is not intended to be for the purpose of leisure. He says
in verse 12 that he \emph{longed} to see them, but then he adds the
clause ``that I may impart to you\ldots{}''. Would that mean that he
does not truly long to see them? Is that something like loving someone
\ldots{} because she is beautiful? The latter clause would diminish the
purity of love, it seems, just like it seems to diminish Paul's longing.
He later explains that strengthening them is not the only reason, Paul
Himself wants to be encouraged as well. This actually only adds to
negative feeling of \#todo/opzoeken/engels niet oprechte gevoelens of
bedoelingen So Paul comes for a purpose, and he does it so that he can
reap the benefits of it himself as well, and he is looking for results.
Where is the love here, one might think? It is true, Paul's first and
foremost job is to obey God and not act out of whatever or whomever he
loves. But this does not mean that both things have to be exclusive.

\emph{14 I am under obligation both to Greeks and to barbarians, both to
the wise and to the foolish. 15 So I am eager to preach the gospel to
you also who are in Rome.}

In a certain sense it seems like Paul is referring to those in Rome as
the foolish, because he uses the word ``so'' which seems to refer to the
latter used description, which is ``foolish''. Of course another
interpretation is that ``so'' refers to ``the wise and the foolish'' and
thus anything in between as well. But even if the addressees are
considered the foolish, would that be wrong? It would be foolish
compared to the Greek. Now that would not be harsh for two reasons.
First of all the Greeks were very intelligent people to whom many
literary, philosophical, methodological and mathematical works may be
attributed. Almost any other people would be like barbarians when
compared with them. Second of all, however, the Greeks were barbaric
\emph{morally speaking} compared to the Jews and Christians. So if
Christians were ``barbaric'' compared to Greeks, then that could only be
a compliment. Being anything other than a Greek on the field of religion
would always be a compliment.

\textbf{The Righteous Shall Live by Faith} \emph{16 For I am not ashamed
of the gospel,}

Why would Paul mention anything here about shame? Why would it be
shameful to have a God Who was foretold, is simultaneously human, and
who died and rose again? It is exactly for those reasons. A god who is
human is a lesser god---a \emph{hero} or a demi-god, as the Greeks would
call---lower than the other real gods, unless they are made into gods. A
god who dies has weaknesses and is not almighty. One who rose
again---out of the Hades that is---indicates that he was weak enough to
be thrown in it in the first place. In other words, compared to the
Greek and Roman gods, and perhaps the Babylonian and all other pagan
gods, Jesus Christ was less than a god. They do not understand many
things about our Lord, however. For one, they do not understand the
concept of the Trinity, and of a Person Who can be god and human at the
same time. A hundred percent plus a hundred percent adds up to two
hundred percent according to logic, but they do not realize that that is
human logic. Once they do understand, they would realize that a human
god would only add to the godliness of Jesus Christ. Jesus felt all the
pain as a human did, a weak one at that, unlike Simson who must have
felt pain when fighting the Philistines or Hercules when he had this
poisoned mantle covering him, but who were very strong to begin with.
Jesus had a normal body, not of a soldier, perhaps a bit stronger than
average due to carpenting, but nothing out of the extraordinary. He had
feelings, much \emph{more} than human beings exactly \emph{because} of
His godliness. As God He could feel everything we were feeling and He
did so on the cross, where He suffered so much that He instantly died,
of all the feelings and pain of all people before and after Him.

So the first argument that Jesus was only a demi-god is not true,
because the Bible uses the wholly different concept of the God-head
(three in one) which has never appeared throughout history in any
religion or literature.

The second argument of Jesus' weakness in His humanness
\#todo/opzoeken/engels is invalid because His humanness actually makes
Him stronger than other gods.

Now a third argument people use is His death. Which is also one of the
main reasons why the Jews did not believe Jesus to be the Messiah. Death
shows weakness, because apparently one was not strong enough to defend
Himself. It also shows terminality, \#todo/opzoeken/engels een
eindigheid, because God's reign would be forever, although (some) Jews
like Ben Shapiro \#tags/famouspeople believe the Messiah is not a Person
who will reign forever, but just a human, a political figure who will
accomplish things on Earth. They are forgetting one of the most
important traits of a real God, however, which other gods do not have,
which is meekness. I rarely hear about meekness or self-sacrifice in
Buddhism, Hinduism and Islam. Yes, there is the story where Buddha
sacrificed himself by killing one person (and perhaps getting executed
for that as well) because he knew this person would kill a hundred.
\#todo/nogaftemaken

\emph{for it is the power of God for salvation to everyone who believes,
to the Jew first and also to the Greek. 17 For in it the righteousness
of God is revealed from faith for faith, as it is written, ``The
righteous shall live by faith.''}

This is a difficult sentence. It says that God's righteousness is ``from
faith for faith''. What would that mean? God's righteousness must be
\emph{revealed}, so not everyone can see it or know about it unless God
chooses to reveal it to someone. Now it is revealed from faith. So faith
is like a pair a of glasses with which righteousness can be seen. What
many do not know, is that God's righteousness contains an aspect of
time. As we have seen in both the Pharaoh in Moses' time, and Noah's and
Jonah's stories is that people were rewarded or punished in a context of
time. God did not punish people directly, but He had his time. Now
delayed punishment requires patience, when one wants justice to occur
immediately, but patience is built on a greater thing called reasoning.
Why would anyone wait with punishment? Because proof of one's innocence
could appear or remorse or retribution ??. If a person on death row can
provide more information about the criminal network he used to work for,
would that not be worth anything? If he converts to Christianity, would
he then not be valuable as a mentor to many other potential criminals?
Waiting can have a positive effect on the outcome. One needs reasoning
for that. But what if one cannot see any changes? What if none of the
above applies to this particular criminal? Then only one thing remains:
\emph{faith}. Through faith one can have hope, hope that change is
\emph{possible}.\\
There are so many stories where the whole world did not see any hope for
a particular person, yet this person arose from the surface all because
one person still believed in him. Take for example Ben Carson's story
depicted in the movie \emph{Gifted Hands} starring Cuba Gooding Jr.~,
or---perhaps---the movie \emph{Jerry McGuire} where Tom Cruise believes
in Cuba Gooding Jr.~ If we believe that God has a plan with
\emph{delaying} His judgment, we may be able to see His righteousness.
In other words, it is only through belief and for those who believe.
\#church/material

\textbf{God's Wrath on Unrighteousness} \emph{18 For the wrath of God is
revealed from heaven against all ungodliness and unrighteousness of men,
who by their unrighteousness suppress the truth.}

The wrath of God too is revealed. Though it does not say here, it is
revealed to those God wants it to be revealed. How is ungodliness
punished? Not everyone can see this. Again it is through faith only that
we can see this. We see some things happening on Earth and we do not
know how to interpret this. We see for example tumult in Iran and chaos
in Myanmar. We see innocents dying by the thousands. The question arises
in us then where righteousness is. Why are innocents being punished? The
real question is, however, are they really innocent? It requires a
revelation from God to see the answer. Did not they and their
forefathers participate in hundreds if not thousands of years of pagan
worship? Did they not persecute Christians directly or indirectly by
assisting their government or voting for policies against Christians?
Myanmar's current (female) leader, \emph{Aung Kie\ldots{}}
\#todo/opzoeken was chosen democratically. Yet she kills muslims by the
thousands. Are the people not responsible for that either? The same goes
for almost any dictator. They have all been chosen by the people at some
point. To name a few more, Che Guevara, Fidel Castro, Mao Ze Dong and
Hitler \#tags/ideology/communism. Even when some of those ``innocent''
people did not actively persecute Christians, who knows whether they
might have done so, had they been in the right situation and time? Only
God knows and only He can be the judge of that.

What is furthermore written in this verse about the sin of man, is that
man suppresses the truth. That alone is worthy of severe punishment. It
is the covering up of the truth that causes all trouble. Can anyone name
one example of a problem that would have been solved if everyone was
truthful? If a murder intents to kill anyone and he is truthful about
it, he might be stopped in his actions even before they happen. If he
has already done it, he will be apprehended to prevent him from doing it
another time. Of course, within marriages the truth is not enough to
solve a problem, even when one commits adultery and immediately admits
it. Remorse and willingness need to be present as well. But generally
speaking, solving a problem always requires the truth. The question we
need to ask is always ``What caused this problem?'', and the
\emph{cause} is the truth.

\emph{19 For what can be known about God is plain to them, because God
has shown it to them. 20 For his invisible attributes, namely, his
eternal power and divine nature, have been clearly perceived, ever since
the creation of the world, in the things that have been made. So they
are without excuse.}

Now one might think that my statement about suppressing the truth is a
bit harsh, but as it says here, the are without excuse. Do I think
adults are deliberately suppressing the truth to their children about
God, His creation and His Almight? I do think that many atheists and
heathens do not know God, but I do not believe that they have no doubts
at all about their beliefs. Everyone, no matter how little, knows or
feels that atheism is empty and hopeless, that evolution is
counterintuitive---and still they teach their children something that is
against their own conscience. All Leftists know that the gender-alphabet
starting with LGBT is going too far and that for such a small percentage
of people systems need to change, for reasons that are not even
scientifically proven (namely that gender is something between the ears)
and are counterintuitive, namely that children know they are of the
opposite sex at age 3. People from other religions, such as Islam,
Buddhism and Hinduism, also know that something is wrong with their
religion. How can it be that some people are worth more than others?
Even including the concept of reincarnation does not resolve this
problem, because human beings can still feel that beggars, even if they
deserve to be one because of their previous life, should be treated
fairly, simple \emph{because they are human beings}. No religion or
philosophy can talk its way out of that. It is in us because of our
intrinsic worth in God's eyes.

\emph{21 For although they knew God, they did not honor him as God or
give thanks to him, but they became futile in their thinking, and their
foolish hearts were darkened. 22 Claiming to be wise, they became fools,
23 and exchanged the glory of the immortal God for images resembling
mortal man and birds and animals and creeping things.}

I am not sure how much the pagans knew God, neither do I know how much
the Jewish people of the first century knew Him. Except for Jesus'
miracles I do not think either of them have seen much of God. The Jewish
people have their ancestors and their Bible and both have creation as
the proof of God's existence. We can see here a logical order in which
they replaced God with all kinds of attributes. The first thing is
denying God. They knew God---whatever that means, it at least indicates
that they knew Who God was and that they should honor and thank
Him---but it is exactly those things they purposely neglect to do. As
with many or most things in life, there are multiple ways of going, but
when one does not go the right way, he has to \#todo/opzoeken/engels
zichzelf wringen in allerlei bochten he has to go through all kinds of
loops (or hoops \#todo/opzoeken/engels?) just to be able to explain his
choices. This is true for the truth and logic. When a murderer claims he
has not killed someone, he must create an alibi, which could be a
person, and this person as well needs to have an alibi that he saw him,
and so forth, which is not possible anymore after a few steps. When the
evolutions say that the world is created in millions of years they need
many ice ages to explain the formation of our current world, while
Occam's Razor can be applied here as well, it is the simplest
solution---one big ice age after Noah's flood---which can explain the
world's formation as well. Now at a certain moment this becomes flat out
lying to them. They were lying in the beginning already, but at least
they were mixing in truth with lies. No they are flat out lying because
they do not have any other choice. This is what we see with the
Leftists. In the beginning they could mix in some lies with the truth,
the lies being that they want to be good to black people and other
minorities, they want to do good to the people, but after so many things
are exposed about their politics, that this is just a modern
\#todo/opzoeken/engels modern jasje, in a new jacket? of socialism as it
was implemented in the previous century---which did not work by the way
and will never work---people like Samantha Nixon (I do not know if she
was first) simply admitted: ``Ok, if this is what you want to call it,
then we'll admit. We want socialism. But it is not standard socialism,
but \emph{democratic socialism}.''

::\emph{24 Therefore God gave them up in the lusts of their hearts to
impurity, to the dishonoring of their bodies among themselves, 25
because they exchanged the truth about God for a lie and worshiped and
served the creature rather than the Creator, who is blessed forever!
Amen.}:: \#biblestudy/memorization/todo

Here too we find a major principle throughout human lives and which can
be found within sin. It is that God gives up people to their own
longings and that people rather worship the created than the creator,
regardless whether this concerns God or not. Throughout the New
Testament we see that Jesus does not force His will or world views on
others. He attracts people and persuades them with reasoning. If they
want to they can follow Him and if they do not, they can do the thing
they want themselves. The same it is with God. God intervenes, as He has
done with the Ninevites, but that is because He knew that in their
hearts that generation was willing to give up their lust and impurity.
But if a people is not ready to do so, then God leaves this person,
unless He is graceful and tries again. I am not sure of the latter, that
is, whether God tries multiple times to persuade peoples, but from what
I see in the Old Testament God did not kill the Hittites, Peruzzites
\#todo/opzoeken etc. in one time, but gave them multiple chances to give
up the fight. We do not know how much knowledge or opportunities these
pagan peoples in and around Canaan were given about our Lord Almighty,
but we know that He is righteous.

\emph{26 For this reason God gave them up to dishonorable passions. For
their women exchanged natural relations for those that are contrary to
nature; 27 and the men likewise gave up natural relations with women and
were consumed with passion for one another, men committing shameless
acts with men and receiving in themselves the due penalty for their
error.}

In verse 25 we read that God gave them up to their lusts and here in
verse 26 God gives them up to their passions. Also, the people
dishonored their bodies by exchanging the truth for lies, and exchanging
natural relations for unnatural ones, respectively. It is this specific
order in which things happen. Granted, homosexuality has always existed,
long before there were ``scientists'' claiming that gender is something
between the ears, but it was always common knowledge that this was not
something natural. So now, in these days, the Left exchanges the truth
for the lie, and many Christians, who should be on the Right, slowly
(but still too fast) start leaning that way and giving in. And as soon
as that happens the action starts as well. Lies cannot stand the truth,
just like darkness hates light. The Bible is the truth, so if one wants
to follow the lie, he needs to cut out the Bible from His life. So now
some churches such as the \emph{Alexanderkerk} in Rotterdam accept gay
marriages, just like the Remonstrantse Kerk in the Netherlands. The
churches are now Law abiding ``except for one rule''. No, if one starts
cutting in the Bible, or in the truth in that case, the whole truth will
disappear. Once again, as verse 25 and 26 say, if one persists in
exchanging the truth for a lie, God will give them up to their lusts. In
due time these churches will avail themselves.

\emph{28 And since they did not see fit to acknowledge God, God gave
them up to a debased mind to do what ought not to be done.}

So after they have been given up to their own lust and passions, God
gave them up to their debased mind as well. Now this seems all the more
true of the Left. The Left is using their mind in reaching their goals.
Their strategies are well thought of. They have purposefully pointed the
media, education and Hollywood in the same direction and thought of the
strategy to mark who does not agree with them as bigots, homophobes or
xenophobes. It requires a lot of planning to wag the dog in this way.
The sad thing, however, is as verse 28 says: their mind is debased. They
have arrived at the stage where they do not even know what or that they
are doing wrong anymore. Those influenced by the Left have never known
perhaps. But the master strategists, those higher ``ranked'' or longer
seating than Hillary Clinton, they used to know better, but not so
anymore. They minds have become debased and their sin is all the more
increasing. It is different from a person born with a debased mind or
intelligence and who really does not know any better. These persons have
hardened their hearts and their logical thinking and God, Who has now
hardened their hearts even more, will punish them for all their sins,
but also---just like in the case of Pharaoh---use these atrocities for
the good of His people.

\emph{29 They were filled with all manner of unrighteousness, evil,
covetousness, malice. They are full of envy, murder, strife, deceit,
maliciousness. They are gossips, 30 slanderers, haters of God, insolent,
haughty, boastful, inventors of evil, disobedient to parents, 31
foolish, faithless, heartless, ruthless.}

What debased means in the previous verse is more explicitly explained
here. The sins range here from very obvious sins to less obscure sins.
We see that these sins which included only the worshipping of other
gods, and sinning towards God, now includes sinning towards people as
well---though that is not very clear from the text. What is clear,
however, is that there sin is ever increasing. The type of sin, the
amount of sins and the way their bodies are harmed, and the continuation
of all of these. Being boastful seems like a small sin, but remember
that their boastfulness is based on their pride of knowledge. They
\emph{think} they are wise and influence other people to do so as well.
So even if they are not hurting other people, they are causing them to
sin when these people are following them.

\emph{32 Though they know God's righteous decree that those who practice
such things deserve to die, they not only do them but give approval to
those who practice them.}

Just when you think that these people Paul is speaking of are the worst
of their kind, it appears there is something that makes them even worse.
So aside from sinning against God, the conscience, their mind, their
body and other people, they now stimulate other people to sin as well.
Let us look at the former verse and this one together and see how much
this resembles the Left party in the United States. The Left hates God
and everything He commands and stands for. They are in fact inventors of
evil, by having invented a multitude of devices, institutions and
policies which discriminate and prey upon the weakest people in society,
such as \emph{Planned Parenthood} which preys on the weakness of
(especially young) women of other ethnicity who have trouble being a
single mother; \emph{Food stamps} have been created to help the weakest
in society, but the excessive distribution of these keeps the poor poor
without any incentive for labor; They have also invented things such as
the decoupling of money with the gold standard and perhaps they have
invented something like \emph{Federal Reserve Bank} which controls all
money to keep America poor. In this verse we see that the Left which has
done all of the above, endorses people to go further than they have and
to accomplish their dream of open borders for America, free education,
medical care---even for illegal immigrants---repayment of all student
debts, and what not more. All of this already depletes what is left of
federal reserves of the United States

\#biblestudy/devotionals/romans \#tags/politics \#tags/lgbt

\hypertarget{romans-10-esv}{%
\section{Romans 10 (ESV)}\label{romans-10-esv}}

\emph{1 Brothers, my heart's desire and prayer to God for them is that
they may be saved. 2 For I bear them witness that they have a zeal for
God, but not according to knowledge. 3 For, being ignorant of the
righteousness of God, and seeking to establish their own, they did not
submit to God's righteousness. 4 For Christ is the end of the law for
righteousness to everyone who believes.}

Being zealous alone is not a sufficient condition for salvation. Their
zeal, or their passion, is what drives them, but Paul has clearly said
that we are in control of our body, in the sense that we can decide when
we do something or not. Note that this does not contrast Romans 7:16 for
there he is speaking of a separation between his flesh and his will. His
will, however, is under his control. So if the Jews are zealous in
converting everyone from Christianity back to Judaism, it might be their
passion that is doing that, but they are ignoring their knowledge about
God then. One cannot use only one aspect to justify his actions. Neither
is that the case for good intentions. Someone with good intentions has
to use his brains as well and think about what he is going to do or the
consequences might be disastrous. In this case it is their will that is
the problem, which is the one thing they do have fully control over. If
they had wanted it, they would have searched for knowledge, to know who
God really is. Anthony Flew \#tags/famouspeople is one of those
intellectuals who lacked knowledge about God. I do not know his past or
how he has been influenced in becoming an atheist---although he did
clearly show his indignation of God's, to him, unrighteous punishment of
hell for unbelievers, which might give a hint as to the origin of his
atheism and battle against God---but I have heard one of his debates
with William Lane Craig. Like most professional debaters, they are---or
at least try to be---intellectually consistent. Flew then had to find a
way to disprove of the Kalaam argument of the latter
(i.e.~\emph{everything that began to exist, must have had a cause}) and
could not, so he ended up in his argument that this argument is valid
only within the universe and we do not know if this argument is valid
for anything outside of our universe. Now that may be true, but it is
not the most intuitive answer and he knows that. But he is trying to go
for any route he can go, as long as he does not have to admit that God
exists. Eventually, though, through the discovery of DNA, he could not
do anything but admit that there must be a God Who could create such
complex and logic material within each cell of each living organism. The
Jews are alike, not only do some not think about searching for the
truth, others try to think of another way of explaining it. But those
who have seen Jesus alive and fully well, in a matter of \emph{days}
after a flogging and crucifixion, have purposely deceived other Jews
into believing that He did not rise and that his body was stolen. They
are not only intellectually inconsistent, but also unfaithful to the
truth. They care more about their view and their position than about the
truth, which would lead them to God.

\textbf{The Message of Salvation to All} \emph{5 For Moses writes about
the righteousness that is based on the law, that the person who does the
commandments shall live by them. 6 But the righteousness based on faith
says, ``Do not say in your heart, `Who will ascend into heaven?'\,''
(that is, to bring Christ down) 7 ``or `Who will descend into the
abyss?'\,'' (that is, to bring Christ up from the dead). 8 But what does
it say? ``The word is near you, in your mouth and in your heart'' (that
is, the word of faith that we proclaim); 9 ::because, if you confess
with your mouth that Jesus is Lord and believe in your heart that God
raised him from the dead, you will be saved. 10 For with the heart one
believes and is justified, and with the mouth one confesses and is
saved.::} \#biblestudy/memorization/todo

Moses' law said that those who do the law, will live by them. They will
not merely do it, in the sense that one does it only when other people
are watching, or do it merely out of obedience or fear of God. No, they
do it because it is a part of their life. When we come to faith,
however, we do them because the Word lives within us. Letting the law
become a part of your life is good, but when it becomes a tradition,
that is a wrong motivation as well. So now we do not have to get Christ
down or ask Him to be with us, because we know He is already with us.
When we pray and ask Him to be with us, we ask, however, that we can
feel his presence. So the Word is in us and that it comes out of our
mouth proves that was in us. That is also why it is so important that we
confess with our mouths that Jesus is Lord. For one can go without the
other, that is confessing with the mouth can be done without believing
with the heart, but believing in the heart cannot be done without
confessing with the mouth.

\emph{11 For the Scripture says, ``Everyone who believes in him will not
be put to shame.''}

We have seen Christians being martyred and being bullied for their
conservative views and even being called old-fashioned, bigots,
homophobes and what more? So we have been put to shame for the world.
But that is exactly what is important here. God wants to be proud of us,
in \emph{His} eyes that is, and then, almost by definition, the world
will automatically put us to shame. Take the example of a white person
in a black country, he would immediately be very noticeable and vice
versa. The same it is for those in the kingdom of God or those
\emph{from} God. It would rather be a problem if you were not
noticeable, for then it would mean you have become like them (when you
were white) or you have never fully become white. So yes, you might be
put to shame, but you will not be shamed in front of God, but rather in
the presence of those who have wrong views, although you should not have
looked up to them in the first place (cf.~Romans 1:16).

\emph{12 For there is no distinction between Jew and Greek; for the same
Lord is Lord of all, bestowing his riches on all who call on him. 13 For
``everyone who calls on the name of the Lord will be saved.''}

Is it really that simple? Is all one has to do call on Jesus in order to
be saved? No, this verse should not be taken out of context. The verses
prior already say this has to be done with the heart and the mouth, and
the verses hereafter clearly indicate that belief is necessary.

\#todo voorbeeld aan KenPei. Je kunt pas gered worden als je ergens in
gelooft. Zie ook het voorbeeld van het touw in de put. Je gelooft dat
het touw er is (de mogelijkheid om gered te worden), en daarmee geloof
je ook dat Jezus Degene is Die het touw heeft uitgeworpen naar je. Je
gelooft dat Jezus het touw niet zal loslaten, want wie zou er nou een
touw gaan pakken dat halverwege losgelaten wordt? Hiermee toon je aan
dat je Jezus vertrouwt met \sout{heel} je leven. Hij heeft nu de
controle over je leven. (het punt over het loslaten van de dingen in je
leven heeft hiermee te maken) Als je dan uit de put komt, zijn er geen
voorwaarden meer. Er is geen voorwaarde dat je niet de put in mag
springen, er is geen voorwaarde dat je Jezus om de hals moet vliegen en
Hem moet bedanken. En er is ook geen voorwaarde dat je Hem moet
aanbidden en prijzen. Dat je hem moet erkennen als God (niet perse jouw
God, maar wel als God zijnde) is niet een voorwaarde, maar is impliciet,
omdat je hebt gezien dat Hij uit de put geklommen is en jou het touw
heeft toegereikt. Als je dat niet had geloofd, dan had je ook niet
geloofd dat het touw echt was. \#todo/familytime \#todo nog af te maken

::\emph{14 How then will they call on him in whom they have not
believed? And how are they to believe in him of whom they have never
heard? And how are they to hear without someone preaching? 15 And how
are they to preach unless they are sent? As it is written, ``How
beautiful are the feet of those who preach the good news!''}::
\#biblestudy/memorization/todo

This is a logical statement that follows from each of the pretexts.
Everyone should agree with this, although some might claim they have
heard through dreams or prophecies, and some preachers might not be sent
but might have felt a calling. Then there are others who call on God
without believing in Him. All of these pretexts are groups that exist in
actuality, but who are not pure in their teachings, because they do not
follow the logic in these verses. The first group are the nominal
Christians, people raised---usually---in a Reformed, Catholic or
Orthodox environment, where going to church has become a cultural thing
which one does not with the heart but out of duty or simple because
everyone does it. They sometimes do good, but quite often live a secular
life when no one sees is. The Academy Award®-nominated movie \emph{As it
is in heaven} \#tags/movies depicts such a situation of a pastor who
prays at home for his sins of lust, after having had sex with his wife.
It is clear that he is from a strict form of religion based on works,
and though he prayed to God in this scene, which is calling on God, he
does not believe in Him or know Him like the Bible represents Him. The
second group are people who have heard of God in the broadest sense of
the word, like Muhammad has heard of Jewish, Christian and apocryphal
stories and put them in the Quran, or even like Muslims of all ages who
have not read the Quran---or even worse, who do not understand Arabic
nor what is being said and explained in mosques---and only hear things
\emph{about} Allah \#tags/religion. Another group that clarifies the
example is a lost group of one the twelve tribes of Israel, Dan. It has
been researched that in Africa there is a tribe with Jewish DNA, customs
and words in their language of which they do not know where these come
from or why they do it. They are animists as well, which means they
worship nature, in the way God should be worshipped according to the Old
Testament. Because they have never heard of God, their whole worship is
erroneous. \#todo/opzoeken/engels faulty, beter woord voor bedenken.
\#todo nog af te maken The third group is closer to God and His Word.
This group has seen God and they know who He is. The people here are for
example often former Muslims, for in Muslim countries God often uses
dreams, perhaps because Muslims believe in the power of dreams or
because in Muslim countries it is not allowed to spread the Gospel. In
China as well there have been cases like this where not only the spread
of the Gospel and the physical Bible was prohibited, but teaching was
very limited as well. People like Brother Yun from the book ``\emph{The
Heavenly Man}'' \#tags/books and Watchman Nee have had teachings from
God, not in the way Paul had, but God spoke to them very clearly and
they both had a lack of godly teachers so---I assume---they had to
extrapolate their knowledge from what was in the Bible (or what they had
memorized thereof) and what God had given them. This of course is not
completely done without speculation or inserting things, for which we
cannot fully blame them, but the ultimate conclusion is that their
teachings might not be completely sound. Now it is a fact that no
teacher has a flawless theology nor does anyone have no opposition
\#todo/opzoeken/engels tegenstanders .---not even Derek Prince, apart
from his Pentecostal views, would have all non-liberal theologians on
his side. But Watchman Nee's teachings are, as far as I know, mainly out
of existence. I have never heard anyone speaking about the teachings in
his book \emph{The Normal Christian Life} \#tags/books in the last three
decades. \#todo/opzoeken in hoeverre is zijn leer nog populair, geldig
en wat is zijn leer eigenlijk? Now the fourth group might have all of
the former except for the calling to preach. It is like Martin
Lloyd-Jones explained in his book \emph{Preaching and Preachers}
\#tags/books about a preacher who did not feel the calling and no matter
how great his sermons, he did not feel the need for it. \#todo/opzoeken
Waar? Now there are also a great number of fake preachers who most
definitely have not received God's calling to preach, but they would not
even meet the former criteria, so we will leave those aside. Or perhaps
put otherwise, I cannot think of an actual example though, but a
preacher who is not called by God will not see the fruit of the Holy
Spirit.

Now those who actually preach the good news, even their ugliest part,
their feet, are considered as beautiful when the news they bring is
good. Of such greatness is the message that it makes all things
beautiful. Indeed, it is the Gospel of Jesus Christ that makes preachers
like Paul, who were most likely of an unattractive stature, just like
Jesus Himself according to Isaiah 52:14 and 53:2-3---although that could
also be referring to His appearance after He was flogged. Read this
\href{http://www.bergsland.org/2011/06/recentposts/how-ugly-was-jesus/}{article}
for more information about this.

\emph{16 But they have not all obeyed the gospel. For Isaiah says,
``Lord, who has believed what he has heard from us?''}

Belief and faith

Now if all requirements are met, and Paul's quote from Isaiah 52:7 in
verse 15 is speaking of someone who is sent (because the messenger's
feet and shoes usually are worn out due to all the traveling) and who
preaches the Good News or in other words, the Gospel
(cf.~\href{bear://x-callback-url/open-note?id=60E171AB-2DA4-4884-A229-0B9E3EE2544A-48930-00019AFB1C102754\&header=The\%20feet\%20of\%20the\%20messenger}{Isaiah
52 (ESV) - The feet of the messenger}) then still there is a chance that
people do not \emph{believe} what is being said. Note that Paul does not
make a distinction between obedience and faith. He puts a direct
correlation between the in verse 16 like this: ``they have not \ldots{}
obeyed \ldots{} {[}for{]} \ldots{} who has believed\ldots?''. So anyone
who believes will obey. As we know, belief in Christianity does not mean
only to believe in something that one has not seen before as in Hebrews
11:1, but what this verse means, is that it is not seen by the eye. For
Moses has seen God, albeit not God Himself, for no man can see Him and
stay alive, so can we say then that Moses did not have faith? Abraham
clearly heard God and Isaiah saw Him in a vision, can we say both of
them did not believe? Of course not. They believed the promise God gave
them without the promise actually being there. We even have the Holy
Spirit as a deposit \#todo/opzoeken waar? , we have hundreds if not
thousands of documented miracles in hospitals, but also in the form of
eye testimonies all across the world, let alone in our own churches; we
have thousands or arguments and proofs for the veracity and historicity
of Jesus and His claims and still our belief is counted as faith for the
very reason that we did not see Jesus nor His promise with our eyes.
When something is seen with the eye or---in Thomas' case---felt by
touch, it is not faith anymore. Thomas did not merely believe Jesus was
raised from the dead now, he \emph{knew} it. We cannot know in the sense
like Thomas did, but we believe it according to the definition of
Hebrews 11:1.

\emph{17 So faith comes from hearing, and hearing through the word of
Christ.}

The verses above and below are a logical deduction of what we see in the
world. Obedience should follow from faith, and faith comes from hearing
and have heard His Word. Could it be that some have not? No, that is not
possible, because God has said that his Word has reached the end. But
what about those who have never opened the Bible then? Note that this is
not the same question as the one posed in Romans 1-3. This is about who
has \emph{heard} the word of Christ, not who has \emph{read} it.

\emph{18 But I ask, have they not heard? Indeed they have, for}
\emph{``Their voice has gone out to all the earth,\emph{ }and their
words to the ends of the world.''} \emph{19 But I ask, did Israel not
understand? First Moses says,} \emph{``I will make you jealous of those
who are not a nation;\emph{ }with a foolish nation I will make you
angry.''}

So can we say then that Israel has an excuse because they did not
understand? We know that many people in Jesus' time could not read and
only had the scribes' and their interpretation as their source of
Biblical truth and the \#todo/opzoeken/engels overleveringen, zoals
Hadith. Paul, however, does not even answer this question in verse 19
and simply continues to the next lines quoted from Deuteronomy 32:21
about Israel's punishment. Has it come true by this time? Right now,
most certainly, we can say that Israelis jealous of other countries that
have sovereignty, while Israel, which has one of the oldest documented
civilizations and the right to their land, is almost unanimously being
expelled from her own country, reprimanded by the same countries in the
United Nations for doing what is normal in the situation of attacks from
neighboring countries, and even being told which city should be the
capital. President Trump was even being \#todo/opzoeken/engels
slandered? niet onder druk gezet, maar negatief in het nieuws gezet, for
moving the American embassy from Tel Aviv to its rightful place,
Jerusalem, which is in itself not only legal but also the rightful thing
to do because of the sovereignty of a nation. Now those who are not even
a nation, such as Palestine, which is a group of territories spread all
over Israel and the Gaza strip, with people in and outside of the
settlements. Perhaps Israel does not actually feel jealous, but I can
imagine that she is. The Palestinians do not have a country, do
everything they can to incite Israel such as attacking first, hiding
Hezbollah within their people and sacrificing innocent people and
children in hospitals by putting them in the locations they knew was
going to be bombed, they put martyrs with bomb vests in city centers to
bring people down to hell---and still, the world cheers them on as if
they are the heroes and they reprimand Israel for starting the fights.
Some jealousy might be at play (from Israel's side).

The last line about the foolish nation could have been about Palestine,
which is utterly foolish. They have the best technology and most
peace-loving country right next to them or literally \emph{around} them
(Palestine consists of pieces of area within and next to Israel). They
are taunting one of the most advanced military and economic powers
\#todo/opzoeken/engels grootmachten right next to them, without any
country in between to protect them. If Egypt were to help them from
being slaughtered by the Israelis, Egypt would never make it in time.
What more is foolish about this country? There is too much to say.

But even if these verses are not about Palestine---and there are more
nations who are not actual nations and are foolish and are all better
off than Israel---the geste of the verses is still true. God punishes
Israel for her deeds, for her atrocities with other gods and her
ungratefulness towards God Himself, by not only prostituting herself
with idols and child sacrifices, but also by willingly declining God's
help at the point of near destruction by an enemy country, and still
hoping that either another country would partner with them or that they
could bribe the oncoming enemy.

\emph{20 Then Isaiah is so bold as to say,} \emph{``I have been found by
those who did not seek me;\emph{ }I have shown myself to those who did
not ask for me.''} \emph{21 But of Israel he says, ``All day long I have
held out my hands to a disobedient and contrary people.''}

Boldness surely was needed for Isaiah. For a prophet of God for His
people to say about them that God did not want them anymore, some would
expect humiliation, that people kneel down out of fear of being left by
God, such as the Ninevites did. But the Israelites hardened their hearts
even more and eventually killed Isaiah. On top of that Isaiah even says
that God has now found other peoples, those who did not even seek God.
Now this was for the Israelites utter blasphemy---although I do not
understand the fuss, for they did not want this God anyway---the fact
that an unclean nation would rob them of their inheritance given by
Abraham and Moses. The Israelites were proud and stubborn. There was
something they did not want, but they did not want anyone else to get
their hands on either. What is the logic in this? They knew their
lineage, it was from Abraham, everyone acknowledged that, I suppose.
They wanted all the blessings God wanted to give them of a fruitful land
and eternal life, and a ruler---the Messiah---but they did not want to
give up everything earthly in order to gain what is heavenly. They
wanted both, as so many Christians do these days.

Now, who are those nations who did not seek for God and how could they
have found God when they were not seeking Him? I know the Chinese used
to believe in a god of whom they only knew he was the highest (in
Chinese \emph{shàng dì} or 上帝), perhaps they knew about Him from the
descendants of Noah after the fall of the tower of Babel. They forgot
about Who \emph{shàng dì} actually was and worship was limited. So the
Chinese were not actually seeking Him in worship, because they already
sort of knew Him. God, however, showed Himself to them through another
way. First through the way of astrologers who discovered the first star
that indicated Jesus birth, the star of Bethlehem, that stayed at the
sky for an approximate 70 days, and afterwards they saw the second star
that guided them to Bethlehem which took them two years so that by that
time Jesus was already a toddler. \#todo/opzoeken reizigers deden er een
paar maanden over om bij Herodes te komen, maar Herodes kwam er na twee
jaar achter dat de wijzen nog niet terug waren? Bleef Jozef dan in
Bethlehem wonen en waarom? Afterwards, in the 19th century, God showed
Himself again to the Chinese through a series of missionaries, not
beginning with Hudson Taylor \#tags/famouspeople but bringing about a
great deal of new missionaries after him, through his books and the
adventures, if you will, described in it. Later, when the missionaries
fled or were killed before, during and after the Boxer's revolution, the
number of missionaries decreased drastically, but the number of Chinese
missionaries grew intensely, either in prison or in underground
churches, up until today. Another one of those examples is India, who,
however, has never been seeking God in any way. In hindsight they might
have retrofitted Jesus and Buddha as one of the reincarnations of one of
their gods, but it was never anything they were seeking nor following,
i.e.~obeying the rules of. Here too God worked through missionaries to
bring about the greatest revival in history, together with China, that
is still ongoing, with more radical Christians than there are in the
West! In the United States 30 percent of Americans claim they are
Christian, but usually only half of those actually are Christian and
only a tenth of this 30 percent, is radical, which makes about 8 million
people---the number of Christians willing to die for their faith---much
less than the number of Christians in India or China.

\#biblestudy/devotionals/romans \#tags/events/pentecost

\hypertarget{romans-11-esv}{%
\section{Romans 11 (ESV)}\label{romans-11-esv}}

\textbf{The Remnant of Israel} \emph{1 I ask, then, has God rejected his
people? By no means! For I myself am an Israelite, a descendant of
Abraham, a member of the tribe of Benjamin. 2 God has not rejected his
people whom he foreknew. Do you not know what the Scripture says of
Elijah, how he appeals to God against Israel? 3 ``Lord, they have killed
your prophets, they have demolished your altars, and I alone am left,
and they seek my life.'' 4 But what is God's reply to him? ``I have kept
for myself seven thousand men who have not bowed the knee to Baal.'' 5
So too at the present time there is a remnant, chosen by grace. 6 But if
it is by grace, it is no longer on the basis of works; otherwise grace
would no longer be grace.}

So God has not rejected His people, but He has kept a remnant, chosen by
grace. It is clear that this remnant consists of the Judahites and some
Aaronites, known by the name Cohen. The rest of the eleven tribes have
been scattered across the world (see notes in Romans 10). A small
remnant remains, Judah, but out of these Judahites also only a small
percentage is saved. The interesting thing is that the Jews (and by now
they are called Jews and not Israelites, because the origin of the word
jew means that they are from Judah) who have held on to God's Law, that
is the orthodox Jews, are usually not the ones to convert to Christians
and become Messianic Jews. No, it is rather the secular Jews who are
guided and chosen by God, which again shows God's grace and not works is
what saves man. Examples of these are Michael Brown, Jonathan Caan
(though being raised in an orthodox fashion they rebelled and became
secular) and Andrew Klavan \#tags/famouspeople who wrote extensively
about his conversion in his book \emph{the Great Good Thing}.
\#tags/books

On one hand I am relieved to hear that God has not rejected His people,
not all of them at least, on the other hand it is such a big part that
has been rejected. Eleven tribes out of twelve, and of this last tribe
(or perhaps two, including Benjamin) a part---I do not know how big
\#todo/opzoeken ---remained in Persia among which Daniel and Esther (a
Benjaminite). The percentages of secular and orthodox jews are unclear
to me as well, let alone the percentage of Messianic Jews. But one thing
strikes me: why would the name Cohen still be in existence and in quite
a number? For there are relatively many famous people whose surname is
Cohen, such as former mayor of Amsterdam, the Netherlands, Job Cohen,
and the American politician \ldots{} Cohan, \#todo/opzoeken which would
indicate there are even more non-famous people. One thing comes to mind
as a possible explanation, which leads to a second explanation of how we
have not only Jews from the tribe of Judah. First, the Levites were
assigned cities of refuge among each of the land of the tribes. There
were forty cities designated for the Levites, so Benjamin and Judah
would have about seven cities full of Levites being deported to Babylon
along with them. \#todo/opzoeken cf.~Joshua 22. Now where do the other
Jews come from? At the end of Judges the Benjaminite tribe was almost
extinct, except for a couple of hundred Benjaminite \#todo/opzoeken men.
They were being told to steal the women of other tribes. I am not sure
whether these women were from one tribe only or of multiple tribes. This
would explain DNA from other tribes than Judah in present day Jews, but
not any surnames. Either way, God created for Himself a remnant which
exists of more than only Judahites, but of Benjaminites, Levites and at
least one other tribe.

\emph{7 What then? Israel failed to obtain what it was seeking. The
elect obtained it, but the rest were hardened, 8 as it is written,}

The elect obtained it and at the same time the rest were hardened. God
does not leave any space in between for people who are not hardened or
near believing. See also the notes on verses 27 about partial hardening,
which is about temporal hardening and of certain people, but not on an
individual level. In practice we see, however, that there are people who
are close to coming to faith. They are rational and although they might
be unwilling to becoming a Christian, they are generally speaking, in
human terms, good people. Usually these are people who lack knowledge of
what religion is and have many misperceptions that need to get rid of.
But what about those who have read more about it? What about people like
Ben Shapiro and Dave Rubin? Ben Shapiro would most likely fall under the
hardened category, even though he believes in God, is morally righteous
and seems to make decisions on a rational base. However, he does not
seem to have changed his mind any bit even after speaking with famous
Christians such as Ravi Zacharias, William Lane Craig and John
MacArthur. \#tags/famouspeople

\emph{``God gave them a spirit of stupor,\emph{ }eyes that would not
see\emph{ }and ears that would not hear,\emph{ }down to this very
day.''}

It is clear that none of the Jews were saved by merit. The Jews despised
Jesus and killed Him. Now of course some did not kill Him, like Mary and
other women. It is \#todo/opzoeken/engels frappant to see the it is the
women who seem to know more about Jesus' death and resurrection, did not
run away and were the ones visit and believe Him first. But even besides
these women there must have been people who stood up for Him, even one
of the Sanhedrin stood up and said that there was no evidence against
Him, at least not enough to indict and prosecute Him. These people, no
matter how righteous, still fall under grace, not only in the sense that
it is impossible to remain sinless one's whole life and that Jesus
therefore had to pay with His life for ours, but also being part of
remnant to be saved.

Now the remnant is, as mentioned before, the part of the people saved by
salvation, part of the people who returned from Persia, part of the
tribes who were taken in captive to Babylon, part of country that had
not been taken by the Assyrians and dispersed, part of those who had
been kept alive or who survived.

\emph{9 And David says,} \emph{``Let their table become a snare and a
trap,\emph{ }a stumbling block and a retribution for them;\emph{ }10 let
their eyes be darkened so that they cannot see,\emph{ }and bend their
backs forever.''}

As mentioned, the elect were given eyes to see while the rest was made
blind. Would that not be unfair? Suppose the whole world was not blind
but merely had their eyes closed. Would it not be unfair then to say
that God would open the eyes of some and forever close or blind those
who did want to open their eyes? The latter is exactly the point. The
people have always had the choice to open their eyes if they had wanted
to. Of course they did not, for none wanted to (cf.~Romans 3:10-17). Is
that then not unfair, that no man is inherently capable of opening or
wanting to open his eyes? No, still that is a righteous act. For indeed,
it is because man is born out of a sinful person who in turn was born
out of sin, up until Adam and Eve. But it is not merely our ancestor's
fault for if we had been in that position, we would have listened to the
snake, eaten of the forbidden fruit and sinned as well. So our
\emph{inherent} sin might as well be called \emph{our} sin. So
concluding, we can see that the fact none of us inherently wants to open
our eyes is \emph{our own choice}. Now to answer our first question, is
it not unfair to take this capability away to ever open our eyes by
blinding us, or as Paul says it, to \emph{harden} our hearts? In the
eyes of man it seems unfair because we think there is always a
\emph{possibility} that man would open his eyes, but God does not work
with possibilities, rather He only knows of certainties. He knows which
man will ever open his eyes and which man will not. Therefore, if He
takes away one's sight, it would not matter anyway in hindsight (no pun
intended). \#church/material And for those whose eyes He opens, it is a
gift, by grace, as is everything from God, to see the truth and accept
it. What do we say then of people like Bart Ehrman and Michael Shermer,
were they not on the right path before they wandered off? Perhaps they
were. I would say that these persons received the grace to open their
eyes for a short time, but then closed their eyes to the truth because
they had preferred seeing their own truth. God told Shermer (most likely
and if I remember correctly) that what He did was righteous, even if
people were killed and if anything unrighteous should happen, it is not
Him but the Satan who did it. Why God let the Satan go His way, is
because He knows better what will happen in the end and because He is
sovereign. We should not impose our definition of righteousness on God.
\emph{Righteousness} is but a human-made word, the concept of it is
heavenly and encompasses more than a human dictionary could
contain---let us say it is situation specific, so it would be impossible
to define righteousness for every situation. So Shermer could not live
with the fact that God's definition was different and broader than
his'---or no one explained it to him, but that would then also mean that
God did not extend His grace to him by sending him a discipler who could
teach him this---which means he was not able to reconcile the atrocities
on Earth with God's almight and righteousness, unless God is not
almighty or righteous. Not being \emph{able} to reconcile something is
one thing, it is another thing, however, to turn your back on God. There
are many Christians who have not been able to reconcile this difficult
question of life. As Ravi Zacharias \#tags/famouspeople mentioned, it is
one of the most asked questions during the Frequently Asked Questions
part after his talks. Even though Zacharias has a verbal answer to this
question, he understands that it is difficult to comprehend. But what we
should always do, is give God the benefit of the doubt and put ourselves
under Him, acknowledge that He is wiser and mightier still. It would be
the only logical thing to do as well. For even if one does not believe
God is almighty or righteous and that His creation has gotten out of
hand, He still is more powerful than all of us and could at least
control our lives. Especially when a god is unrighteous it would be wise
to follow him, for what would this unrighteous god do with your life
when he finds out you are not following him anymore, like the Egyptian
or Canaanite gods did? He would be like a mob or maffia boss would
punish you for leaving the group (which is unrighteous). So the fact
that you dare to turn your back on God and that you have not been
punished for it only means you acknowledge His gentleness and
righteousness. Now Shermer explicitly claims God does not exist and it
appears from his debate with Lawrence Krauss against Dinesh D'Souza that
his falling away \#todo/opzoeken/engels tegenovergestelde van bekeren is
based on emotions which is made even more clear when he shows his
understanding of people who experience miracles or even God Himself are
not lying---He even seems to be open to the results of miracles that
have happened in history, such as people being healed from cancer,
provided that they are natural phenomena which simply have not been
explained yet. Ehrman on the other hand does not make this claim and
does not even say anything about it in his famous debate with William
Lane Craig about the resurrection of Jesus Christ, when Craig mentions
this fact (that Ehrman has never denied that God exists). Craig also
mentions in his
\href{https://www.reasonablefaith.org/media/debates/is-there-historical-evidence-for-the-resurrection-of-jesus-the-craig-ehrman/}{opening}
that Ehrman ``abandoned his Christian faith'' after his studies due to
the issue of errancy of the Bible, which he thinks---I presume---must be
\emph{inerrant} before he could believe in it or in the author of it.
This, however, is also a proof of putting oneself over God. Just like
Shermer was unwilling to accept a truth from God and that his truth had
to be right one, so Ehrman is unwilling to accept that God could allow
errors to exist in the Bible and that \emph{his} claim of a perfect God
should not only be able, but also should create a perfect Bible, should
be true. Once again the fallacy is made here of imposing our definitions
over God's sovereignty and---even leaving God's sovereignty aside---His
infinite wisdom. One reason why God could have left errors in the Bible
is because He did not want to restrict people of their free will. It is
an enormously hard job to maintain free will and maintain the Bible. As
has been told in a few legends or stories, when the protagonist of the
story had hidden his treasure in a tree trunk marked with a handkerchief
knotted to a branch and made his enemy promise he would not remove the
handkerchief, the enemy would spend the whole night knotting the same
type of handkerchiefs on other tree trunks so that the protagonist could
not find the original anymore. In the same way God has protected His
Word and put a curse on those who added or removed any word from it, be
it from Revelations or (as most interpret it) for the whole Bible, and
perhaps those people have actually been cursed, but the fact is that
there multiple versions of the Bible. What God could do then, is to at
least let His people know which version is true. A simple god would then
point to a specific version and letting His Holy Spirit speak to those
who have authority on Earth to indicate the correct version or the
erroneous verses. Our God, however, is more complex and has a better
way, and tells us which parts of the fakes Bible are wrong and why they
are wrong, and which parts have been added but are allowed to stay in
because they have been added with good intentions and make the text
clearer (for future generations for example). This makes its
authenticity harder to trace, for example when in Moses' books in
Numbers \#todo/opzoeken there is written that Moses is the most humble
or meek man in the world, which most obviously a meek person would never
write of himself, but it does make clear why Moses did not speak against
Aaron and Maryam. This is merely an explanation, but it shows that there
is a much greater truth behind what humans can think of. Perhaps a fun
fact to know, is that Ehrman
\href{https://ehrmanblog.org/ehrman-vs-craig-evidence-for-resurrection}{writes}
of this debate, ten years later, that ``maybe'' Craig did mop him up. I
hope he has gained some insight after reading the transcript of his
debate to see that he did not answer Craig's arguments and misses some
of the strong arguments made by him due to his strong focus on bringing
down the resurrection instead of being critical of the truth. Without
knowing I have commented on verse 8-10. The religious and Biblical
critics Shermer and Ehrman have blinded themselves and put a snare and
trap for themselves with their arguments against a living and almighty
God.

\textbf{Gentiles Grafted In} \emph{11 So I ask, did they stumble in
order that they might fall? By no means! Rather, through their trespass
salvation has come to the Gentiles, so as to make Israel jealous. 12 Now
if their trespass means riches for the world, and if their failure means
riches for the Gentiles, how much more will their full inclusion mean!}
\emph{13 Now I am speaking to you Gentiles. Inasmuch then as I am an
apostle to the Gentiles, I magnify my ministry 14 in order somehow to
make my fellow Jews jealous, and thus save some of them. 15 For if their
rejection means the reconciliation of the world, what will their
acceptance mean but life from the dead? 16 If the dough offered as
firstfruits is holy, so is the whole lump, and if the root is holy, so
are the branches.} \emph{17 But if some of the branches were broken off,
and you, although a wild olive shoot, were grafted in among the others
and now share in the nourishing root of the olive tree, 18 do not be
arrogant toward the branches. If you are, remember it is not you who
support the root, but the root that supports you. 19 Then you will say,
``Branches were broken off so that I might be grafted in.'' 20 That is
true. They were broken off because of their unbelief, but you stand fast
through faith. So do not become proud, but fear. 21 For if God did not
spare the natural branches, neither will he spare you. 22 Note then the
kindness and the severity of God: severity toward those who have fallen,
but God's kindness to you, provided you continue in his kindness.
Otherwise you too will be cut off. 23 And even they, if they do not
continue in their unbelief, will be grafted in, for God has the power to
graft them in again.}

Note once more that Paul does not say that if the Jews continue with
their good works that they will be grafted in again, but only when they
do not continue their \emph{unbelief}, that is not of works but of mere
faith in God. What, then, does this mean for the Gentiles who have left
the faith, can they be grafted in again? We have established above that
this chapter does not contradict the \emph{Once Saved Always Saved}
principle. The Gentiles were grafted in, indeed, but it is the fruit,
i.e.~the result, that shows whether they have truly been grafted in.
\#biblestudy/theology/calvinism/osas My father, for example, ``grafted''
branches from other trees onto his apple tree, but using ropes and tape.
Of course this did not work, because the inside of the branches were not
connecting in the right way with the trunk. On the other hand, even if a
branch connects correctly, it can still die because of other reasons,
perhaps certain trees are not compatible or require the right season or
temperature. When the branches have been grafted in and when the inside
of the branches intertwine and the branches become one with the root,
however, then the branches will not die as long as the root is strong.
\#todo/opzoeken waar staat het stuk over mensen die het geloof verlaten
hebben en dat ze niet meer kunnen terugkomen? Dat kan toch niet, want
iedereen kan toch komen? Terugkomen zou per definitie niet moeten
bestaan, omdat iemand niet kan afvallen van het geloof. Maar ``terug''
komen in de zin van ``iemand is al een keer erin gegraveerd, maar het
was de vorige keer niet gelukt'', in zo'n geval zal God het niet
nogmaals proberen---dit zou de betekenis van dat vers kunnen zijn.

\emph{24 For if you were cut from what is by nature a wild olive tree,
and grafted, contrary to nature, into a cultivated olive tree, how much
more will these, the natural branches, be grafted back into their own
olive tree.}

The Jews, however, will be grafted into their own tree. They differ from
the Gentiles, first of all, because they are God's children. It would
not be unfair for God to take back His children after He has abandoned
them temporarily from His home. A father can do to His own children as
he likes, especially God the Father. On the other hand, even though
these branches are originals, from the root of Jesus Christ, they had
never borne fruit. So in that sense they are not actually grafted
``back'' into the tree, for the first time they were like dead branches
being grafted in the wrong way.

\textbf{The Mystery of Israel's Salvation} \emph{25 Lest you be wise in
your own sight, I do not want you to be unaware of this mystery,
brothers: a partial hardening has come upon Israel, until the fullness
of the Gentiles has come in. 26 And in this way all Israel will be
saved, as it is written,} \emph{``The Deliverer will come from
Zion,\emph{ }he will banish ungodliness from Jacob'';} \emph{27 ``and
this will be my covenant with them\emph{ }when I take away their
sins.''}

Because the Jews did not listen, the Gospel was spread to and received
by the Gentiles. But God gave the Jews only a \emph{partial} hardening.
Partial in the sense that only a part of the Israelites were hardened,
and in the sense that of those hardened, one part (which is, I think,
all tribes except for Judah and Benjamin) was completely
hardened---perhaps irreversible---and another was only partially
hardened. Another interpretation is that this concerns Israel as a
whole. The part that is hardened has been completely hardened,
regardless of the tribes, and some have been elected in order to be
saved. Also, Paul clearly says that this partial hardening is of a
temporal sense, it is only \emph{until} the Gentiles have reached their
fullness---which probably means the moment when the people that God has
elected have converted---that God wil unharden the hearts of the Jewish
or Israelite people. In the case of non-Judahite people, they will need
to hear the Gospel first. It is striking to see that the Judahites and
Benjaminites are highly educated and though poor are still very
civilized. The other tribes that have been found, scattered all over the
world, are living in bushes or as shepherds, do not even know they are
Jewish or know about God. I once heard from a minister about Jews being
discovered in Iraq who did not know they were Jewish. It appeared then
that after the Second World War \#tags/historic events/wwii\# they
escaped to the Middle-East and the Muslims let them live there in peace,
provided they would become Muslims. They then did not command the Laws
of God anymore and did not teach their children to obey the Lord Yahweh
of Israel (cf.~Deuteronomy 5). Perhaps this is also what Paul refers to
when he mentions the ``unbelief'' of the Jews (cf.~verses 20, 23).

\emph{28 As regards the gospel, they are enemies for your sake. But as
regards election, they are beloved for the sake of their forefathers.}

Paul is speaking of the Jewish people when he uses ``they''. The
non-Messianic Jews are indeed the enemies of the gospel for the proclaim
their own gospel of the Law and think that that will save them. Some
Christians I know have been pulled a little backwards by the Jews. They
admire the Jews so much, like I do as well, because they are God's
children, blessed with intelligence and a plethora of talents in all
arts and sciences. These Christians I know have then stopped eating pork
and have started eating kosher food. Fortunately they have not been
changed in their thinking related to the Gospel. Regarding election,
however, we should be grateful of the jews. It is because of them that
we are saved. Not because of their merit, but because they were crucial
in God's plan to save humanity. Also, they are beloved by God, so we
should love them. \#biblestudy/theology/predestination/election In this
time there is not a lot of respect for whom or where you are from, only
for who you are. This is good in some sense, a person should be judged
by his merit and character, not for his \#todo/opzoeken/engels stand
(van adel bijvoorbeeld) or incluence. On the other hand, however, it is
important that we realize that whom a person is from, or sent by, is
also the one whom we should fear. A messenger sent by a king should be
treated better than when he were sent by a farmer, when one fears the
king more than the farmer. So the Jews themselves we do not fear, but
their Father we should, and those who have not, have seen the
consequences of that.

\emph{29 For the gifts and the calling of God are irrevocable. 30 For
just as you were at one time disobedient to God but now have received
mercy because of their disobedience, 31 so they too have now been
disobedient in order that by the mercy shown to you they also may now
receive mercy. 32 For God has consigned all to disobedience, that he may
have mercy on all.}

God is merciful to us, that is one thing that is certain from these
verses. When He gives something to someone it is irrevocable. One cannot
give it to someone else or throw it away. God had called the Israelites
to be a light to the nations, they were not, and so God has send the
Christians to be that, but it does not mean the Israelites can now rest
and let the Christians do the work. Their calling is still the same and
will always be. Their gift is what they have received with which they
can put their calling in practice. It includes riches, fame, wealth and
intelligence---as I have earlier noted of Jews, even though there are
many poor ones as well. Their gift, even though they are using it for
bad purposes---Jewish Hollywood actors like Natalie Portman, Barbra
Streisand and Jewish politicians such as Michael Bloomberg and Bernie
Sanders promote Leftist ideologies, to name but a few
\#tags/famouspeople , is not taken away from them, exactly because of
this reason: it is irrevocable.

\#todo/opzoeken Number of Jewish people in the Avengers (except for
three) Mark Ruffalo?

What should we say then of the gift of eternal life? Is that gift not
greater? Should it then not be even more irrevocable? Yes, the gift
itself is irrevocable for sure, but it can be used for the wrong
purpose, that is no condition. \#biblestudy/theology/calvinism/osas
However, do note that the consequence of the gift of eternal life is the
Holy Spirit and that God promises us that it is the Holy Spirit Who will
work in us (cf.~Philippians 1:6, Matthew 28:18, \#todo/opzoeken waar nog
meer?) and change us. so if anyone were to misuse his gift of eternal
life, would God then not be keeping His promise? This rhetoric question
must be answered with \emph{yes}, but some may say that person still has
free will, even when in possession of the Holy Spirit. That is true, but
the free will we have before and after obtaining the Holy Spirit is
different. Before we had it, we would use it in a selfish way. Paul
clearly says that although he is allowed to do everything, not
everything is beneficial and therefore he will not do everything.
(\#todo/opzoeken referentie) Anyone who keeps sinning therefore proves
that he does not have the Holy Spirit (\#todo/opzoeken referentie) Also,
God has set the condition for receiving eternal life, which is to have
the Son, Jesus Christ, which means of course that one should love Him
(cf.~1 John 5:13) and if anyone love Jesus he would not keep sinning
against Him and misuse this gift. What then about people who do not love
Jesus anymore? They might have loved Jesus in the beginning, received
the Holy Spirit, which they cannot discard anymore, and then misuse this
gift? If that were the case, the Holy Spirit would continue knocking on
the door of their hearts (\#todo/opzoeken not Rev.~3:20 but where?) and
speak to their conscience, \#todo/opzoeken/engels aanklagen and it
\emph{would} have effect for God is all-powerful. Would it not be ironic
if God were all powerful, yet He could does not even have the freedom to
give only loyal people eternal life, especially because He gave them
free will? This of course is a rhetorical question and the answer is
\emph{yes, it is ironic} but it is not true. God is by no means
compelled \#todo/opzoeken/engels gedwongen to give people eternal life
just because they love Him for only a short period. God did not make the
promise to give everyone the free gift of grace in order to regret it
later on. No, when He makes a promise He will think clearly about it.
Granted, the following contains a little but of speculating with human
logic, but it is most intuitive that He either chooses those who really
love Him and will continue loving Him the rest of His life, just like
when one chooses his bride, or that He will help those who start
stopping loving Him to love Him more.

\emph{33 Oh, the depth of the riches and wisdom and knowledge of God!
How unsearchable are his judgments and how inscrutable his ways!}
\emph{34 ``For who has known the mind of the Lord,\emph{ }or who has
been his counselor?''} \emph{35 ``Or who has given a gift to him\emph{
}that he might be repaid?''} \emph{36 For from him and through him and
to him are all things. To him be glory forever. Amen.}

As mentioned, there was some speculation in my explanation on my
definitive position \#todo/opzoeken/engels standpunt on the theory of
\emph{Once Saved, Always Saved}. Verses 33 and further emphasize that
the depth of God's knowledge is even inscrutable. God would know about
subjects we never even thought of, let alone name the contents. We think
we have covered every subject on the current world, for what would be
smaller than quarks, but even within the subject matter of the present
time, God could show things we have never seen. Even the simplest cell
has not even been thorough searched through; Even the cracking of
fingers cannot be explained by current medicine!

\#biblestudy/devotionals/romans

\hypertarget{romans-12-esv}{%
\section{Romans 12 (ESV)}\label{romans-12-esv}}

\textbf{A Living Sacrifice} ::\emph{1 I appeal to you therefore,
brothers, by the mercies of God, to present your bodies as a living
sacrifice, holy and acceptable to God, which is your spiritual
worship.}::

\#church/material \#todo/nogaftemaken Logisch, wat is dat? dat volgt uit
11:36, dat het logisch is dat je God eer geeft door het meest kostbare
te geven van jezelf. Niet een onderdeel van je lichaam, maar alles.
\textbf{Voorbeeld.} Als je heel erg moet plassen, ga je\ldots{} een kop
koffie drinken. -\textgreater{} dat is niet logisch.

Note that presenting our bodies, though physical they are, is
\emph{spiritual} worship. It is not \emph{what} is being used that
defines the worship, but \emph{how} it is used. We can use our skills in
the office to make money for ourselves in a selfish way or to make money
to give to others. We can use the same skills to help growing companies,
even if these are not charitable ones, even when getting paid commercial
tariffs for it, and still honoring God, by delivering good work. It is
the last part of the sentence that makes it spiritual, that is holy and
acceptable to God. Now what does it mean to be \emph{holy} and
\emph{acceptable}? If something is holy it automatically is acceptable,
but does it go the other way around as well? I think that being
acceptable is what God accepts from us, which is, by definition due to
our sinful nature, never fully holy---for a sinful person cannot bring a
sinless sacrifice. Of course Jesus Christ died for us and therefore we
can bring Him gifts which have been cleansed by Him. It is only for this
reason that God can accept our gifts, such as our praise and our works.
But initially, God wanted us to be holy and then sacrifice not only our
gifts, but our \emph{bodies} themself. Now it is of no use to sacrifice
a dead body to God. God can make the dead rise again but except for
showing that He can do that and by this showing that He is God, for only
God can raise the dead, there are no other examples in the Bible where
God uses the dead. The living, however, are much more useful, once they
have been cleansed. Maria of Magdala for example was cleansed from seven
demon spirits \#todo/opzoeken and after that she devoted her life to
Jesus Christ and became His follower. Paul as well had to be cleansed.
First of all, his eyes had to be cleansed, which was literally yet
symbolically done when the scales fell off his eyes. He received new
eyes with which he could see that what he was doing was wrong.
\textasciitilde Holiness entails first of all seeing what is good and
wrong.\textasciitilde{} \#todo/nogaftemaken

Spiritual worship entails * Living sacrifice * Voluntary * It needs to
cost something * Holy * It needs to be clean * Acceptable * It needs to
be something God wants

\textbf{Voorbeeld.} Laat kinderen hun handen vies maken en daarna mij
een schoon A4-papiertje geven. Het lukt niet om iets schoon te geven als
jezelf vies bent.

So in order to be holy, we first need to know what is. Our eyes need to
be cleaned. Then, in verse 2, even after being cleaned, we read that we
must change in a way to stay clean. Now that your eyes are clear you can
see which things are unclean. What will you do then? You must stay far
away from them and not conform to them but rather let yourself be
transformed by God. Why, because only after being transformed do we know
what is good, acceptable and perfect. There are many things in this
world that we see as good, but that which is good, is not always the
will of God, it may not be acceptable to His standard and it may not be
perfect. Now nothing is ever perfect in this world, but if out of two
choices one choice must be made, one choice is the good one, no matter
the consequences, and that is the perfect choice. Take for example the
story of the judge \#todo/opzoeken Gibeon? He made an oath to sacrifice
to God whomever would be running out from the gate after his victory
over the \ldots{} . It turned out to be his daughter. Now, no matter who
it had been, Gibeon would have to break either his oath which he swore
to God, or he would have to break the seventh \#todo/opzoeken
commandment not to murder anyone (although it is doubtful whether this
was murdering or killing, either way, it was not a righteous kill). It
is (intuitively) clear that God does not want anyone dead for this
reason and breaking the oath would have been the best choice. Better
yet, it would have been the \emph{perfect} choice, as there was no other
choice. In reality our choices are not as clear cut or binary as
Gibeon's, unfortunately, but we can trust in it that God has His perfect
ways, just like Jesus knew the perfect answers to the Pharisees' trick
questions and traps. Solomon knew the perfect answer to the two
prostitutes who were fighting over a child in order to determine the
real mother. By transforming our mind we will also know the will of God
and know this perfect answer. Now about acceptability, it is more than
just something that is good. It is something God wants and it must be
above a certain standard. Being clean is one of those standards, but it
is not sufficient. A menorah for example needed to have seven arms and a
hanukah nine. Even if it were \emph{kosher}, i.e.~it was completely made
of gold, by a Jew, with legal money, not touched by heathens or cleansed
with the fire after it had been touched, then still it would not be
acceptable if the candle stand only had six arms. God's standard are
perfect, it needs to be done exactly the way he wants.

::\emph{2 Do not be conformed to this world, but be transformed by the
renewal of your mind, that by testing you may discern what is the will
of God, what is good and acceptable and perfect.}::
\#biblestudy/memorization/todo

The ruler of this world wants to transform us into anything but what is
said in the first verse. God wants us to give all of our lives to Him,
and not merely give it, He wants the gift to be holy and acceptable. And
still not only does He want that, He wants it to be voluntary. Then,
when we have given it, even though our gift right now is not perfect, we
continue perfecting this gift. It sounds like when one gives another a
car as a gift, and afterwards the giver starts testing what is good for
the car. Should that not be done beforehand? Yes it should, but the
problem is that the giver himself does not even know what is best for
the car, because he did not build the car. In the same way we cannot
give ourselves in a perfect way, because we do not understand ourselves
perfectly nor have we tested everything. Now this car, if it has any
settings, should not be conformed to this world's settings, but rather
to the one who is going to use it. Likewise with our bodies, they should
be conformed to the Father Who wants to use it for His purposes, whether
it is destruction at an early age with only a short period of
evangelization and Islamic apologetics, as in the case of Nabeel Qureshi
\#tags/famouspeople or for spreading the Gospel until a very high age,
like Billy Graham. Either way, it is all for His glory. The godless do
not even know, but they and their bodies too have been created for His
glory and even the evil is somehow attributing to His glory. Only the
fact that God and His works and the works of His people are in stark
contrast with the devil's work, helps the world see the devil's real
face. \#church/material

\textbf{Gifts of Grace} \emph{3 For by the grace given to me I say to
everyone among you not to think of himself more highly than he ought to
think, but to think with sober judgment, each according to the measure
of faith that God has assigned.}

Right after the first two verses, about transforming your mind to God's,
Paul explains the reason why. The answer is that we are not as high as
we think we are. If we think we do not need to be transformed, then we
should again, but this time with sober judgment. Now it is true that
need more transformation then others. What should we say then of those
who transformed so much as is humanly possible? People like David Pawson
perhaps or Rick Warren \#tags/famouspeople, both of whom I do not know
personally, seem to so friendly that even when they disagree, they can
stay friends with the other party. Those people should measure
themselves with the faith God has assigned to them. Now it is for sure
that these two people have great faith. With my faith, which is small
compared to theirs, I would see hardly any transformation needed in
them. But since their faith is greater, I would expect they will see
that there still is a lot that needs to be changed, maybe even more than
I can see in my own life.

Note that the measure of faith has been \emph{assigned} to one. Faith is
not from oneself (cf.~Ephesians 2:8-9), neither is the amount of it, but
from God. This also humbles us in knowing that if our faith is greater,
even that was an initial gift by God.

\emph{4 For as in one body we have many members, and the members do not
all have the same function, 5 so we, though many, are one body in
Christ, and individually members one of another. 6 Having gifts that
differ according to the grace given to us, let us use them: if prophecy,
in proportion to our faith; 7 if service, in our serving; the one who
teaches, in his teaching; 8 the one who exhorts, in his exhortation; the
one who contributes, in generosity; the one who leads, with zeal; the
one who does acts of mercy, with cheerfulness.}

Another way of measuring oneself and to think more highly of others (and
less of oneself) is by looking the functions we have been given. If we
have been given an honorable and visible function, we should thank God,
for it was He Who gave us this function. If someone is working in his
function, which is less honorable than yours, we should also thank God
that he is working in his function. It would be better than that he were
trying to work in another function which does not suit him. That would
not make him feel better nor the people around him. So we can humble
ourself in knowing that is was God Who gave us our faith and Who gave us
our function, and on top of that measure our faith and works within this
function by the measure of faith of we have been given. A prophet with a
lof of faith, who does not say much, but like Elijah at one time was
very anxious, is not a good prophet given the amount of gifts he has
received.

\textbf{Marks of the True Christian} \emph{9 Let love be genuine. Abhor
what is evil; hold fast to what is good.}

Genuine love shows itself in both loving and abhorring. If one loves
one, he hates the other. This goes for serving God (cf.~Matthew 6:24) as
well as loving your neighbor. Now I, personally, do not abhor certain
things, when I hear about someone cheating on his wife, while I see
other people clearly being abhorred by it, such as Liz Wheeler and even
more so Allie Beth-Stuckey \#tags/famouspeople, when they are talking
about Trump's personality (as seen in Ben Shapiro's Sunday Special).
This is then, as a logical consequence of the above reasoning, because I
do not love people and women in particular in this case, as I should. Of
course, it is something that needs to change within me, but this verse
is a good indicator of such things.

\emph{10 ::Love one another with brotherly affection. Outdo one another
in showing honor.::}

This is very big commandment. I was speaking about the latter part of
the verse but after looking again I see that the first one is very
difficult as well and possibly even more difficult. The second command
to show honor is very common thing in eastern cultures. During a
\href{https://www.rzim.org/read/rzim-global/nabeel-qureshi-debates-muslim-apologist-shabir-ally}{debate}
between Nabeel Qureshi and Shabir Ally \#tags/famouspeople Qureshi
started by honoring Shabir as the better and more experienced debater,
upon which Ally replied during his turn that Qureshi is the more
knowledgeable about the Christian God \#todo/opzoeken wat er precies
werd gezegd. This prompted a very friendly debate in which, even after
Ally said about an argument Qureshi made: ``You can do better than that,
Nabeel'' he instantly \#todo/opzoeken/engels and told the audience: ``I
am sorry {[}\ldots{]} I take that back.'' The honoring of people can
really change the mood between people and is as powerful as James
mentioned when he spoke of the power of the tongue (cf.~James 3:1-11).

Now Chinese people, for example, honor others well, but not always with
genuine feelings for each other. They lift up the other so that the
other will not lose face, and in return they expect the same.

Of course the Bible is always speaking of genuine feelings when it
commands something. But it is like there is a way to at least fake it
for others, although no one can cheat God Himself. It is just like the
Pharisees, they know they are not cheating God, but they care more about
the short term benefits, which is honor from man. The former verse,
however, cannot be faked. We must love our fellowmen genuinely, and with
\emph{brotherly} affection! This means so much as giving our life for
the other. There is no command in the Bible to love one only a little,
only as much as it will not hurt yourself, your materials or your
privacy. Does this mean then that one should allow all beggars to live
in his house and use up all of his food? Fortunately---and I say this
with shame, because I am not one of those big hearted people---we do not
have to. There is a responsibility for each person. A man's
responsibility is to take care of the safety of his family, and to
provide for them. This means that he should block each factor with
excessive risk out of the house, including certain multimedia items,
such as movies and music, but also certain people. On the other hand, a
beggar has a responsibility as well and helping him is not always equal
to loving him. A drug addict is not helped when provided with drugs or
money, with which he most likely will buy more drugs, or even with food,
if he refuses to work. Francis Chan \#tags/famouspeople once took an
ex-convict into his home, who was tall and big and whom he had just met
on the streets. He was just converted to Christianity and had no place
to live, so Francis brought him home and decided to let him live there
for a few months. This is a very risky thing to do, and Francis had his
doubts as well, he mentioned in a sermon, but he also had a feeling,
most likely the Holy Spirit, telling him that this was the right thing
to do. Ultimately, the Holy Spirit is our guidance. He can tell you to
do something utterly risky or He can tell you not to do something that
seems completely safe. Either way, what He tells you to do will be out
of brotherly affection.

\emph{11 Do not be slothful in zeal, be fervent in spirit, serve the
Lord.}

This is a whole list of things that we as Christians need to do.
Multiple commentaries explain it this way. But why would Paul, after
writing a big deal of prose, suddenly switch to listing attributes? I
know he introduces Chapter 12 with his first two verses, explaining what
we should do, but it is not as if he said ``here is a list of all the
marks of a Christian.'' In order to get the answer to this question we
need to zoom out see what Chapters 12 and further aim for. Then we will
see what Paul says the following: * Be transformed to God's will (verses
1-2) * Do not think you do not need change, you might need change even
more, depending on your gift (verses 3-8) * This is what needs to change
in you (verse 9 until further chapters)

Now these changes are grouped by pericopes and paragraphs. The pericope
starting with verse 9 is called ``Marks of the True Christian'' in the
ESV, but actually everything following even in later chapters are marks
of the true Christian. Then, the paragraph starting at verse 9 begins
with ``Let love be genuine'', to show that this paragraph is about the
marks of love, while the paragraph starting at verse 14 is about
blessing. However, this does not really seem to be the case, as verses
11 and further are about zeal, joy and prayer. In any case, these
paragraphs are made up by people, but the contents are what is most
important.

So as for being slothful in zeal. Sloth itself is a big sin, according
to Catholicism it is even one of the seven deadly sins, which are the
main sins from which every sin is derived. Do not be slothful in zeal,
be fervent in spirit, serve the Lord. I will take the sentences at face
value, though, and assume that the different elements in this sentence
are related to each other. It is ultimately about serving the Lord, not
only in this sentence but in everything we do. Whether we obey Him and
love others, is done because we serve the Lord, and we serve the Lord
because we love Him. That is why, when serving someone we love, we are
not slothful but rather zealous. We are not asleep, but fervent in
spirit. Now what does it mean to be fervent? It is the same feeling one
has when he wakes up, after having had a good night's rest, with plenty
of hours, good dreams, a good mattress and cool pillow. Fervent
according to the dictionary means ``hot'' or ``passionate'', but it is
not only about \emph{being} passionate, it is also about \emph{doing}
things here, which is why I am using the example of waking up well
because on those mornings one really feels like taking action. The
reason we do not daily have a fervent feeling is because of the lack of
the above-mentioned things. I am not talking about sleep perse, but
about our position in life. Are not many of us tired of what we do in
life? And why is that? It is because the things that are supposed to
give us rest, such as the pillow and the mattress in the example above
are not of good quality, or the position we take before sleeping was a
bad one to begin with, such as when someone lies partially straight up
to read a book and then falls asleep while reading. Spiritually the
former examples are given in the form prayer and silent time. Just like
a pillow is supposed to support your neck, which is a vital in
supporting your brains in fact, and also is it actually part of your
peripheral nerve system \#todo/opzoeken of heet het intern/extern? and a
mattress is supposed to support your back, in the same way we need a
spiritual pillow and mattress to support our spiritual head and body.
Our head is what we think and feel with. When we think the wrong way, we
will not be peaceful or passionate about anything. The peace that guards
the minds as mentioned by Paul in Philippians 4:7 comes from the
mentioned prayer in the verse preceding it, and not just prayer, but a
threefold combination existing of prayer, petition and thanksgiving.
Then what can we compare the mattress with? That would be unity then.
Colossians 3:15 says \textgreater{} \emph{``And let the peace of Christ
rule in your hearts, to which indeed you were called in one body. And be
thankful.''}\\
and Philippians 1:27b says \textgreater{} ``you are standing firm in one
spirit, with one mind striving side by side for the faith of the
gospel''\\
This peace between people, not to be confused with the peace in our
minds, is what makes us one in the spirit. For how could one body fight
against itself? This happens when people have auto-immune diseases, but
spiritually, as God created it, this is not possible. Discussions
between theologians such as Michael Brown and John MacArthur
\#tags/famouspeople about for example cessationism are not meant as
polemic devices to destroy one another, but rather to build the other up
(as well as the readers of their respective books). They can be peaceful
because their purpose is to show each other what they believe is the
truth, not for their own selfish purposes, but in order that the other
may stand correctly before God and with a clear conscience.

\emph{12 Rejoice in hope, be patient in tribulation, be constant in
prayer. 13 Contribute to the needs of the saints and seek to show
hospitality.}

It is for the same reasons that the following two verses emphasize
prayer and the unity with other Christians. Verse 12 is not even stated
as ``pray and then you will rejoice in hope''. Rejoicing is not put as
there as a consequence but rather as a something a Christian should just
``do'', even though it is not a thing that can be done, but must occur
out of a feeling. For Christians, however, this should be a normal thing
to do, when prayer is also on this same list.

\emph{14 Bless those who persecute you; bless and do not curse them.}

Continuing in these marks of the Christian we get one of the most
difficult traits to actually commit to on a daily basis: the blessing.
Not only is this verse asking to \emph{love} those whom we hate---and we
could hate people for a number of small to larger things---but it is
asking us to \emph{bless} those who \emph{persecute} us. The Bible very
often does not only ask us to do the opposite of what people usually do,
which already is a hard task, but it then asks us to add more love on
top of that. See what Jesus said in the Gospels about walking the extra
mile with the Roman soldier for example, or what is written below in
verses 19 and further. In this verse we should not only not hate our
persecutors, but we should first of all, love them and second of all
bless them, which means we wish them well, that their life is good, and
thirdly, they are not mere people who hurt us, but who \emph{persecuted}
us and perhaps have killed some of us already. How is it ever possible
to love people like that? The only way possible is if you think you are
able to do so, \textasciitilde that is when you have hope. Now why would
it require hope to bless someone?\textasciitilde{} \#todo/nogaftemaken
this hope is needed so that one can finish this process of forgiveness
Remember the three steps we took in the previous paragraph. We need to
1. \emph{Relativize} or \emph{View the persecutor as a normal human
being} 2. \emph{Forgive} 3. \emph{Bless}

\textbf{Step 1. View the persecutor as a normal human being} Often we
can forgive people for smaller mistakes, because they are human. By
``human'' we usually mean that we have committed these mistakes
ourselves as well. This means that we, ``normal'' human beings commit
``normal'' size sins and sins of others can be forgiven when these are
within our range of acceptance, which is defined by the sins we do
ourselves. As soon as the sin we are confronted with is bigger than a
sin we have ever done ourselves, we are shocked by it and we see no way
of forgiving this person.

(Looking at Jesus will relativize the sin of others \#todo/nogaftemaken

\textbf{Step 2. Forgive} Ultimately, blessing the other means two
things---actually, they are two consecutive steps. First, you do not
hold a grudge against this person, that is you forgive this person and
do not want anything bad to happen to this person. Second, you wish the
other person well---that is you wish something good will happen to that
person. As for forgiveness, this itself consists of two parts. The first
part is pointed towards yourself and is causing one being unable to have
peace. Suppose you are a doctor. You are performing surgery on someone
you love and your assistant brings you the wrong tools. Due to the loss
of critical time the person on the operating table dies or is severely
hurt. Now you cannot forgive your assistant. The questions that keep
popping up in your head are: ``Why did you bring me the wrong tool? Do
you know how it feels to lose someone you love?'' The person asking that
question has no peace of mind until his questions are answered. Now the
last question implies he wants the other person to feel the same thing
he is feeling. This brings us to the second part of forgiveness, which
is pointed towards others. Forgiving means that you should not
\emph{hold} a grudge. It is phrased very well by the word ``hold''
because unforgiveness actually is like holding on to something that
hurts you. If you let it go it cannot hurt you anymore. Now what is this
thing that you are holding on to? It is the desire for the other person
to be inflicted with pain. Every kind of unforgiveness ultimately has to
do with someone who hurt you, directly or indirectly. This is, as far as
I know, \emph{always} the intent of unforgiveness and revenge. The only
difference between the two is that in the former the latter is not
enacted and resides within the heart. The movie \emph{Don't Breathe}
\#tags/movies depicts this very well in a father whose daughter has been
killed in a car accident by a woman. Out of justice he takes the woman's
child. Supposedly simply in order for him to get back what was his, that
is a child, but also to let the woman feel what it means to lose
something. It is not mentioned in the movie that this was his ulterior
motive, but when one thinks logically it is clear that he wants her to
suffer whether it is by losing her freedom or her child.

\textbf{Step 3. Bless} The word bless is even written two times in this
verse, to emphasize that this needs to be done as if it is not clear to
us. Now it is true that most of us do not know what blessing another
means. Is it simply telling the other ``I bless you with a happy or
healthy life?'' This has no meaning at all. First of all, our words have
no power perse. God's words have powers of creation and destruction, but
ours do not. God does warn us against cursing him or others and even
poses the death penalty for that, so perhaps there is some power in
that, but only because God gives it power. Second of all, how do you
bless another if you cannot meet him? Blessing ultimately happens by
God. He is the One Who blesses man with riches, happiness or health and
all we can do is ask God to bless him. So a blessing always means that
we pray to God for the other, for our persecutors. Of course we bless
our persecutors with knowing God first, and then with being blessed in
their new life.

\emph{15 Rejoice with those who rejoice, weep with those who weep.}

A hallmark of true Christians is empathy or better yet, compassion.
Rejoicing with others who rejoice, even if it is not a reason you would
rejoice for, is show of selflessness. It is easy for one to rejoice when
one has something to be joyful about. If another person wins the lottery
and you get a small part of it, you will be rejoicing, but more for
yourself than for the other. When we watch a movie and we are rejoicing
for a good ending for the protagonist, such as in the movie \emph{The
Pursuit of Happyness} \#tags/movies\# , some of us are selfless and
rooting for Will Smith because we wish him the best, others, however,
are only rooting because they do not want to be left with a bad feeling,
as one has with movies like \emph{Memento}. Weeping therefore is an
easier thing to do than rejoicing and so in my opinion rejoicing with
others is the true hallmark, but sincere weeping and not merely pitying
another is not far from it.

\emph{16 ::Live in harmony with one another. Do not be haughty, but
associate with the lowly. Never be wise in your own sight.::}

My conscience is speaking to me as I have been writing about this and
the previous verses. Being a Christian is not an easy job, it is not
something one must do for a short while, it is what he must do
continuously and live out. Especially the ones who live with you know
your true character. That is what makes living in harmony so difficult.
You cannot hide it from the other. What also makes it difficult is to
keep this harmony. There are so many ways in which the harmony can be
disrupted. For each of those forgiveness must be granted and sometimes
it is forgiveness upon forgiveness---and that is aside from all the
other things the encompass harmony, such as patience, friendliness and
all the other virtuous characteristics of love. As for being haughty,
this too is one of my pitfalls. It is one of the main reasons why I
cannot live nor even \emph{work} with people in harmony. I always want
things to go the way I want them to go, because I think my way is the
best way. Now that in itself is not causing the problems, it is when I
encounter people who do not want to go my way. In those cases I need to
accept that we go for a compromise or a different solution altogether.
My haughtiness, however, rejects this kind of solution. I think of
myself so highly that it seems like it is the only way to go forward.
The last verse is speaking to me as well, for I am wise in my own sight.
In others' sights, especially God's I am not, for there are several
things I am missing here. I am missing the importance of relationships.
By pushing my way other people will get out of my way either voluntarily
or forcibly, which destroys the trust and ultimately the relationship.
The second thing I am missing is related and is the appreciation and
encouragement of others. I am neglecting the input of others and
especially those lacking self-confidence will think their ideas are not
good enough and are not stimulated to come up more often with new ideas.
The idea of this verse is never to be wise in your own sight. If you do,
you will not see the greatness of the blessing God has for you either
through His or others' wisdom.

\emph{17 Repay no one evil for evil, but give thought to do what is
honorable in the sight of all.}

We often think in terms of equity and equality. When we think about
justice, a person needs to receive punishment according to the principle
of \emph{an eye for an eye}. However, in practice only God knows the
real value of the one eye and whether it can be repaid for by the other
eye. It is for this reason that God has also commanded that the
Israelites bring forward their case to the priests who would then judge
over it and later on to the judge (although I am not sure whether the
judges in the book of \emph{Judges} are among this category)
\#todo/opzoeken \#biblestudy/questions and king. 1 Kings also clearly
shows that wisdom should be a trait of the person who judges, in this
case Solomon, for it is needed to distinguish between minute details of
each unique case. Now an eye is a material thing which can be given to
another, but a death is not something one can give to another. A family
of which a member has been killed by a murderer cannot easily be
compensated by the mere death of the murderer. That life might not be
worth the same as their family member, e.g.~their son. Now the son of
that murderer might be worth it, or might even be worth more, but would
the family want the innocent son of a murderer to die just so that the
murderer would have the same feeling and hopefully regret? Most likely a
righteous family would not want that kind of blood to be spilled and
even if the murderer feels the same kind of regret, what good is it to
them? Their forgiveness may have been granted then, but both are
ultimately in a state of loss, i.e.~having less than what they had
before. No wonder the second part of this verse tells us to do what is
\emph{honorable} in the sight of \emph{all}. These rules of punishment
were given by God in order to enforce law and order, and it is the
perfect way of justice. However, it was never meant as perfect way
solving a relationship or emotional hurts, for that would require mercy,
forgiveness and love, but God could not write that in His Law, for love
and forgiveness are not things one can command the other to do. He has
shown this, however, many times over in His own examples of how He
forgave the Israelites, showed his patience and mercy, and He proclaimed
this even through David and His other prophets.

\emph{18 ::If possible, so far as it depends on you, live peaceably with
all. 19 Beloved, never avenge yourselves, but leave it to the wrath of
God, for it is written, ``Vengeance is mine, I will repay, says the
Lord.'' 20 To the contrary, ``if your enemy is hungry, feed him; if he
is thirsty, give him something to drink; for by so doing you will heap
burning coals on his head.'' 21 Do not be overcome by evil, but overcome
evil with good.}::

Paul ends this chapter, the section about marks of the true Christian
with one thing. He uses an important condition: \emph{If it depends on
you}. Now this is not a condition behind which we can hide. We cannot
say each time whenever we have a fight with someone that it was not our
fault, because there are many ways where we can avoid a fight or show
the characteristics of the Fruit of the Holy Spirit (cf.~Galatians
5:22). But in the rare occasion where even that does not help and where
our meekness is abused, we can still leave the conversation or
situation. Then there is also verse 19, which tells us never to avenge
ourselves. While in verse 14 it already tells us to bless our
persecutors and in verse 17 we should not repay evil for evil, and these
should be clear enough to rule out any vengeance. However, Paul still
warns us for this to make sure that we leave it to the wrath of God.

Heaping these coals on his head will let your enemy be speechless. He
will not know what to do. The reason why we ought to make him
speechless, is because we should let him stop and think. We can attack
him and let him run away out of fear, or we can cower ourselves and let
him attack us, but in both of these cases his heart is self-righteous,
that is he thinks he is doing the right thing. But as soon as we do good
to him, he will stop and start thinking about what to do. That will give
him a moment's rest to think about why we did what we did, and hopefully
he will come to the conclusion that it is because of our love for
God---and His love for our enemies---that we treat our enemies in this
way.

\#church/material echt doen met een bak en ballen erin bijvoorbeeld.

\#biblestudy/memorization/todo

\#biblestudy/devotionals/romans

\hypertarget{romans-13-esv}{%
\section{Romans 13 (ESV)}\label{romans-13-esv}}

\textbf{Submission to the Authorities} \emph{1 Let every person be
subject to the governing authorities. For there is no authority except
from God, and those that exist have been instituted by God.}

All authorities, even those from the government and everything beneath
it, especially those in church but also those of village chiefs, have
been instituted by God. As evolutionists say this kind of social
hierarchy is the cause of what we creationists call ``morality''.
According to them there is no morality, but only acting in everyone's
favor so that there is peace. Humans would steal and kill, but because
we can think, it would not be beneficial if everyone were to do that.
This is utter nonsense of course because we see in many post-apocalyptic
and dystopian movies or other productions such as the series \emph{See}
\#tags/movies with Jason Momoa or \emph{Mad Max} that if this were the
case, the strongest or fittest would survive and we have seen the same
things going on in our current time in Nigeria, Liberia and and short
while ago in Rwanda. There was no \emph{utilitarian} thinking there that
it would be best for everyone if they did not kill the people but rather
use them or their skills in Rwanda; Did they not know that destroying
the electric circuit in Liberia would also remove the possibility for
electricity for themselves? Of course they did, but their anger and
madness overruled everything. That is why we, even though we do believe
in moral values, still have to listen to the government because it
creates a sort of order in the chaos. Would this have prevented the
above examples in Africa? No, because they were caused by rebels. But if
the rebels had obeyed the government it would have been less chaotic.
Would it help any in Islamic countries where it is not rebels, but the
government who persecutes you? No, there we have a different problem.
This chapter is speaking about governments in general, and generally
speaking governments are good. Most governments are good and of those
that are not good, even most of the law is good. Countries with bad
social and economic conditions such as Malaysia and perhaps all
South-East Asian countries have good economic policies (once again
generally speaking), but it usually is their justice department that is
corrupt and as a consequence they cannot enforce their laws, which is
why their economy is still not on the level of those of the West.

\emph{2 Therefore whoever resists the authorities resists what God has
appointed, and those who resist will incur judgment.}

In light of the current Coronavirus-crisis many Christians, but
unfortunately not all, are in a dilemma as to whether they should obey
the government and gather as Christians or that they should obey the
Bible who commands us to (cf.~\#todo/opzoeken Hebrews 10). It is my view
that it is the devil's work right now to use this grey area where the
government says something good (provided that risk of the Coronavirus is
real, which I still have doubts about because over 99\% of everyone who
dies already had illnesses, whether young or old, and being old itself
already was a weakening factor) and let's Christians first of all submit
to their fears, and then second of all obey the government and disobey
God's command to gather regularly. This particular strategy of the devil
has caused many Christians---although I have heard of two only,
personally---to become colder and feel less passionate as a Christian.
Our simple opportunities to spread the Gospel or exemplify it have been
taken away, although there still are other opportunities, albeit harder
to grasp. When Christians are submitting to fear that is not the biggest
problem perse. They can call out to God in their fears and ask them to
provide strength and courage. However, I do not hear those people in
church or in online (prayer) meetings calling out to God for their own
fears. Perhaps it is because I am surrounded---by the grace of
God---with fearless Christians in my church and through my church,
online, with other fearless Christians. What I hear during our prayers
is intercession for those who are afraid, for those who are not joining
the meetings, be it online or in person. That is what truly is
indicative of these people. They do not come to the meetings in person,
they are for closure of the church meetings, they do not join the prayer
meetings and they are afraid. There is one step I have left out, and
that is that they do not worship God with all their hearts, soul, mind
and strength. I am judging, too soon, because I do not know what is in
their hearts, and for that I ask forgiveness. But if I were to be right
after all, I would not be surprised. For the time being I will treat
everyone as if they are rational, influenced by \#todo/opzoeken bedoeld
voor de mensen, zoiets als populair, en dus niet bedoeld om de waarheid
te verspreiden, biased and fear-inciting news, and obedient to the
government.

\emph{3 For rulers are not a terror to good conduct, but to bad. Would
you have no fear of the one who is in authority? Then do what is good,
and you will receive his approval,}

Should we, Christians, be afraid then when we gather despite the
government's advise, warning or threat not to? Since we know that what
we do is good we do not have anything to fear. On the other hand, it
\emph{could} be something bad, such as spreading the virus even more so
that it becomes even more unmanageable. That \emph{could} be the case
indeed, but the question is: are the measures taken in proportion with
the risk itself? My view on this is that these measures are far too
high. There is no clear knowledge---and I am wondering why not---about
how the coronavirus spreads (whether through air or touch) or how far
the distance is that the virus can spread; Furthermore, they do not even
know whether someone can get the Coronavirus even when one is sneezed
upon. We could be carriers, but it does not mean we will be symptomatic
and suffering from it. There is just too little info on this and the
statistics are blown out of proportion. What percentage of people dies
from Corona and what percentage dies from the flu? These numbers are
never shown broadly on the news to still people. Either way, I do not
think any party can say whether we are doing good or bad based on the
available information. Some humility needs to be in place here and there
needs to be mutual respect as well for each other's choices. The
government's policies alone is not a sufficient argument.

\emph{4 for he is God's servant for your good. But if you do wrong, be
afraid, for he does not bear the sword in vain. For he is the servant of
God, an avenger who carries out God's wrath on the wrongdoer.}

I have the feeling that a lot of Christians and churches suddenly raise
this verse of being obedient to the government and though I agree with
this verse I am reluctant in believing their conviction. When did I ever
hear Christians speaking of or preaching about obedience to the
government? Almost never. I have heard one pastor using one example
twice or thrice, about making payments transparant for the IRS, but
aside from him I cannot remember anyone preaching about it. Furthermore,
I think a lot of copying is done without the payments for copyrights. So
the real reason why Christians all of a sudden want to obey the
government and want to close their church doors is due to their own fear
of the Coronavirus.

\emph{5 Therefore one must be in subjection, not only to avoid God's
wrath but also for the sake of conscience. 6 For because of this you
also pay taxes, for the authorities are ministers of God, attending to
this very thing. 7 Pay to all what is owed to them: taxes to whom taxes
are owed, revenue to whom revenue is owed, respect to whom respect is
owed, honor to whom honor is owed.}

So for the sake of conscience and to avoid God's wrath we pay taxes, but
in verse 7 it says that we should pay what is owed to them. Would this
also apply if the government imposes exorbitant taxes? On the one hand
yes, it could mean that we only pay what we \emph{should} pay them,
which could be less than what they ask for, but on the other hand it is
their country, so they can decide. We could always decide to move to
another country. My brother, for example, decided to move to Germany due
to better immigration laws for his wife who was from another country.
Another person I know has moved to Belgium to get more tax benefits. I
think that because we have the freedom to move, we should pay the
government what is due, i.e.~what they ask. If we do not agree, we can
more. However, for others in the lower class within the Netherlands, or
people from other countries such as Iran and China, where people are not
allowed to emigrate to other countries, I do not think this rule
applies, for the government could ask for their lives, so to speak, and
they would then have to pay for it according to the Bible. That could
not be the God's intention---in my humble opinion. God asks us for our
lives and has asked us to (be willing to) give our lives for others, but
that was in a different context, as in catching the bullet for one
another. Although I can imagine that if the whole country rebels against
these tax laws, except the Christians, it would give a very impression
to the government of Christian obedience to their God, which in turn
might change the heart of the government. But these situations and
reactions would require a context specific insight from the Holy Spirit
and the right answer cannot be known beforehand.

\textbf{Fulfilling the Law Through Love} \emph{::8 Owe no one anything,
except to love each other, for the one who loves another has fulfilled
the law.::} \#biblestudy/memorization/todo

In the same line as with stewardship over money, God asks us to owe no
one anything, except to love each other. The character of God and His
commands are quite often the opposite of what the world teaches. Just as
does for love, the law of \emph{the more you spend, you less you have}
does not apply here. Now the whole world is about money, everything in
life is about money, whether you want to or not. Some need money to buy
everything they need to survive in this world, for others money is
everything they need to survive in their (high) class. But we are not
from this world, we are merely \emph{in} this world. Our bodies are
still present here, but our soul belongs in heaven. So we use money to
support our bodies and the morale of others. The act of \emph{giving
away} money, by seeing it as worthless to us, provides hope for those
who need it so much. For they can now see that it is possible to treat
money as worthless. That thing they have been hoping for and which is
unattainable can be done away with so easily, by putting one's hope in
the Lord, which is free. So giving money to others provides relief on
two fronts. First of all it takes care of their basic needs such as
food, water and housing. Second of all it gives them food for their
soul, by showing them that there is no hope in money, but there is hope
in something far greater than money and which is also more attainable
than money. Now as for hope in money, I need to elaborate on that,
because it can be interpreted in two ways. One can hope \emph{for} money
and hope \emph{in} money. \emph{Hoping for money} means that one hopes
to get money some day. This hope is small, because one usually gets
money when they work, and then still it is not a lot for most people.
\emph{Hoping in money} means that one believes money can save them from
their problems, their misery and their poverty. The latter may be true,
but the former is not caused by the latter. That is what they are
missing. The Gospel provides hope on both fronts, because it is easily
attainable. All one has to do is accept what God is already reaching out
to them, salvation and God's grace and with it eternal life. When one
has this salvation one can also put their hope \emph{in} God and in this
salvation, because it was made by the power of the Resurrection. As real
as the Resurrection is, so real is the fact that we will resurrect.

\#tags/prayer/money Help me, Lord, that I might be as generous as you
are in giving money and love to others, for these are my weak points. I
am too selfish to love others equally to or more than myself and I love
myself too much to give money to others instead of to myself. Aside from
that, I fear too and do not have enough faith that You will provide, and
so I keep money on the side.

\emph{9 For the commandments, ``You shall not commit adultery, You shall
not murder, You shall not steal, You shall not covet,'' and any other
commandment, are summed up in this word: ``You shall love your neighbor
as yourself.'' 10 Love does no wrong to a neighbor; therefore love is
the fulfilling of the law.}

Every commandment ultimately can be summed up in love. Every commandment
that has to do with hurting someone or doing good to someone, has do
with one loving the other too little or sufficiently well. It is amazing
to see how God has created a society where He put one variable and that
when this variable is optimized the whole of society is optimized as
well. If everyone were to aim for maximum love there would be no wars,
economic stress, poverty or any other negative things, except for
physical discomfort due to sickness for example. Note that this is
different from what the hippies in the 70s were claiming. They were also
claiming peace and \emph{make love, not war}, but real love has justice
on the other side of the coin. It means criminals need to be punished
and communism needs to be expelled---I am referring to the continuation
of the Vietnam War---so that real love, which flows from the freedom the
Vietnamese would then have, can thrive. Real love also pertains to
loving one's own and another's body. It would not mean free sex and
smoking marijuana, because these hurt the body and the soul.
Furthermore, they cause a decline in morale and diligence---have you
ever seen anyone smoking pot while being truly diligent at work or
looking for work? Usually it is one way or the other, but not both. In
practice this has led to many more drug addicts, an increase in
unemployment and an impediment to letting the economy grow faster.

The people of this world know that people are selfish. Economic and Game
theories are based on that premise and that of maximization of utility.
So economic policy makers have put policies in place that are driven by
greed. That is eventually what keeps the market in check and causes the
Free Market Economy to work well, which is the base of Capitalism.
Socialism would work very well if love were the driver, but it is
exactly because \emph{greed is the driver, not love} that socialism has
failed in every country throughout history. \#tags/politics In the
kingdom of God this would not fail of course and we can see a minute
example of that in our churches, which represent the Church which is the
Body of Christ. Local congregations operate on a voluntary basis. Nearly
everyone who helps maintaining these work for free and also donate to
keep the church and its activities going. In return the poor are helped
and the social and emotional weak people are raised to higher levels
through programs such as the \emph{Peace Plan} and \emph{Celebrate
Recovery®}, both created by Saddleback Church, California. For those who
do not have enough funds to participate in meals or Bible Studies
usually will be found a way because they are all part of the Body of
Christ. It is not perfect, because we still live in a broken world, but
it any country can boast about the results of socialism---where people
are willing to sell their goods for the greater cause, where the rich
offer their money for the poor and where real poverty is nearly
inexistent---it is the Kingdom of God. \#church/material

So the driver for the economy in this world is money and the greed for
it, then what is the driver for morality or in other words justice?
First of all it is money again, by imposing fines on everything from
speeding to theft and robbery; Second of all it is withholding one's
freedom by imprisoning people. I heard a person once say that the Bible
never commands people to be jailed for a crime. It is either a physical
punishment or a fine for retribution or compensation of the inflicted
material or emotional damage. If this person is right, a wealthy man
could go his way up until the monetary punishments, assuming the
physical punishments prevent him from going any further. Personally, I
think this idea of physical punishment is better than jailing people,
because taking one's freedom away, how harsh it may be, has not
prevented people from crime if we look at the crime rates. Here too, by
the way, with love as a driving factor criminality would be at all-time
lows. Have you have seen criminality in the Kingdom of God? That would
be contradictory. If there was any---and there have been many pastors
and elders who have stolen money from the church---they were not
Christians and at least not loving persons.

\emph{11 Besides this you know the time, that the hour has come for you
to wake from sleep. ::For salvation is nearer to us now than when we
first believed.::}

Perhaps this reference to time is because some people are thinking that
they can postpone their good behavior until Jesus comes back. Now,
however, Paul warns us that salvation is near. So if there is a time to
wake up and be a good Christian it is now. Could this be the right
interpretation? I do not think so. Remember that we are still in the
continuation of Romans 12 where Paul sums up the marks of a true
Christian. So Paul is speaking to Christians and love is just one of
those marks, albeit the most important one. The next verses shed more
light (no pun intended) on the explanation of this verse.

\emph{12 The night is far gone; the day is at hand. So then let us cast
off the works of darkness and put on the armor of light.}

What Paul describes here is that we used to live in the dark. Everyone,
whether Jewish or Christian, used to walk in the dark and doing things
that belong to the dark. The sleep verse 11 refers to could mean the
time of night when everyone is asleep. This seems to contradict the
immoral acts in verse 13 though, for how could anyone do these things
when they are asleep. But as often is with Paul, this does not have to
be a contradiction. Paul could mean multiple things with a single
expression. He could mean that we as Christians have been sleeping, that
is \emph{while} we were Christians we had not been paying attention to
the importance of love and fulfilment of the Law. This is true even now,
many whom I believe to be Christians, live a fairly good life, but they
do not exert the most out of it. They spend their times on holidays and
hobbies, while they good evangelize or help the Church in another way.
It is all because they are sleeping and do not see the need of people of
Jesus Christ and that their city and country can only be saved when
people come to Christ. Another meaning Paul could also be intending
\emph{simultaneously} is the time of day, as mentioned above. One could
parafrase this as: ``Wake up, it's time to get up. The day has come. To
those who were not sleeping, it not time to go sleep now, for Jesus is
coming!'' \#todo/opzoeken Though the exegesis or eisegesis in this case
might be a bit off the books, we have to understand that Paul, even
though he writes like a perfectionist, and even though he is inspired by
the Holy Spirit, could be writing verse 11 with people needing to wake
up from sleep because the day has arrived, and verse 12 with those who
are not Christians and still sin in mind. It is not like Paul had lots
of drafting paper on which to sketch the details of the outline of this
letter, but rather that he had to cite it while Timothy or Epaphroditus
was writing it down, with merely seconds to think it through. Hence we
often see additions such as ``this is not from the spirit, but from
me''. With this in mind a textbook exegesis is not applicable but a more
complex human interpretation is necessary.

\emph{13 Let us walk properly as in the daytime, not in orgies and
drunkenness, not in sexual immorality and sensuality, not in quarreling
and jealousy.}

As for these sins, it is not normal to do these things as Christians and
then say: ``Jesus is coming. Let us stop doing this and act normal
now''. Although that \emph{does} happen in this world. How many men are
not involved in prostitution and watching porn at night, and then
showing up at work the next day? In some cultures this has even become
normal, like Japan, Thailand and Hong Kong. In the daytime we all act
morally. Some, like the political Left, still are not, but even then
they make their acts seem like righteous acts. As mentioned before, the
nighttime proves what a person is really capable of doing. I myself am
the perfect example of that. What people show during daytime, does not
indicate their behavior at nighttime is similar. In the book \emph{Lord
of the Flies} \#tags/books \#tags/movies we see that the behavior of the
children with a specific character and trait on the boat, that is
daytime, was amplified when they were on the deserted island. This could
be true, of course, but we also see in Brett Easton Ellis'
\emph{American Psycho} that the protagonist Patrick Bateman showed his
perverse hobbies in the evenings when no one was around. As Christians
we need to change this culture. The behavior shown above with
prostitution is done by people in Hong Kong (with Buddhist backgrounds),
India (with Hindu backgrounds) and in Arabian countries (with Islamic
backgrounds). The Hindus and other religions such as the Greek and
possibly Wicca-like religions actually openly allow (temple)
prostitution or sexual immorality in another form, or condone it. As far
as I know Wicca has nothing in its book about moral behavior, everything
is accepted. Christianity is the only religion that openly condemns
sexual immoral behavior \emph{and} practices it which can be seen
through the practices of its followers.

Aside from sexual immorality verse 13 also discusses jealousy and
quarreling. How big of a problem is that these days within the Church?
Although it is addressed openly in the Bible, it is unfortunately still
a practice within many churches, although it happens more frequently in
churches who focus more on religion than on the Gospel, or in other
words on \href{https://www.gotquestions.org/moralism.html}{legalism and
moralism} than on God's grace in order to reach heaven. But it is
exactly because the Church has an active community that these things can
happen. In Buddhism, in my view and limited knowledge, there is no
community and people only visit the temple and pray or do studies by
themselves. Perhaps there is a class on their scriptures, but it is not
like they are working together, having fellowship, preparing for music
together and have many formal and informal meetings. The same goes for
the Hindus. The Muslims, however, are in this respect more similar to
Christians, they have more community activities but are more quarreling
at the same time. This still is no excuse for Christians to quarrel
though and we should be very aware of the power of our tongue, that
gossiping is a big sin addressed multiple times in the Bible, and the
unity is the one thing we should all strive for (cf.~Philippians 1:27).

\emph{14 But put on the Lord Jesus Christ, and make no provision for the
flesh, to gratify its desires.}

Now this is a big thing to do: putting on the Lord Jesus Christ. As many
facets as there are to clothes, even more there are to Jesus Christ. In
Principle 3 of Celebrate Recovery we learn that we have to give over our
\emph{whole} life to Jesus Christ. That alone is a huge step, but it is
only the first of two parts, which is taking off ourselves (i.e.~our own
clothes) and the second part is taking on Jesus Christ (i.e.~His
clothes). You will notice how hard it is to put on the glove of Jesus
while your glove is still on, but it is even harder to take off your
glove when His glove is already on your hand. This is the situation we
put ourselves in when we decide to mix Jesus' teachings with our own.
When we try to reconcile the teaching of the Bible with our own, and
slowly adapt our way to that of Jesus, that is the hard way. It is much
easier to throw away everything and going back to the basics. What is it
that the Bible teaches? Not what your church of teacher has taught, but
what is actually from the Bible itself? Throw everything you want
yourself, and only take that which is from God for yourself. Verse 14
says it so clearly: \emph{make no provision}. A provision is a reserve.
You reserve a little bit of space for yourself to enjoy the things of
the world, or a little slack for your own sins, to tell yourself you are
taking small steps, but it does not work that way! You cannot tell
yourself to visit the prostitute a bit less frequent every week or to
drink less when you are an alcoholic. There can be no provision. You try
to quit \emph{cold turkey}---and if you manage to maintain sober for a
day, that is victory. When you fail, it is a loss, but you try again and
if you maintain sober for two days it is a victory again. That is what
is meant by small steps.

\#church/material/crlessons/principle3

\#biblestudy/devotionals/romans \#tags/socialism

\hypertarget{romans-14-esv}{%
\section{Romans 14 (ESV)}\label{romans-14-esv}}

\textbf{Do Not Pass Judgment on One Another} \emph{As for the one who is
weak in faith, welcome him, but not to quarrel over opinions.}

It is quite often that I would like to welcome people to have a debate
over certain issues, such as apologetic, doctrinal or political. Perhaps
that is in the didactic personality I have, but it is more in the drive
I have to convince others of what I believe is the truth---or perhaps to
convince others that I am right. Now I realize that this is not the
right way---it is not the right way to evangelize nor is it the right
way to build up Christians. Teaching them and inviting them to seminars
is another thing. With this in mind they know what they are choosing for
and will enter my house with an open mind, but when one is invited for
dinner or a movie, he will be surprised by the subjects and will likely
feel the need to defend himself.

\emph{One person believes he may eat anything, while the weak person
eats only vegetables. ::Let not the one who eats despise the one who
abstains, and let not the one who abstains pass judgment on the one who
eats, for God has welcomed him.::}

From \emph{MacSBNnkjv} we obtain a different interpretation of the verb
``to pass judgment'' as I have written in my notes to verse 4 below,
which I will continue there. It is also a perfect response to our
current situation with the Coronavirus. We are despising the other or
having some kind of other contempt, while they are holding us to be
irresponsible and perhaps depraved. This is what \emph{MacSBNnkjv}
writes:

\begin{quote}
\emph{(MacSBNnkjv) ``Despise'' indicates a contempt for someone as
worthless, who deserves only disdain and abhorrence. ``Judge'' is
equally strong and means ``to condemn.'' Paul uses them synonymously:
The strong hold the weak in contempt as legalistic and self-righteous;
the weak judge the strong to be irresponsible at best and perhaps
depraved.}
\end{quote}

The problem between Christians right now, in the current
Coronavirus-crisis is that of letting another take offence
\#todo/opzoeken/engels aanstoot geven Now usually this only concerns
brothers, that is other Christians, with a weaker faith. We can help
build them up, by not putting a stumbling block in their and restricting
their growth by imposing anything on them which---to God is neither
wrong nor right, for it is not wrong to abstain from eating meat nor to
eat meat offered to idols---rather than letting them ignore their
conscience, which is a bad habit (the latter is also taken from
\emph{MacSBNnkjv}). We would also not purposefully drink alcohol when
there is a non-Christian alcoholic, but that has more to do with general
love for the other, instead of building up the other person, for the
non-Christian has no faith, so there is no faith to build up. It could
give a bad image to other Christians, however. Notice then the following
two points: 1. Refraining from passing judgment is only applicable when
the subject matter is neither wrong nor right, and 2. this is done to
build up the fellow Christian's faith or to love the other when it
concerns non-Christians and provide a good example of Christ. With the
current issue between Christians where one part says we must stay home
(whether it is out of fear or to obey the government) and another says
we should still meet as Christian (which is almost always because they
obey the Bible, but which could also be out of legalism or
irresponsibility) it is hard to say whether this is wrong or right. The
only way we can make a distinction here is by having scientific proof
and a standard of what is dangerous. The thing is, we do not have
scientific of the contagiousness \#todo/opzoeken/engels besmettelijkheid
? of COVID-19 and even if we had it, say it is 30\% when a person is in
the direct vicinity of 0.5 meters, then what is the threshold for
something actually being dangerous? We also know that most persons are
asymptomatic carriers and that of those who do get symptoms, a large
part will recover. But that still does not answer the question, is that
part large \emph{enough}. Who decides on that?

We can say, for example, that if the efficacy of COVID-19 is equal to
that of a ``yearly'' influenzavirus, which means that the contagiousness
might be less, but the mortality rate might be higher, in effect causing
the same number of deaths, but over a shorter period, would the amount
of danger or risk then be the same? And if that is the case, would it
then be acceptable to meet each other in churches and ignore the
government's instructions?

On one hand ``yes'' but on the other hand ``no''. As for the risk-factor
I do not see any arguments anymore why Christians should not meet each
other, but there is another factor, and that is that of taking offence
to another. It is this for this reason, which is even stronger than
other reasons, that we should change our attitude. Notice that some laws
in the Bible are emphasized more than others and some transgressions
have heavier punishments than others. What God wants most of all from
us, is this: Philippians 1:27a \#todo/opzoeken \textgreater{} \emph{No
matter what happens, always conduct yourselves in a manner worthy of the
Gospel. Then, whether I come back\ldots. unity} \#todo/nogaftemaken

Concluding \#todo/nogaftemaken

\emph{4 Who are you to pass judgment on the servant of another?}

As for a person weak in faith, it is the same, that is, he will be
surprised by our ``attacks'', but Paul goes on to explain that it is not
for that reason but rather for the reason that we should not pass
judgment on others who do or think differently than we do. If we merely
think about \emph{judgment} we might think it strange that we are not
allowed to judge the one who eats. Most Christians will actually think
it is not strange, but for a wrong reason. They will use Paul's last
line ``\emph{Who are we to judge others?}'' and ``\emph{Why should we
judge anyone for what he is eating?}'' They do not understand, however,
that this last line is not a rhetorical question perse. It could be
rhetorical if you are a mere church member without knowledge of
doctrine. But if you were to ask God or a godly priest this question, it
would not be rhetorical. So judging could indeed be done here, but Paul
is rather saying that first of all \emph{we} should not be the ones to
judge---if we are not in the position---and second of all we should not
judge others' servants. So for example, we should not judge a Buddhist
for not being a good Buddhist or a good Christian, for he has a
different master.

\emph{It is before his own master that he stands or falls. And he will
be upheld, for the Lord is able to make him stand.}

Now a new question arises and that is whether Paul is speaking to
Christians here with a weak faith or to Jews with stricter dietary rules
who essentially have another master. It seems to be clear from this line
that it is not about Christians but about people with their \emph{own}
master. On the other hand, the next line where it says that God will
uphold this person, and verses 6 and 7 seem to indicate that is about
Christians. As mentioned before, it would be strange, but not impossible
if Paul were to speak of non-Christians in verse 5 and of Christians in
the later verses. Remember that he is in prison and all kinds of
thoughts are running through his head and he is writing down what he
thinks of, without the possibility to interject a few lines between his
previous ones such as ``Now as for Christians'' before verse 5. In this
case, however, it does not seem to be likely and there is an explanation
for the seeming contradictions. Paul means to say here that if a fellow
Christian with a weak faith stands or falls, he will stand or fall
before God, not before us. So we should not be the ones to judge him.
That is why a priest and a teacher do have the right to judge, because
they are responsible to teach the right things to the people. Then if we
do not have the right to judge non-Christians, do we have any right to
judge Christians at all when we are not in position like that of a
teacher or a priest? I am still of the opinion that we should be careful
in judging because we cannot read the other person's heart. Also, we can
form a judgment about something to know whether it is good or wrong, but
we do not necessarily have to tell others or convict them of their
faults. I have the feeling Paul should have used the term ``convict''
(ἐλέγχω) here, but Paul has not used this word for a reason, but rather
``to pass judgment'' and in the Greek translation we read the word
\emph{(kata)krinou} (κρίνω/κατακρίνω) which also means to damn or to
condemn (cf.~John 8:10 ``has no man condemned you?''). Most likely there
is a difference between the Dutch and the English translations here,
because in Dutch we have a clear distinction between the judgment
without the punishment, that is only the ruling, and the punishment
itself. So for example, a judgment would be ``The verdict is guilty''
and the punishment would be brought forward as ``I condemn you to three
years in state prison'' which in Dutch would be ``oordeel'' and
``veroordeling'', respectively. It could that in English the verb ``to
pass judgment'' encompasses both of actions at once. In any case, we
know from the Greek translation that Paul meant that we should not
punish a person. This would render the translation of Matthew 7:1 as
``do not condemn, so that you will not be condemned''.

Will a person always be upheld then? No, not always, but my guess is
that when it concerns smaller matters such as eating or not eating, that
God will uphold one though his struggle with his conscience.

\emph{5 One person esteems one day as better than another, while another
esteems all days alike. Each one should be fully convinced in his own
mind. 6 The one who observes the day, observes it in honor of the Lord.
The one who eats, eats in honor of the Lord, since he gives thanks to
God, while the one who abstains, abstains in honor of the Lord and gives
thanks to God.}

If we are talking about Christians these verses are easily
interpretable, but then the question would be why the previous verses
would be speaking of non-Christians?

What is most important here is that we do everything for the Lord and we
can do it either by doing something or not doing something, although
\emph{not} doing something has to entail some kind of hardship. Not
working and resting when you really want to work, is hardship for some,
while for others the work itself is the hardship. Abstaining from food
would be difficult for everyone, although for some more difficult than
for others.

\#todo/opzoeken fasting with stomach problems

\emph{7 For none of us lives to himself, and none of us dies to himself.
8 For if we live, we live to the Lord, and if we die, we die to the
Lord. So then, whether we live or whether we die, we are the Lord's. 9
For to this end Christ died and lived again, that he might be Lord both
of the dead and of the living.}

Here it is made even clearer that the people Paul speaks to are
Christians.

Jesus never intended that only our lives would be His', He also wanted
our \emph{afterlives}, which include our deaths. We should not be afraid
of dying, not for ourselves and not even for the thought of our deaths
being a waste because we cannot serve the Lord anymore after that. No,
even in our death we can be of use as well as after our deaths. Now I do
not know how we could be of any use after our death, but just like God
uses each person, no matter how helpless, or even someone like Nick
Vujicic \#tags/famouspeople to bless millions of people, children and
teenagers who, though they have a perfectly working body, still have the
same thoughts and hurts as he has. Nick wanted to die, but Jesus wanted
to use him here on Earth. Hudson Taylor most likely wanted to live,
because he still had a lot of (according to him) unfinished work in
China, but God wanted him to die---perhaps so that others could take
over his work, and it has been many more people after that who have
become missionaries. If we look at the Torchlighters® series we see that
many of those missionaries were inspired by another missionary from the
same series, and that it leads back Hudson Taylor. He was one of the
first and his sickness \#todo/nogaftemaken and his death have led to
much more result---perhaps more than he has done in his lifetime.

\emph{10 Why do you pass judgment on your brother? Or you, why do you
despise your brother? For we will all stand before the judgment seat of
God; 11 for it is written,} \emph{``As I live, says the Lord, every knee
shall bow to me,\emph{ }and every tongue shall confess to God.''}
\emph{12 So then each of us will give an account of himself to God.}

\#todo See MBP Bear notes

\textbf{Do Not Cause Another to Stumble} \emph{13 Therefore let us not
pass judgment on one another any longer, but rather decide never to put
a stumbling block or hindrance in the way of a brother.}

Very often I have given my wife tests by making it more difficult for
her to come to her senses. The other day we had a fight and in my
opinion she was wrong for being angry and not wanting to talk about it
anymore. I believed she should be taught a lesson. I then left the home
for some rest, knowing she would not like it, but I did not do that to
get back at her. I did it so that she would come to her senses and think
about the wrong that she had done as well. Things only got worse
afterwards and instead of thinking about her wrongs she only focused on
what I did wrong. However, verse 13 makes me realize that I was passing
judgment on her---or in my own words, I was \emph{judging} her and
\emph{condemning} her---and putting a hindrance in her way to recovery.
Now I am not saying that my judgment was wrong, but Paul clearly tells
us not to do that. Perhaps the word ``rather'' even indicates that this
passing of judgment---whether the judging or the condemning part---is
the same as the hindrance.

\emph{14 I know and am persuaded in the Lord Jesus that nothing is
unclean in itself, but it is unclean for anyone who thinks it unclean.}

\#todo See also MBP Bear notes

\emph{15 For if your brother is grieved by what you eat, you are no
longer walking in love. By what you eat, do not destroy the one for whom
Christ died. 16 So do not let what you regard as good be spoken of as
evil. 17 For the kingdom of God is not a matter of eating and drinking
but of righteousness and peace and joy in the Holy Spirit. 18 Whoever
thus serves Christ is acceptable to God and approved by men. 19 So then
let us pursue what makes for peace and for mutual upbuilding.} \emph{20
Do not, for the sake of food, destroy the work of God. Everything is
indeed clean, but it is wrong for anyone to make another stumble by what
he eats. 21 It is good not to eat meat or drink wine or do anything that
causes your brother to stumble. 22 The faith that you have, keep between
yourself and God. Blessed is the one who has no reason to pass judgment
on himself for what he approves. 23 But whoever has doubts is condemned
if he eats, because the eating is not from faith. For whatever does not
proceed from faith is sin.} \#todo See also MBP Bear notes

\#biblestudy/devotionals/romans

\hypertarget{romans-15-esv}{%
\section{Romans 15 (ESV)}\label{romans-15-esv}}

\textbf{The Example of Christ} \emph{1 We who are strong have an
obligation to bear with the failings of the weak, and not to please
ourselves.}

We have to bear with the failings of the weak. What would be the
opposite of bearing? Bear requires a lot of patience, what comes to mind
is anything that does not require patience, such as getting food right
now, like fast food instead of healthier food, or even slower food when
one makes it himself. What else requires bearing? Irritating behavior of
our fellowmen. In everything that we bear, we are not pleasing
ourselves. Paul says it rightly, pleasing ourselves is the opposite of
bearing with another, especially bearing with the \emph{failings of the
weak}. Now it is bad enough to bear with another person, because there
so many differences between people---which is not wrong, but among
others DISC and enneagrams explain the differences in personalities and
how one would avoid fights and is rather peacekeeping, but another would
confront it to solve things quickly---but bearing the \emph{failings},
which is a fault, and even more so, that failings \emph{of the weak} is
even worse---and opposite all of that is the pleasing of ourselves. Why
does God command us to do this? Pleasing ourselves is not wrong in and
of itself, but it does not train us for harder work, just like resting
is not wrong, but one does not get stronger by it. So we do not become
stronger and neither does our neighbor get any closer to God. But when
the two of us come together and when the stronger Christian shows his
love and bearing, the weaker one, whether Christian or not, will see and
feel God's love for him and mature, and so will the Christian---at the
cost of his own pleasure. There is ample pleasure in heaven, let us work
now, and relax later. It is not as if God asks us never to relax. We
have received the Sabbath from God as a \emph{gift}, to rest and to
worship God---it is a gift, we cannot take it for granted. God could
have forced us to work seven days a week---but at the same time we
cannot discard it either. A gift from a respectable person with wisdom
or king is never thrown away, and from One Who is both and even more,
this gift should be used for it was given for a reason.

\emph{2 Let each of us please his neighbor for his good, to build him
up. 3 For Christ did not please himself, but as it is written, ``The
reproaches of those who reproached you fell on me.''}

God had a plan when He wrote the Scriptures. It has multiple layers for
multiple purposes. He wrote it for the first hearers, such as the people
of Israel so that they would know the Law; for the people in later
generations, so that they would know history; for the lawmakers and
other leaders, to know what is wisdom; for personal use, so that each
can know God and feel Him; of course it is also for the world, to let
them know Who God is, Who created the world and for what purpose, namely
to send His Son and to save them; and for His Church, so that they have
hope when doing the difficult things in life. One piece or story in the
Bible could have several purposes. The story of Jesus, i.e.~the Gospel,
lets us know that we are saved through Him, but it also tells us to be
like Him and to bear with the failings of others. Furthermore, it also
gives us strength in doing that. It gives us strength because it tells
us that everyone that the pain we bear is borne by Him. It is more than
just giving an example. It is not like when I tell my son: ``Look at me,
people are reproaching me incorrectly, but I am gentle, and so you must
be as well.'' It is rather the like following example: ``Be gentle like
I am, and every time they reproach you, I will bear the pain.'' Would
that not give you immense strength and courage to do anything? It is
like you go out and attack the villains and every bullet that is shot at
you is felt by Jesus, but not you. On one hand you would be fearless and
do all the good you want, but on the other hand you would try to
minimize the hurt, because you know Someone is going to pay for it. Some
people would rather not go out, not out of fear of being hurt, but
because they do not want Jesus to be hurt. There is something these
people do not realize. First of all, Jesus could feel all of the hurt,
when He was human, and now He still feels hurt, but the hurt He feels,
is not because of sticks and stones, or even words, but because people
reject Him. He is sad in two ways when one person attacks another:
because the attacker does not obey Jesus and because the victim is
suffering and Jesus is suffering with Him because of His compassionate
heart. If we do not go into battle Jesus would be suffering two times,
because if the attacker does not attack us, he will attack someone else.
But if we do go into battle, there is a chance that this attacker might
have a change of heart and if he does, Jesus will be suffering less for
it. However, if we return our attacks towards this attacker, Jesus will
be suffering four times, for the same amount of hurt that was done the
first time, is now towards the other person. So therefore, when someone
attacks you, enter the line of fire and do not fear, for it is the only
way to approach the other, and when you are near him, open your heart
towards the other and love him, and the more you are hurt be stay
gentle, the bigger the odds the other will hurt you less and be changed.

\#stories/bible/idea

\emph{4 For whatever was written in former days was written for our
instruction, that through endurance and through the encouragement of the
Scriptures we might have hope.}

\emph{5 May the God of endurance and encouragement grant you to live in
such harmony with one another, in accord with Christ Jesus, 6 that
together you may with one voice glorify the God and Father of our Lord
Jesus Christ. 7 Therefore welcome one another as Christ has welcomed
you, for the glory of God.}

Everything we do has one purpose and that is that we have harmony with
one another. How marvelous that is! It is so marvelous because God
always has a plan. In \emph{The A-Team} \#tags/movies Hannibal always
says: ``I love it when a plan comes together'' and even though we
expected him to say that, because he always does that at the end of an
episode, it still is exciting to see what he has come up with. Now God's
plans are even more surprising because He makes plans that span a
thousand years or more and then comes up with a twist no one could ever
think of, with the ultimate plot twist of sending His Son, Jesus Christ,
to die on Earth to bring Jews and Gentiles together. The plan is so
intricate that for thousands of years of easter eggs (no pun intended,
as easter or Pesach is also referring to Christ's resurrection), some
hidden better than others within the typologies of the stories of
Abraham and the Exodus, while others are implicitly proclaimed by the
prophets of Israel such as Isaiah 53, and some explicitly prophesied by
the likes of Daniel---and still no one, except to whom it was revealed,
figured out what God had planned, not even when it happened right in
front of their eyes. It was publicly revealed by God through Paul who
then wrote down this mystery in the letter to the Ephesians (chapter 3)
so that the whole world could know it. So marvelous is this that no has
can and ever will establish---and the greatest plot twist is yet to
come!

\textbf{Christ the Hope of Jews and Gentiles} \emph{8 For I tell you
that Christ became a servant to the circumcised to show God's
truthfulness, in order to confirm the promises given to the patriarchs,}

If this verse is not written correctly one might interpret this as if
Christ has~~come to Earth only for the circumcised. First of all, verse
9 needs to be taken into account as well. Second of all, this verse
could also be interpreted as follows. ``I became a servant to the
circumcised to show God's truthfulness; I became a servant to my parents
to show them the fruit of parenthood and the meaning of family; I became
a servant to my friends to show what humility is.'' There are multiple
reasons why Jesus came, which can be seen in the notes on verse 4 \#todo
corrigeren? We should take the first part of this chapter into context
when interpreting this verse, which is implied by the word ``for.'' So
this means that we should bear all the failings of the weak because
confirm the promises given to the patriarchs and so that the Gentiles
will glorify God and become followers of Him of course as well. The
second reason is extensively supported by the quotes in verse 9 and
further, but what does this bearing have to do with the promises to the
patriarchs? Those promises to Abraham, Isaac and Jacob were about Yahweh
being their God and the land of Canaan being their inheritance in return
for Israel's faithfulness to God. But what does this have to do with
bearing one another? \#todo/nogaftemaken It is not directly linked, but
it would be strange to inherit a country and then continue fighting with
each other as the two kingdoms did. This does not affirm anything to the
Gentiles about God's promise. The land might've been split up in ten
parts if God didn't intervene, or be taken over, because all these
little parts refused to work together. Also, it wouldn't show how great
God is. So both promises would not be affirmed then.

\emph{9 and in order that the Gentiles might glorify God for his mercy.
As it is written,} \emph{``Therefore I will praise you among the
Gentiles,} \emph{and sing to your name.''} \emph{10 And again it is
said,} \emph{``Rejoice, O Gentiles, with his people.''} \emph{11 And
again,} \emph{``Praise the Lord, all you Gentiles,} \emph{and let all
the peoples extol him.''} \emph{12 And again Isaiah says,} \emph{``The
root of Jesse will come,} \emph{even he who arises to rule the
Gentiles;} \emph{in him will the Gentiles hope.''}

God wants to reach the Gentiles and He wants them to glorify Him for His
mercy---not just glorifying Him because of His beauty and His strength.
Paul already explained in Romans 1:18 and further that God's invisible
attributes are already seen by man through the things He made, which is
all of the universe and everything here on Earth. But His mercy He
revealed through His Son to His people. Greatness can be revealed in
what one does by himself, but for mercy a second party is needed. Love
is what God the Father could give to His Son and to the Holy Spirit, but
mercy He could not, because Jesus never required any mercy nor grace.
Only to sinful people can mercy be granted. So through the sinfulness of
the Jewish people the goodness of Jesus is contrasted and the grace and
mercy of God the Father are shown. His mercy is shown in the withholding
of eternal death and His grace is shown in providing another chance to
them to be reconciled with Him through Jesus Christ.

\emph{13 May the God of hope fill you with all joy and peace in
believing, so that by the power of the Holy Spirit you may abound in
hope.}

But the Lord Yahweh is a God of hope. Through His Holy Spirit He
provides us this power to hope. Though hope is a powerful thing, one
needs a lot of power to obtain it and to contain it. \textbf{Obtaining
hope.} God grants us this hope, it is right before us, free of charge,
and one can read about it in the Bible. All it requires is one single
step: hand over your life to God, or in other words, put your trust in
Him. This means so much as doing something when God asks you to, or not
doing it when He tell you to. It means refusing a job, giving up a
career or breaking up with a partner with the possibility of not getting
a replacement for either of those. \textbf{Containing hope through
faith.} Yes, there is hope in getting a better \emph{alternative} for
those things we give up---even when it concerns our \emph{life}---but
this hope requires \emph{faith}. Now faith as well is a powerful thing.
With enough faith it is impossible for man to break you. History has
shown that man of faith were unbreakable, even when tortured, on the
verge of death or even when their family was killed in front of their
eyes. \textbf{Sustaining hope through faith.} Take for example the
\href{\%5Bhttps://en.wikipedia.org/wiki/Boxer_Rebellion\%5D(https://en.wikipedia.org/wiki/Boxer_Rebellion)}{Boxer
Rebellion} \#tags/historic events\# where so many Christian missionaries
were killed with not only their family being killed in front of them,
but also the long line of other Christian ministers and their families
before them. When one sees the body of Christ being slaughtered one by
one for hours in a row, people without faith would recant their belief
in God in order to be saved. But these missionaries and their families
persisted until the end. \textbf{Remaining faithful through hope.} They
never recanted because they knew what was waiting for them in heaven.
They never rebelled because they loved the Chinese people and attacking
them would only move them farther \#todo/opzoeken/engels of further?
away from God. \textbf{Gaining joy through hope.} During this period a
great number of missionaries fled and the Christian world mourned for
the loss of their Christian family as well the loss of the Chinese,
because no one reached out to them. A few years before that though,
Mrs.~Saunders from Melbourne, Australia, lost her two daughters,
Elizabeth Maud and Harriette Elinor Saunders, who were missionaries from
the Church Missionary Society in China, in another attack by Chinese
Buddhists, the
\href{\%5Bhttps://en.wikipedia.org/wiki/Kucheng_massacre\%5D(https://en.wikipedia.org/wiki/Kucheng_massacre)}{Kucheng
massacre - Wikipedia}, and decided to go instead of her daughters and
said the quote \textgreater{} \emph{I am sorry I do not have more
daughters to send to China}\\
She still could find the joy to serve the Lord in this great loss. That
is how powerful hope is. It is not something human or even humane, it is
supernatural and it can only be obtained from and sustained through the
Holy Spirit. See also
\href{\%5Bhttps://www.youtube.com/watch?v=-DSjhwICEB4\%5D(https://www.youtube.com/watch?v=-DSjhwICEB4)}{A
True Story of Missionaries in China}.

\textbf{Paul the Minister to the Gentiles} \emph{14 I myself am
satisfied about you, my brothers, that you yourselves are full of
goodness, filled with all knowledge and able to instruct one another. 15
But on some points I have written to you very boldly by way of reminder,
because of the grace given me by God 16 to be a minister of Christ Jesus
to the Gentiles in the priestly service of the gospel of God, so that
the offering of the Gentiles may be acceptable, sanctified by the Holy
Spirit.}

\emph{17 In Christ Jesus, then, I have reason to be proud of my work for
God. 18 For I will not venture to speak of anything except what Christ
has accomplished through me to bring the Gentiles to obedience---by word
and deed, 19 by the power of signs and wonders, by the power of the
Spirit of God---so that from Jerusalem and all the way around to
Illyricum I have fulfilled the ministry of the gospel of Christ;}

\#biblestudy/questions What is the relationship between verses 17 and
18? Paul says that he has a reason to be proud of his work, and what is
this reason? What does the ``for'' in verse 18 refer to? It could refer
to the fact that it is \emph{in Christ Jesus} that Paul has a reason to
be proud, or to the \emph{pride} or to the \emph{work of God}. The last
one seems to be most in line within the context. Paul has many things to
be proud of, but he only mentions his work for God---and even that is
\emph{in Christ Jesus}. Why does he not mention other things he is proud
of? He does not explain. He only says that he will not \emph{venture} to
speak of anything else. It is a strange sentence, because the answer
seems to be lacking. It is as if I am saying: ``I will talk about my
music skills, because I dare not speak of anything else.'' But why would
I not dare to do this? The answer seems to become more apparent in the
next verse.

In verse 20 Paul explains that he has already fulfilled the ministry of
the Gospel between Jerusalem and Illyricum---which is somewhere in
Hungary--- \#todo/opzoeken~~but not further than that. Therefore he
makes it his ambition to preach the Gospel in other places where Christ
has not been named yet.

So in short Paul says that he will only speak of his work for God,
because he will not venture or dare speak of anything else, because his
ambition is to preach the Gospel.

\emph{20 and thus I make it my ambition to preach the gospel, not where
Christ has already been named, lest I build on someone else's
foundation, 21 but as it is written,} \emph{``Those who have never been
told of him will see,} \emph{and those who have never heard will
understand.''}

Paul gained his ambition from the places where he saw that the Holy
Spirit was working, which is in ministering to the Gentiles. However, he
adds an extra condition to that and that is that it needs to concern
places where they do not know Jesus yet. This extra condition is founded
on two arguments: he does not want to build on someone else's foundation
and he wants to fulfill the foretelling from the Old Testament from
Isaiah 52:15. The argument is not a valid one in and of itself. Why
would it be bad to build on someone else's foundation? Did not everyone
do that when they continued \#tags/famouspeople Hudson Taylor's
\emph{China Inland Mission}? I am not certain of this, but perhaps it is
because of what Paul wrote in 1 Corinthians 1:13 about the division
occurring between people over who baptized them. Perhaps Paul did not
want to have any discussions about whose teaching it was they followed.
In a way, Paul would not prevent this problem, but only move it to other
persons. If he decided not to minister the people who had already heard
of Jesus, then another person would have to do that and then this person
would encounter the problems associated with teachings of another which
might need to be corrected. So I do not understand the first argument.
\#biblestudy/questions

\textbf{Paul's Plan to Visit Rome} \emph{22 This is the reason why I
have so often been hindered from coming to you.}

Paul's reason for being hindered is a valid one. He had his ambition
which was motivated by the purpose given to him by God. He knew it was
his purpose because he saw that what he did among the Gentiles was
blessed. Did he see that the fruit of his work at other places was not
blessed? I do not know. As far as I know the Bible does not speak on
that, but I can imagine that changing a culture could be harder than
introducing one, and changing a religion could be even worse. However,
introducing a new religion among the Greeks and other Gentiles is not an
easy or safe job either. It seems to me that Paul also liked what he
did. He felt that he needed to teach the basic teachings of Christ and
even of Judaism. It is what he was a Pharisee for. All pieces fall
together. He was instructed in academia, many of whom, and in those
times perhaps everyone, were to become teachers. So adding his talent to
his passion of teaching, his background of being both a Pharisee Jew
(who was strict about the Law) and a Christian and the spiritual
blessings of signs and wonders and words and deeds (perhaps Paul had
these respective gifts as well) it almost forms a perfect S.H.A.P.E. of
Spiritual Gifts, Heart, Ability, Personality and Experience---if I had
known what his personality is. From his writing I cannot extract his
personality. My writing, for example, is much different from how I
speak, and how I speak also depends on how I have prepared myself for
the situation. I can trust in God and I have my hope in Him that He has
given everyone a perfect SHAPE, but a Romans 8:28 already says, this is
only for those who love God. But even for Christians, the only way to
know one's SHAPE is to go like Paul did, and experience the wonders that
the Holy Spirit gives you, to find out if your passion and ambition lie
there and to see if you are suited to do the things you are supposed to
do there.

\emph{23 But now, since I no longer have any room for work in these
regions, and since I have longed for many years to come to you, 24 I
hope to see you in passing as I go to Spain, and to be helped on my
journey there by you, once I have enjoyed your company for a while. 25
At present, however, I am going to Jerusalem bringing aid to the saints.
26 For Macedonia and Achaia have been pleased to make some contribution
for the poor among the saints at Jerusalem. 27 For they were pleased to
do it, and indeed they owe it to them. For if the Gentiles have come to
share in their spiritual blessings, they ought also to be of service to
them in material blessings. 28 When therefore I have completed this and
have delivered to them what has been collected, I will leave for Spain
by way of you. 29 I know that when I come to you I will come in the
fullness of the blessing of Christ.} \emph{30 I appeal to you, brothers,
by our Lord Jesus Christ and by the love of the Spirit, to strive
together with me in your prayers to God on my behalf, 31 that I may be
delivered from the unbelievers in Judea, and that my service for
Jerusalem may be acceptable to the saints, 32 so that by God's will I
may come to you with joy and be refreshed in your company. 33 May the
God of peace be with you all. Amen.}

\#todo/nogaftemaken

\#biblestudy/devotionals/romans \#tags/events/easter

\hypertarget{romans-2-esv}{%
\section{Romans 2 (ESV)}\label{romans-2-esv}}

\textbf{God's Righteous Judgment} \emph{1 Therefore you have no excuse,
O man, every one of you who judges. For in passing judgment on another
you condemn yourself, because you, the judge, practice the very same
things.}

When passing a judgment on a transgression or sin this should apply to
everyone who commits this same transgression. This includes the person
who passed the judgment himself as well. This even goes for God. God has
put up a standard of holiness to which He ``abides'' Himself---although
He does not have to because He is the \emph{definition} of holiness,
righteousness and all His other characteristics.

\emph{2 We know that the judgment of God rightly falls on those who
practice such things. 3 Do you suppose, O man---you who judge those who
practice such things and yet do them yourself---that you will escape the
judgment of God?}

Now the problem is not that these people passed judgments. It is not
even a problem when these persons themselves are indicted for their own
trespasses. The problem starts when these people think that \emph{they}
are excluded. This too leads to false promises to the people when they
say: ``Do like I do and you will be on the safe side---on the
\emph{right} side of history.'' In fact, that is literally what the Left
has said. I am not sure about the Jews and Gentiles who passed judgment
here, but the Left is ever passing judgment on the Right for racism and
all kinds of other \emph{-isms}. The worst thing is that politicians and
policymakers know they are wrong and yet they stimulate people with this
lie of being on the right side.

\emph{4 Or do you presume on the riches of his kindness and forbearance
and patience, not knowing that God's kindness is meant to lead you to
repentance?}

This is an important verse because it not only shows God's character but
also one of His purposes for us. He wants us to be saved and be with
Him, and for that we need to be forgiven, and for that we need to have
repentance. God is patient and therefore He gives us time so that we can
rethink our actions, He gives us opportunities and minimized punishments
so that we have another chance for repentance in order to escape the big
punishment to come. God is forbearing and though He does not overlook
the small things, He does not require an equal payment for every sin,
though He could require---He pays for it Himself! God is kind and He
shows this by the many \emph{hints}, if you will, in life through which
we keep being reminded of repenting

\emph{5 But because of your hard and impenitent heart you are storing up
wrath for yourself on the day of wrath when God's righteous judgment
will be revealed.}

This part is meant for the Jews as is clear from verse 4. It is about
those who know God and know His kindness, although they mistake His
kindness for unconditional gifts. The condition we are speaking of is
explained in the previous verse as well and states that repentance is
necessary. If this is not done, then wrath is stored up. What does this
mean? First of all, storing means that this wrath will not dissipate by
itself. It will not diminish in value by any external factor, such as
good deeds. Only God can remove this wrath. It also means that if this
wrath is not diminishing then it is ever increasing because all people
sin daily and without repentance the punishment for all of this sin gets
stored as well. When will this barn open and show all the stored up
wrath? On the day of God's \emph{righteous} judgment.

\emph{6 He will render to each one according to his works: 7 to those
who by patience in well-doing seek for glory and honor and immortality,
he will give eternal life; 8 but for those who are self-seeking and do
not obey the truth, but obey unrighteousness, there will be wrath and
fury.}

God renders to each one according to His works. This does not mean that
we deserve eternal life or any other gift of God, not even temporal life
here on Earth or even one breath, but it does mean that God's gifts are
in line with what people have done. God gives gifts to everyone, He lets
it rain on both the wicked and the righteous, but eternal life will be
for those who seek glory, honor and immortality. You might wonder how
this applies for people who do not believe in Jesus Christ and have
heard of Him, or those who have not heard of Him. The answer to that is
given in verse 12, but from this verse we can already see a coherence
between following God through the Law and outside of the law, in the
requirements stated here for eternal life. The first requirement is
being patient in well-doing. There are lots of atheists who are patient
and persistent in well-doing and have sacrificed their whole life for
good causes. Now as for the cause, verse 7 states that this needs to be
glory, honor and immortality. People like Jane Goodall \#todo/opzoeken
who spent her life on protecting gorillas (cf.~the movie \emph{Gorillas
in the Mist}) may have done good things, all of them not self-seeking,
but had they brought about the aforementioned requirements? She had
definitely not aimed for immortality, but a case can perhaps be made for
glory and honor, although she wanted to protect gorillas because of her
own feelings for them, not because of a higher cause, which is God, for
Whom she wanted to protect nature. Now what is striking in the next
part, is that Paul specifies what these people \emph{do} obey. He does
not leave it at ``they do not obey the truth'' because that would cause
someone to think that they must then obey the lie. However, Paul makes
clear that it is \emph{unrighteousness} that they obey. The lie is
indeed the opposite of the truth, but the lie is a very broad concept.
The obey the lie by selling unrighteousness as righteousness. I have
already given the example of food coupons, another example would be
minimum wages and labor union (with excessive power and demands). The
Left demands minimum wages, which will cause the employer to hire less
workers. He will then have to work harder with less people, while those
who are fired, will have trouble finding a job because competition will
steadily rise. That is not a form of righteousness for either party.

\emph{9 There will be tribulation and distress for every human being who
does evil, the Jew first and also the Greek, 10 but glory and honor and
peace for everyone who does good, the Jew first and also the Greek. 11
For God shows no partiality.}

It is important to notice that \emph{the Jew first} means that the Jew
gets tribulation first but gets glory first as well. God does \emph{not}
show partiality. It is a distribution of responsibilities and higher
responsibilities go hand in hand with higher rewards, just like higher
risk, which has almost become like a law of nature in business
economics. The Jewish Christians carry the greater responsibility
because he has received the commandment to spread the Gospel first to
other peoples. Without this Gospel other peoples would not even have any
responsibility as Christians because there would not be any Christians.
The non-Christian Jews also carry a greater responsibility because they
have received the Gospel first and all the messages foretelling the
coming of the Messiah in the form of our Lord Jesus Christ. If they
decide not to do anything with that, they are foregoing two of their
responsibilities, which are to act on that which has been given them
(i.e.~the news of the Messiah) and to pass on this message to others.
Even though the Jews are God's children, God treats them fairly and
punishes them more harshly than He does other peoples and nations,
because the latter do not know Him nor His Law and have never heard the
message of the Messiah.

\textbf{God's Judgment and the Law} \emph{12 For all who have sinned
without the law will also perish without the law, and all who have
sinned under the law will be judged by the law. 13 For it is not the
hearers of the law who are righteous before God, but the doers of the
law who will be justified.}

There is an emphasis on doing. Martin Luther could not understand, after
his conversion, how doing the Law could save someone, as salvation is
only possible through grace. However, he did not see that salvation
entails conversion and conversion leads to good works and obedience to
the Law. So what happens then when a person converts but he does not
have enough time to live to do any good works? Then grace comes into
play. But when a person does have enough time His good works show the
fruit of his salvation. It is then not by this fruit that he has been
saved or justified, but by the work of God that caused the salvation, or
in other words, the salvation that God worked in us. So who are those
doers of the Law? They are those who have salvation worked in them, when
they know God. When they do not know God, it is those who listen to
their conscience. However, what do we say then of those people who do
everything perfectly according to the law of the country they live in,
even inwardly? There are two kinds of situations, one where they have
heard of God and one where they have not. In the first case, if inwardly
they have the right intention to abide by the Law, unlike the Pharisees,
and they practice it as well, I do not see why they would not come to
Christ. Ben Shapiro and many other Jews would fall in this category, as
well as many people who grew up in a Reformed environment but who have
not had the right teaching. This is a difficult case. In the second
case, when they do not have any knowledge of God, it is easier, because
then what is lacking about their faith can be attributed to this lack of
knowledge.

\emph{14 For when Gentiles, who do not have the law, by nature do what
the law requires, they are a law to themselves, even though they do not
have the law.}

The question arises what is meant by verse 14 when Paul describes how
the Gentiles are a law to themselves by doing what the law requires. In
my interpretation this means that for example male Gentiles
\emph{usually} marry females ones, so that is normal; Usually people are
free, and not slaves, so that is normal; Usually people are dressed and
sex is prohibited to one's spouse only and only within a marriage, so
that is the way it should be. Paul uses the argument of what is standard
within society to show that nature exists and the Law of God is already
written in human nature. By deviating from that, one is already doing
something against their conscience. By nature, suppose no one knows
where this comes from, one has to be pledged, betrothed and then get
married to one before starting a family with this person. One feels
instinctively that it is wrong to betroth one and then break up and
start a family with another woman, or to start two at the same time.
These laws come from God.

Modern society, however, especially the Left, claims that there is no
such thing as nature. They claim that ``everything'' is imposed by man.
Girls should behave like girls, because that is how we as a society have
put pressure on people, women should not stay home at work, but should
get a career themselves, etc. By denying there is such as thing as
nature, they can say that there is no such thing as ``going against
one's nature'' nor that it is wrong. However, they cannot disprove the
fact that both in nature as in mankind it always requires two people or
animals of the opposite sex to be able to reproduce. So this already
proves there is something as nature and that going against it will have
consequences.

\emph{15 ::They show that the work of the law is written on their
hearts, while their conscience also bears witness, and their conflicting
thoughts accuse or even excuse them:: 16 on that day when, according to
my gospel, God judges the secrets of men by Christ Jesus.}
\#biblestudy/memorization/todo

Now aside from what nature says, people also have a conscience and this
conscience causes conflicts in their minds. That too is proof of
nature's existence---or would the Left say that even the conscience of a
person has been indoctrinated by society? It would be strange then that
peoples and tribes all over the world have this same nature. What is
important to notice here is that these conflicting thoughts may accuse
them, but also \emph{excuse} them for sinning.\\
So first they have what we call nature, not the environment, but the
logical and natural way of order in this world, which everyone
knows---not \emph{feels} but knows---from what is taught, seen in nature
(the environment) and in the people around you. Secondly, they have
their conscience which tells them that something is right or wrong, even
when people around them do this. In the case of child sacrifices I would
expect that even when someone has grown up with this ritual, he would
not feel it is right to do such a thing. Thirdly, they have conflicting
thoughts as a cause of this conflict between conscience and nature. Now
as I interpret this, even when some sin, they could be excused, because
of these conflicting thoughts. For example, if a German soldier growing
up in WWII was severely indoctrinated from his youth on to see the
Jewish race as a threat, and all around him he would see the same thing
happening as he was taught, for example, he would see Jews killing
Nazis, then his conscience might still tell him that it is not right to
kill human beings, but it would be on the lesser side as everything else
in him is saying otherwise. This conscience is bearing witness against
him, but at the same time---I think---this person would be considered
innocent.

\emph{17 But if you call yourself a Jew and rely on the law and boast in
God 18 and know his will and approve what is excellent, because you are
instructed from the law; 19 and if you are sure that you yourself are a
guide to the blind, a light to those who are in darkness, 20 an
instructor of the foolish, a teacher of children, having in the law the
embodiment of knowledge and truth---}

Before we go on to the conclusion in the next verse, it is important to
note the summing up of functions that Jews should have. Paul is not
speaking to the scribes or to the Pharisees or any group specifically.
No, he is writing to the Jews in Rome, to \emph{every} Jew, whether he
has turned Christian or not. Of course there exceptions, because not
everyone can be a teacher, but in general we should all---as
Christians---be * a guide to the blind, * a light to those who are in
darkness, * an instructor of the foolish, * a teacher of children, The
last \emph{requirement} as I would call it, is that Jews should have in
the law the embodiment of knowledge and truth, which could mean that
whenever Jews (or Christians) are in a position to change or vote for a
law, they should do so according to the Bible---this goes for everyone
able to vote and for politicians---or it could mean that no matter what
the law is of the country, among Jews and Christians alike there should
be laws embodying knowledge and truth. This is a stark contrast with the
Left where they replace the truth with their feelings. Where they lie
even about the Constitution and deliberately interpret it otherwise. The
Left has replaced freedom of speech with incriminating ``hate speech''
where the latter is defined as anything they do not like to hear, which
goes right against the First Amendment \#tags/politics/amendments/first.

Either way, \emph{all} of the above are important. All of the above the
Left are not doing or doing wrong on purpose, but even Christians are
not doing all of these. They may teach their own children but not
others, they may do good things in their own community (i.e.~family and
friends), but they are not leading the blind (those with a different
world view) and shining light on those in darkness (such as beggars).

\emph{21 you then who teach others, do you not teach yourself? While you
preach against stealing, do you steal? 22 You who say that one must not
commit adultery, do you commit adultery? You who abhor idols, do you rob
temples?}

It is so easy to point at others but not do these things yourself. It is
also so easy, especially for teachers, to define a specific sin but not
accuse oneself because this specific sin was not committed by you in the
exact way you described. I must say I am guilty of this myself as well.
The more I know about the Law, the better I can find the loopholes in
it. However, this only goes up to a certain point. After a certain
point, the more you know about the Law, the more you realize that God
does not have any loopholes and if He had, He has many covers for those.
Love one another, love and fear God. Do not sin against your conscience.
This last commandment alone is enough for most people in the world to
stop sinning.

\emph{23 You who boast in the law dishonor God by breaking the law. 24
For, as it is written, ``::The name of God is blasphemed among the
Gentiles because of you.::''}

Now it is clear that all are sinning, as we will also see in the next
chapter (cf.~Romans 3:23) but who would be the worse sinner: the teacher
of the student, the ones who grew up and studied the Law, or those who
do not know about it or have just heard of it? Of course it is the
former. Now sinning against God is one bad thing to do, causing other
people to sin is another much worse thing to do, \emph{not evangelizing}
is another form of disobedience to the Law and this latter thing is
exactly what these Jews are doing---and once again, this goes for us as
well. The worst thing one can do is not sinning for yourself, because
that would only cause yourself to be removed from God's presence and
that is only one person. However, causing others to be removed from
God's presence by blaspheming God's name, especially to the Gentiles who
have no other way but to know God through you (unless God intervenes) is
probably one of the most worst sins one can do.

\emph{25 For circumcision indeed is of value if you obey the law, but if
you break the law, your circumcision becomes uncircumcision. 26 So, if a
man who is uncircumcised keeps the precepts of the law, will not his
uncircumcision be regarded as circumcision? 27 Then he who is physically
uncircumcised but keeps the law will condemn you who have the written
code and circumcision but break the law.}

A circumcision in itself does not have strength in the relationship with
God. Yes, the promise of God to Abraham has caused God to be ``more''
merciful and patient, if I may disrespectfully say so---without implying
that God is more merciful and patient to Jews than to the heathens---but
it is not because of the circumcision, but because of being ``of God'',
and one shows he is ``of God'' by being circumcised. Paul says here in
verse 25 that this circumcision, however, is as easily undone as it is
done. Also, circumcision does not atone for part of the committed sins.
Verse 27 makes this clear. It is more important to keep the law, even if
it is not an expanded law such as the Bible's, and not be circumcised,
than to have the Bible and break the Law.

\emph{28 For no one is a Jew who is merely one outwardly, nor is
circumcision outward and physical. 29 But a Jew is one inwardly, and
circumcision is a matter of the heart, by the Spirit, not by the letter.
His praise is not from man but from God.} If being Jewish is a matter of
the heart, then why would God distinguish between the Jew and Gentile
and the Jew first? I understand why the Jew gets the responsibility
first, because he knows more about it. I understand that one people
needs to be trained first, because that is the logical order of
processes---it always starts with one. So it follows from the
responsibility, that if one gets the responsibility first, he gets the
rewards first as well. That is why salvation (i.e.~Jesus) came from the
Jews and for the Jews first and then to the rest of the world. God did
decide to choose the Jews as His people as a consequence of His promise
to Abraham. He did that because He wanted a people who was different,
excluded and holy for Him, not to exclude \emph{others} out of it.
Everyone could join this group, provided they were devout in keeping the
commandments and devoted to God. People like Rahab (from Jericho), Ruth
(from Moab) and the two Samaritan women (at the well and the Jesus'
table) were all included to God's people after having pledged their
faith to the One true God.

\#biblestudy/devotionals/romans

\hypertarget{romans-3-esv}{%
\section{Romans 3 (ESV)}\label{romans-3-esv}}

\textbf{God's Righteousness Upheld} \emph{1 Then what advantage has the
Jew? Or what is the value of circumcision? 2 Much in every way. To begin
with, the Jews were entrusted with the oracles of God.}

Not coincidentally this is the same subject and question I was talking
about in the prior devotional. The advantage mentioned here is that they
were the first ones to hear about God and receive His revelations. Even
more, they were \emph{entrusted} with it. Now in itself the latter does
not have any advantages, but everyone knows that when a master, teacher
or parent entrusts you with something, it makes you feel proud, because
this person trusts you.

\emph{3 What if some were unfaithful? Does their faithlessness nullify
the faithfulness of God? 4 By no means! Let God be true though every one
were a liar, as it is written,} \textgreater{} \emph{``That you may be
justified in your words,}\\
\textgreater{} \emph{and prevail when you are judged.''}

The point made here is that God is by no means nullified. None of His
characteristics, especially not His faithfulness, are minimized by the
Jews' unfaithfulness. It can intuitively be seen that one's faithfulness
is dependent on the other party's faithfulness. This goes for everything
in life, business, partnerships and marriage. For example, what if a
husband pledges loyalty and faithfulness to his wife, but the wife keeps
committing adultery? The husband will not be unfaithful to his wife, but
he will separate himself from his wife, until she repents. God has not
become unfaithful, but He has separated Himself from the Jews until they
repent. God's Word is true, as are His promises. Even if everyone else
lies, God will keep His Word. God did not lie, He transparently foretold
the consequences of leaving God for idols or what would happen when the
Israelites hardened their hearts.

\emph{5 But if our unrighteousness serves to show the righteousness of
God, what shall we say? That God is unrighteous to inflict wrath on us?
(I speak in a human way.) 6 By no means! For then how could God judge
the world?}

There are many important points in this part. First, the fact is that
when everyone in the world sins, it only contrasts the goodness of God.
For no matter how hard He is pressured, He will not sin, just like Jesus
showed us on Earth under immense pressure and unrighteousness. Can it be
stated then that our unrighteous was ``good'' or a so-called
\emph{necessary evil} to show the goodness of God? If we were not
unrighteous, then no one would know the goodness of God. The latter may
be true in theory, but in practice no one is righteous. Unlike some
stories or movies where a father introduces some evil in his child's
life in order to train the child in such a way that the child will
become stronger or see the contrast between real enemies and his strict
father, God does not need to do this. There is enough evil in this world
already. If anything, He has been withholding evil from our lives.

Now then comes the human question why God is righteousness in inflicting
wrath on us, while we were bringing glory to Him? There is a difference
here between you bringing glory or the situation bringing glory. When
you do something wrong and God needs to save you and others, it is God
Who used the situation to bring glory to Himself. You cannot say that
you did something good. No, in fact, you put people in harm's way. That
is why you deserve punishment. So God is not wrong if He inflicts harm
on you.

\emph{7 But if through my lie God's truth abounds to his glory, why am I
still being condemned as a sinner?}

As for lying. It is through the many lies that God used people like Josh
McDowell to revive the topic apologetics, research what had been written
about this in history but also prove once again to many ``Christians'',
agnostics and atheists alike that the claims of God in the Bible are
real and true. Did those who lie then contribute to any of this? In part
of the equation, the answer is \emph{yes}. Without those lies Josh
McDowell would not have given God glory in this way. But in the total
equation the answer is \emph{no}, for they have caused so many to fall
away, throughout all ages, which is most likely irrecompensable by the
number of new Christians that are reborn through Josh McDowell's
ministry.

\emph{8 ::And why not do evil that good may come?::---as some people
slanderously charge us with saying. Their condemnation is just.}

This brings us to the question of Buddha asking whether (or stating
that) it would be righteous to kill a person if he knew this person
would kill others. God has not stated an explicit law for this kind of
situation in the Bible, but He has placed similar \emph{situations}
within of which we only know the outcome in hindsight. Saul was supposed
to kill everyone, men, women and children in the war with the Amalekites
\#todo/opzoeken , as were others in the Bible in certain occasions.
Whenever this did not occur, we see the repercussions of this. Joshua
did not kill the Gibeonites and later on this people demanded the death
of seven of Saul's sons, which would be the end of his whole line except
for Mephibosheth. I am not sure if the Israelites were ever intended to
kill all of the Edomites, but that guy in Esther who hated the Jews, was
an Edomites, as was Herod, who killed Jesus and who was a traitor for
letting the Romans rule Israel. But those who used verse 8 were not
talking about such situations, and they were {[}\#todo/opzoeken/engels\#
helemaal al niet{]} not saying this for a selfless cause\\
---out of altruism that is. No, it is the complete opposite. They want
to do what they want, no matter how many people get hurt in the process.
Then they do not want to accept the consequences, i.e.~be punished, but
rather protest that either God is unrighteous for punishing them for
creating a situation in which He can be glorified, and perhaps they even
want the glory for that as well. The movie \emph{Unbreakable} shows this
plot as well, when Samuel L. Jackson tries to justify his criminal
actions to find the superhero.

\textbf{No One Is Righteous} \emph{9 What then? Are we Jews any better
off? No, not at all. For we have already charged that all, both Jews and
Greeks, are under sin, 10 as it is written:} \#todo/opzoeken waar?
::\emph{``None is righteous, no, not one;\emph{:: ::}11 no one
understands;\emph{:: ::}no one seeks for God.\emph{:: ::}12 All have
turned aside; together they have become worthless;\emph{:: ::}no one
does good,\emph{:: ::}not even one.''}:: \emph{13 ``Their throat is an
open grave;\emph{ }they use their tongues to deceive.''} \emph{``The
venom of asps is under their lips.''} \emph{14 ``Their mouth is full of
curses and bitterness.''} \emph{15 ``Their feet are swift to shed
blood;\emph{ }16 in their paths are ruin and misery,\emph{ }17 and the
way of peace they have not known.''} \emph{18 ``There is no fear of God
before their eyes.''}

Note very carefully that this is not only meant for Jews or for the bad
Jews. No, it is for everyone, Jews, Greeks and Christians alike---about
our flesh-like state.

In everyone, both Jew and heathen, all of the above statements are
valid. These statements are no small matters. They are accusing of
killing people. Their feet are swift to shed blood. Does that not mean
that we are not only willing, but easily willing to kill people? Perhaps
even worse, we put people in ruin and misery, and let them suffer their
whole lives. Now we can think of some bad people responsible for those
actions, like politicians and mighty people such as George Soros, who
bought up businesses to sell them afterwards for a huge profit. Then
what about their throat being an open grave with mouths being full of
deception, and curses and bitterness? These are not directly killing,
but they have a tremendous effect on people's lives. Bullying and verbal
abuse for example will cause people's character to change and the whole
course of their life. If these do not cause depressions in one's life
then they will cause some kind of sociopathic behavior.

\emph{19 Now we know that whatever the law says it speaks to those who
are under the law, so that every mouth may be stopped, and the whole
world may be held accountable to God.}

No arguments are possible for God's infallible reasoning. The law is
enough proof of man's sin. Man could provide many arguments to prove his
innocent in the case of many transgression's, but never for all---and
that is what matters here. Even one sin alone could send one to hell.
But even more so, it deems one \emph{accountable to God} and \emph{that}
is what man should know. So, summarizing, man has nothing to say about
his accountability to God. He is the only one they should be accountable
to.

\#todo/opzoeken MacArthur rechtszaak

\emph{20 For by works of the law no human being will be justified in his
sight, since through the law comes knowledge of sin.}

The law is not a method or way to be justified, first of all because no
one can keep all of the law, but second of all because whatever
transgression you had which was not sin because you were not aware of
it, now \emph{becomes} sin because you have been made aware of it by the
law.

\textbf{The Righteousness of God Through Faith} \emph{21 But now the
righteousness of God has been manifested apart from the law, although
the Law and the Prophets bear witness to it--- 22 the righteousness of
God through faith in Jesus Christ for all who believe.}

Now God still provides a way to get to Him, and though it is not
\emph{through} the Law, the Law and the prophets, who proclaimed the
Law, are part of the plan. The plan is, as we all know, through faith in
Jesus Christ. Now in what way would the Law and the Prophets be
involved? The Prophets indicate the books of the small and big prophets
here, and in those books the Law was proclaimed to the people of Israel,
and other nations, and the consequences of not following it or setting
your heart on God. It is through the Law that we can see how sinful we
are and therefore we see how great the forgiveness is of God. A reading
of the Law alone as in the books of Moses would not suffice in our
understanding of the gravity of our transgressions.

We would think per Leviticus that the sin of the people amounts to the
sacrifice of bull, while the sin of a leader amounts to that of a goat.
But we need the warnings of Jonah and the wailing of Jeremiah, and among
others the suffering of Joel and Ezekiel to show how serious grave the
punishment is for us. The Jews kept thinking they were not as bad as the
Babylonians or the Assyrians, but Paul is saying here in Romans that
they are, everyone is! We also need the prophets and the Law to show
that atonement is the only to get closer to God and the prophets point
us to the Messiah who makes all of this possible. Personally, I have not
been able to find passages about the Messiah coming to forgive us all,
only that the Messiah will be powerful and rule with an iron scepter and
that He will be sacrificed like a lamb (cf.~Psalm 2 and Isaiah 51). The
sacrifice mentioned here, however, does not say specifically that it is
for the sin of man. On the other hand, I am not a theologian and I have
not studied the whole Bible yet. \#todo/opzoeken foretellings of the
Messiah.

\emph{For there is no distinction: 23 ::for all have sinned and fall
short of the glory of God, 24 and are justified by his grace as a gift,
through the redemption that is in Christ Jesus,::}
\#biblestudy/memorization \emph{25 whom God put forward as a
::propitiation:: by his blood, to be received by faith.}

God put something forward as propitiation to Himself. That is like
telling someone what to buy for one's own birthday. That alone in itself
is a strange thought, but what if this person even pays for his own
present? Now this happened in the case of Zeus, who put forward a
beautiful white bull to be offered to him by king Minos and he was
furious when it was not offered. The reason for offering, however, was
because he wanted to be worshipped, which was what kept the gods alive.
Zeus did not case about king Minos at all. This is the big difference
between Zeus and God. God does care about human beings, so much so that
he did not \emph{create} a perfect animal to be offered, but He
sacrificed Himself (because God does not do wealth creation. If He did,
it would have been an easy sacrifice for Him). And after He offered
Himself to all humans, the thing He was angry about was not that He had
to die, but that some did not even want to \emph{accept appeasement}
with God! He died so that mankind could be united with God but most of
mankind does not even want unity. Those who do not want it then try to
influence and convince the few remaining people who do want it, and when
that does not succeed they create a web of lies---almost literally, in
the \emph{world wide web} of the Internet---and spread lies about God,
Jesus, Jews and all of history, even in science---just to remove any
evidence of God's presence.

Now all that God asks us in return for receiving this propitiation, is
faith---faith in that Jesus is righteous and that God has saved us
through His Son Jesus and that His salvation covers all sins and is
everlasting. Is that so hard? Would it be like receiving a car only upon
having faith that this car is faster than all other cars? That is a hard
thing to believe, but still it is different from what God asks of us.
For in the case of salvation we ask God for eternal life, which already
is something impossible and non-existent according to the secular world.
God then says, \emph{if you want me to give you eternal life, the least
you have got to do is} believe \emph{that I can give it.}

\#todo/opzoeken Make list of statements that say that Jesus is God, both
by Himself and by His disciples.

\emph{This was to show God's righteousness, because in his divine
::forbearance:: he had passed over former sins. 26 It was to show his
righteousness at the present time, so that he might be just and the
justifier of the one who has faith in Jesus.}

Why is it so hard to see that God is fair. Why does the whole world
think that God is unfair for not saving everybody but that instead He
\emph{requires} something. They already find it strange that Jesus had
to die for us, while God could have just wiped away our sins. That is
because they do not understand the logic of consequences and payments.
But aside from that, even after Jesus has died, why does God not give
this offer to everyone? God has indeed selected a few, but that is
outside the scope of this question. What they really want to know is why
God does not simply provide his offer to everyone and then everyone will
be saved, whether or not they have given their lives to Jesus Christ or
even have faith in them. ::The first answer to this question is answered
in our own lives. Do we do things for others when they do not believe in
us?:: \#todo nog af te maken

Another related answer is, is why God would let anyone in heaven to live
with Him if this person does not even want to live with God? Another
reason is that God only allows people of good character in heaven. Now
no one has a good character, except for the Father, but God can still
accept the imperfect characters and make them perfect when they arrive
in heaven. Can he not do this then with people who do not believe in
him? Many have good characters according to the standard of this world,
although many have very bad characters either. If God loves them, why
can He not give them the free gift of eternal life, for which He has
already paid? The answer is love. It is exactly \emph{because} God loves
them that He does not want to force people to go into heaven when they
do not want to. Here on Earth the question has already been asked of
them whether they want to go to heaven. Is there any atheist who says he
wants that? Is there any atheist who says he wants to be with God all of
his life? If there is, they why does he not start doing so right now?
::Now what if an atheist says he does not want to know God, but if---it
appears---that in the second after his death he comes to find that the
Bible was true after all, and he repents of not having known God, will
God then forgive Him?:: \#todo nog af te maken \#biblestudy/questions
Misschien is het antwoord dat je nog je lichaam kunt trainen en testen
zolang je nog een lichaam hebt. In de hemel is dat niet meer mogelijk.
Een ander antwoord is dat als je in de hemel bent er geen platform is
waar je nog kunt kiezen tussen de hel of de hemel. De rijke man in de
parabel met Abraham, zat in een domein van waar hij niet meer naar de
hemel kon. Vanaf de dood is het een Y-splitsing.

The answer as far as I can get this from the Bible is \emph{no}, because
God thinks people have had plenty of opportunity to repent. Either
without the Bible they would have known what is bad, and with the Bible
they would have had enough situations in which they could feel the love
of God or perceive the logic of God. It is like an evolutionist who at
some point in his career sees that his theory of evolution does not fit
the results and starts doubting the validity. If at that point---and God
is the judge of whether that doubt is strong enough to give rise to a
consideration of alternative theories---he does not give in to this
doubt as a \emph{rational} thinker---for he is an evolutionist and
therefore claims to be and must be rational---he forfeits this chance of
knowing God. He might get more chances in life, God is very merciful,
but it still is one chance he has forfeited.

\emph{27 Then what becomes of our boasting? It is excluded. By what kind
of law? By a law of works? No, but by the law of faith.}

\textbf{Example.} Man who is claimed innocent by judge who puts guilt on
his own actually innocent son (i.e.~Jesus). Will this man have fame? Of
course not. The journalists outside the court will despise him but the
judge as well. Example of ``cosmic child abuse'' quoted from atheist.
\#todo/nogaftemaken \#church/material/sermon

The question posed here in verse 27, ``\emph{By what kind of law''},
means that when one says it is not acceptable to be boasting, this means
that one has to provide a reason for that. So when I tell you you should
not boast, I will also tell you why, and that is, for example, because
there is someone who can do it even better than you can, or because you
cheated, so it does not count. Here we also see the word''count'' and
one might ask again, \emph{according to whom does this not count?} The
same goes for verse 27. Why is boasting excluded? Because of the law of
faith. Now that does not really answer the question, most certainly not
for unbelievers. The premise here is that some have been justified,
i.e.~their sins have been removed. Now if one has had faith and done
works in in his lifetime, God sees both and accepts this faith as
sufficient to grant you justification and eternal life. He did not
consider the amount of works as a payment for your salvation. It would
be just like handing in two exam papers, both with the same name. Let us
say the first one was done by someone else, but the second was done by
you. The first one passes the test while the second one does not. You
cannot boast about passing when the test was not done by you. Now in
this case God sees our faith and that passes the test, and our works
never pass the test, because that is literally impossible. So we cannot
boast in our works. Can we then boast in our faith? No, neither can we
do that for it is God Who works in us to act and to will (cf.~Ephesians
2:10 \#todo/opzoeken ) and even our faith is imputed unto us.
\#church/material

\emph{28 For we hold that one is justified by faith apart from works of
the law. 29 Or is God the God of Jews only? Is he not the God of
Gentiles also? Yes, of Gentiles also,}

Another question is asked here that starts with ``or''. This is a
rhetoric question to prove that God is the God of Gentiles as well. Now
what does that have to do with justification? It is connected to what is
said in the previous verse. It is the second reason why boasting is
excluded. This reason is that it would not be fair for Gentiles to
compete in this race. Gentiles would not be able to boast in their works
of the law, because they do not have the law that Jews have. Yes, as
mentioned before, they have their own law, the law written on their
hearts, by which they will be judged, but the law by which they should
be justified is the Law of Moses, c.q. the Law of God. So if boasting in
the Law was not excluded (by faith) then the Jews could boast in their
works, but only because the Gentiles did not have this work. This would
be like boasting in the results of your exam while you have your books
to learn from but others do not. Passing your exam then is hardly
something to boast about. Then again, everyone would fail this exam, for
no one would be able to achieve all the works possible.

\emph{30 since God is one---who will justify the circumcised by faith
and the uncircumcised through faith. 31 Do we then overthrow the law by
this faith? By no means! On the contrary, we uphold the law.}

What does verse 29 have to do with the Trinity of God? This unity refers
not to the tri-unity of God but of the unity between people. God is also
one in that He treats people in the same way, as verse 30 explains.
Verse 31 then concludes that if the law is not used to justify anyone,
then why do we have it?

\#biblestudy/devotionals/romans

\hypertarget{romans-4-esv}{%
\section{Romans 4 (ESV)}\label{romans-4-esv}}

\textbf{Abraham Justified by Faith} \emph{What then shall we say was
gained by Abraham, our forefather according to the flesh? 2 For if
Abraham was justified by works, he has something to boast about, but not
before God. 3 For what does the Scripture say? ``Abraham believed God,
and it was counted to him as righteousness.''}

Genesis 15:1-6 said that Abraham believed in Yahweh.

\emph{4 Now to the one who works, his wages are not counted as a gift
but as his due. 5 And to the one who does not work but believes in him
who justifies the ungodly, his faith is counted as righteousness,}

This is very difficult to explain. \emph{If} Abraham was justified by
works, and yes, he must have done a lot of good things---he listened to
God and he took the step to go to the promised land---so he could boast
before us, but he could not boast before God, for several reasons. First
of all, it is never possible to boast before God because He is always
immeasurably better than we are. Second of all, God \emph{credited}
salvation to him as righteousness. Verses 4 and 5 then explain how we
can tell from this line whether Abraham worked or whether he believed.
It explains that in the former case it is the worker's due and in the
second case it is righteousness that is given to a person.

\emph{6 just as David also speaks of the blessing of the one to whom God
counts righteousness apart from works:} \emph{7 ``Blessed are those
whose lawless deeds are forgiven,\emph{ }and whose sins are
covered;\emph{ }8 blessed is the man against whom the Lord will not
count his sin.''}

Paul then goes on to say that those who \emph{receive} this
righteousness \emph{apart from works} but by faith---that is, the people
who did not earn their righteousness by works but by faith (of which
there are none by the way)---are blessed because they are forgiven for
their sins. Or put in other words, the one who \emph{receives}
righteousness is blessed because his sins are forgiven. What do we say
then of the man who worked for his salvation? He does not need
forgiveness because he has no sin, so he does not need to receive this
blessing and thus this blessing would not apply on him.

\emph{9 Is this blessing then only for the circumcised, or also for the
uncircumcised? For we say that faith was counted to Abraham as
righteousness. 10 How then was it counted to him? Was it before or after
he had been circumcised? It was not after, but before he was
circumcised. 11 He received the sign of circumcision as a seal of the
righteousness that he had by faith while he was still uncircumcised. The
purpose was to make him the father of all who believe without being
circumcised, so that righteousness would be counted to them as well, 12
and to make him the father of the circumcised who are not merely
circumcised but who also walk in the footsteps of the faith that our
father Abraham had before he was circumcised.}

The text here explains clearly whether this blessing is for the Jews or
also for the rest of the world. I could not explain it any better. God's
purpose is clear. He has always wanted the whole world to be saved and
this message had to be started through one man, Abraham. Paul clearly
explains that if Abraham received righteousness before his circumcision
then he would be the father of both the circumcised and the
uncircumcised. The uncircumcised would then know that even an
uncircumcised person is able to receive righteousness and the
circumcised person would know that righteousness is given to those who
are like the first circumcised person, i.e.~Abraham, who walked by
faith.

\textbf{The Promise Realized Through Faith} \emph{13 For the promise to
Abraham and his offspring that he would be heir of the world did not
come through the law but through the righteousness of faith. 14 For if
it is the adherents of the law who are to be the heirs, faith is null
and the promise is void. 15 For the law brings wrath, but where there is
no law there is no transgression.}

God made Abraham a promise to become heir of the world. Theoretically
there could have been two ways in which God could have made Abraham
heir, through the law or through righteousness of faith. If it were the
former, then it would not really have been a promise from God, for
Abraham would have worked for it himself. It would be a void promise,
like a parent promising his child: ``I promise you, you will pass your
exams.'' Unless the parent does all he can to bribe the examinators and
knows it will succeed in it, there is no promise in here. First of all,
it is not a promise because passing the exams is not something the
parent can do. A promise is only valid when someone actually \emph{does}
something. Second of all, if there is no certainty it cannot be called a
promise---at least not one from God. So we conclude that it must be the
latter. \emph{God} works and makes sure Abraham receives what God
promised him. We could have gotten to this conclusion through another
way and that is that Abraham could never live a life without sin, even
if one were only to count the life after God's promise.
\#church/material

\emph{16 That is why it depends on faith, in order that the promise may
rest on grace and be guaranteed to all his offspring---not only to the
adherent of the law but also to the one who shares the faith of Abraham,
who is the father of us all,}

There is also a \emph{reason} why God wanted this promise to come
through righteousness of faith, and that is because that is the only way
God can guarantee all of Abraham's offspring. Abraham living a
tenaciously sinless life does not guarantee that his son would do the
same. As a matter of fact, Isaac was a little worse and both of his sons
were far worse in being upright---Jacob was changed by God, not by
Isaac's teaching.

\emph{17 as it is written, ``I have made you the father of many
nations''---in the presence of the God in whom he believed, who gives
life to the dead and calls into existence the things that do not exist.
18 In hope he believed against hope, that he should become the father of
many nations, as he had been told, ``So shall your offspring be.'' 19 He
did not weaken in faith when he considered his own body, which was as
good as dead (since he was about a hundred years old), or when he
considered the barrenness of Sarah's womb.}

Abraham kept his faith and he is credited a lot of honor for this by the
Jews, who see him as their father, because he had faith in having
another child even though he was a hundred years old and his wife was
90. So tremendous was his faith, because nothing like this had happened
before, as far as I think the Jews knew about. All other examples of
barren women, like Rachel, Hannah and Elizabeth would come hundreds or
thousands of years later. Those women could have faith like Abraham did
on behalf of Sarah, who at first did not believe (but probably did
afterwards, after finding out that these men were angels). For this
perhaps just as much faith is needed as believing that God can raise
someone from the dead. In fact that is what God has done and I think
that somewhere the Bible explains this as well. Elizabeth's womb as good
as dead, Abraham's body as well. Now if God can do those things, He can
resurrect people as well.

\emph{20 No unbelief made him waver concerning the promise of God, but
he grew strong in his faith as he gave glory to God, 21 fully convinced
that God was able to do what he had promised.}

What is especially remarkable is that Abraham did not have any unbelief.
Even though he had a child with Hagar, I think that could either be
credited to Sarah's unbelief and Abraham's obedience to her, or
Abraham's wrong thinking in that that child would be God's, but never
had it been the case that Abraham doubted God's ability to bring forth a
child from Sarah's womb.

\emph{22 That is why his faith was ``counted to him as righteousness.''
23 But the words ``it was counted to him'' were not written for his sake
alone, 24 but for ours also. It will be counted to us who believe in him
who raised from the dead Jesus our Lord, 25 who was delivered up for our
trespasses and raised for our justification.}

\#biblestudy/questions wat wordt er bedoeld met vers 22 waarin staat dat
het geloof aan Abraham wordt toegerekend uit rechtvaardigheid? Heeft
Abraham iets gedaan waardoor hij zo'n sterk geloof gekregen heeft?

These words were written so that we may know that what happened to
Abraham is available to us as well. First of all for the Jews, because
they read it first and because they have received the promise, but
second of all for us as well, because this promise is dependent on faith
and to anyone who has faith.

\#biblestudy/devotionals/romans

\hypertarget{romans-5-esv}{%
\section{Romans 5 (ESV)}\label{romans-5-esv}}

\textbf{Peace with God Through Faith} \emph{1 Therefore, since we have
been justified by faith, we have peace with God through our Lord Jesus
Christ. 2 Through him we have also obtained access by faith into this
grace in which we stand, and we rejoice in hope of the glory of God.}

There are two things that we have been given by Jesus Christ which lead
to two outcomes. 1. We have been justified and therefore we have peace
with God. 2. We have access to His grace and therefore we rejoice and
have hope.

Note that the first outcome only speaks of peace \emph{with God}---which
is an external, objective reality (see notes in \emph{MacSBNnkjv})---it
does not say peace \emph{of heart}, which is a subjective, internal
sense of calm and serenity. We can be sure now that God is not angry
with us, as we have been justified (the first point), but do we have
peace with our own sins of the past and our current ones? Do we have
peace with who we are, what we are able to do and what we are worth? Do
we have peace with our circumstances, whether rich or poor
(cf.~Philippians 4:11-13)? This first part is very important and it is
actually all that matters if we want to access heaven and be with God
eternally. One needs \emph{access} and access is only given to those who
have been judged as not guilty. Through this justification, by
\emph{faith} in Jesus Christ, who has paid the full ransom for us, we
have now been declared sinless at the Day of Judgment (note that the
``have been'' here refers to a past tense, although the Day of Judgment
is in the future. This logical fallacy does not apply though for God.)
So even though we are not in heaven with the Father right now, we know
that we will be. Jesus is accumulating all of our sins right now up to
the day we die. He already has paid for it on the cross, but on that day
he will deduct it from the price He has already paid at the cross. Even
if we do not have peace at heart, we have peace with God. Our lack of
peace at heart comes from the fact that we have not fully changed yet.
Our bodies and soul have not been made new yet, but the Holy Spirit
lives in us and cleanses it as He works. God is willing to put His
Spirit in an unclean body like ours, because He knows that this body is
one that will be cleansed at the end of ages. So we have more peace than
before we came to Christ, but still we are not fully peaceful. This
leads us to the second part. We have access to God's grace. Jesus Christ
has paid the price in full, and He could give us all that He has
promised us when we go to heaven. But He did not. He is giving it to us
right now. He is carrying the burden with us and lightening our load. He
has already forgiven us and taken the shame from us. This is one reason
why we can rejoice. Through His grace we have received the Holy Spirit
in \emph{advance}, He was sent to us as a bond, as bail or a deposit or
a security \#todo/opzoeken/engels borg. God's promise should suffice,
but He has given us certainty in our faith and also shows this through
the many miracles and powers and wisdom we get from the Holy Spirit. Now
this gives us hope as well. We could live a whole life resting on the
promise of God, but never seeing any of this. We could never see it from
anyone, except from those who returned from the dead, which are but a
handful. But now there is no need for this, for God has sent His Holy
Spirit to tell us directly what He is saying to us.

::\emph{3 Not only that, but we rejoice in our sufferings, knowing that
suffering produces endurance, 4 and endurance produces character, and
character produces hope, 5 and hope does not put us to shame, because
God's love has been poured into our hearts through the Holy Spirit who
has been given to us.}:: \#biblestudy/memorization

What God is saying, He is saying most clearly in our sufferings. As the
above verses explain the meaning of suffering and how we can rejoice in
it, it ultimately leads to hope. \emph{With} hope suffering is much more
easily to endure and after having endured it, suffering provides hope to
see that the suffering has an end. The first parts described in verses 3
and 4 I will skip for now, because they seem rather logical and
consequential. Verse 5, however, describes a strange consequence, namely
that ``hope does not put us to shame, because God's love has been poured
into our hearts''. What does that mean? Being put to shame means that we
are ashamed of something because something does not come. Suppose a
father promised his young child that he will come back, or even specify
the time and says that he will be back before his birthday, and after a
while the father still does not appear, then this boy will be put to
shame by his friends or opponents laughing not so much at him but rather
at his (lousy) father. However, if he has love for the father, then in
that love there is trust, and the boy will not be ashamed, because he
trusts his father to have said the truth. Now if this is heavenly love
given to us through the Holy Spirit, then this trust and hope is
stronger than human love.

\#todo/nogaftemaken waarom ``for'' hier? Zie de vorige zin.

See also notes on 2 Corinthians 1:4-7.

\emph{6 For while we were still weak, at the right time Christ died for
the ungodly.}

God always has the right timing. We see this clearly depicted in Jesus'
life, how He meets people at the right time, heals them eats with them.
It is all fitting and part of His three (and a half)-year ministry to do
all of it as efficient as possible. He even knew the devil would stop
Him but that did not stop Jesus from sticking to His plan. His plan has
been made, purposed, and it will stand (cf.~Isaiah 14:24). We might be
wondering, however, why the year of approximately 35 AD was the right
time for Christ to die. There are many things that can be said about it.
If Jesus had been born later, He would have been born under another
Herod, who would not have been so cruel to kill all boys under two years
of age. This was to fulfill what was foretold by the prophets.
\#todo/opzoeken rama? where? Other reasons were the spread of the
Gospel. If Jesus were to have died later, the Gospel would not have had
enough time to flourish and the extinction by the Jews, Romans in and
outside of Israel, and the Jewish War, which most likely also killed a
lot of Christians, must have led to another outcome. If Jesus died
sooner then He most likely would have been born sooner as well, assuming
He must start His ministry at age 30 and must fulfill a period of three
years of discipling in order to teach His disciples well---this in
itself is also an explanation. If Jesus had died later, He would have
spent more than three years with His disciples, which would have been
unnecessary (even though there is always more to be learned from Jesus)
because His disciples would have received all the knowledge they need to
continue their way. If Jesus had been born sooner, then His parents,
Joseph and Mary, would have had to stay for a longer period in Egypt to
wait until Herod had died, which would not have been God's preference
(is my guess) as God did not want His people to ever go back to Egypt.
If Jesus had been born before Herod the Great's rise to kingship then
the temple would not have been built and Jesus would not have been able
to be blessed there.

Now this statement in verse 6 is probably only about God's perfect
timing for all of the above-mentioned reasons, although these are
reasons (I think) that God chose this time but it is also because
humanity could not live longer without the coming of Jesus. We see
throughout history that the Israelites started to abandon God in Egypt
after a certain period, which was 400 years, by worshipping idols. The
fact they did that in the desert most likely means they were doing that
in Egypt as well. God wanted them punished, but not for too long, so He
brought the Jews into exile in Babylon, but there, as well, not for too
long or they would follow the gods of Babylon and Persia. Now, back in
Israel, God refrain from any contact with the Jews, but here too He
could not wait too long, for after 400 years---though the Jews were not
worshipping idols in the literal sense---the Jews started to worship
their own kind of idols, such as the Law (through legalism) and
Pharisees, scribes and priests started their own kingdoms and cults,
having both money and power, and clashes began to occur between the four
different sects of the Pharisees, Sadducees, Zealots and Essenes. I am
not sure if this was also one of the country's lowest points in their
economy with a great division between the rich and the poor, but from
what can be read in the New Testament poverty seemed to be present
everywhere.

\emph{7 For one will scarcely die for a righteous person---though
perhaps for a good person one would dare even to die--- 8 but God shows
his love for us in that while we were still sinners, Christ died for us.
9 Since, therefore, we have now been justified by his blood, much more
shall we be saved by him from the wrath of God.}

A version of the part above is: ``\emph{God shows His love for us by
dying for sinners.} People are selfish and they show that through the
sacrifices. Scarcely would one give up his life for other, but if one
does it, it is (usually) for good people. I have heard this story from
Francis Chan about one of his (?) congregation members who donated his
kidney to a stranger---a man whom he's only just met. Now that is a very
heroic and rare, but still, no matter how good he is, he---most
likely---gave this kidney because he liked the person, perhaps because
the person was a good man as well and because he had a family. I
presuppose there is something emotional behind it. I do not suppose it
is to convert him---even though he was converted by this act of
love---because that would not be a sustainable method, however, I do
think that this man's story touched the donator. Perhaps it was about
how sad his life was, or how he could not provide for his family or find
a suitable job having only one kidney and needing constant dialyses in
the hospital. So this person was touched, and his conscience and
emotions spoke to him. However, what God has done is far greater than
this. He gave His life to someone who disliked Him and literally nailed
Him on the cross.

\emph{10 For if while we were enemies we were reconciled to God by the
death of his Son, much more, now that we are reconciled, shall we be
saved by his life.}

The logic used here is superlative reasoning \#todo/opzoeken wat is de
echte term hiervoor? If God treats his enemies with method \emph{A} and
He treats His loved ones with method \emph{B}, then \emph{B} must be
better than \emph{A} given the assumptions about God and logic
reasoning. So God is willing to give His own Son's life for His enemies,
then what would He be willing to do for those who are reconciled with
Him? The logic used here is not perfect, because it implies two
(consecutive) gifts. If God were to give only one gift to each person
then the enemy of God would receive reconciliation, and the friend of
God (for whom no reconciliation is needed) would receive salvation. One
cannot use the argument that everyone will salvation because the enemy
first receives reconciliation and because he is reconciled with God he
receives salvation as well. That would only apply if God were to give
two gifts. However, Paul is implicitly stating that the gift of Jesus,
which is His life, is big enough to provide both reconciliation and
salvation. \#todo nog af te maken Jesus' gift reconciles us with God and
make us His children, and because we are His children, we are now under
His protection. It is like buying Roman citizenship. You pay one price
to get it (even though it is not legally obtainable but only through
corruption) but obtaining this status is not enough, one needs also the
benefits of it, and that is what one gets when paying this price. Now
obtaining reconciliation with God is a good thing in and of itself, but
what comes with that is direct communication with Jesus Christ and
therefore with God, direct contact with God through the Holy Spirit in
order to receive all of God's attributes such as His peace and love and
comfort.

\emph{11 More than that, we also rejoice in God through our Lord Jesus
Christ, through whom we have now received reconciliation.}

One thing I have not mentioned in the last line above is joy. Joy is one
of the foremost traits of being a Christian and what convinces others to
become one. I am not sure if it seeing others being so joyful and then
wanting to become one, or whether it is that one feels the joy of
Christian momentarily and then decides to want this all of his life. I
think it is rather the former or otherwise a shout out to God that one
cannot live without Him anymore, and that upon that joy descends on you,
heavenly peace that goes above one's mind. This joy is what the Holy
Spirit gives us. We do not only get this from the hope we have in
Christ---that is, everything that happens to us, all of the negative
things, we now see in the light of God's plan. So if we break a leg or
lose our jobs or even a friend or family member, we know that God
allowed it for His good purpose, so that we may be shaped. Remember,
suffering produces endurance and endurance produces character. So in
this we can rejoice and know that our suffering was not for nothing and
actually it was for our own good, to be shaped, but more importantly, to
fulfill God's plan. But there is another reason to rejoice, and that is
because we are at a party now and the groom is in our midst. There is
not reason to be more rejoicing than otherwise, but it just is. Having
the groom in your circle simply is an honor. There is not much to say
about that.

Now when we have Jesus Christ through the Holy Spirit in our lives, we
are not alone, we have someone to talk to. We have an important person
by our side Whom we are proud of, in Whom we find our identity. It is
like a son who is proud to see his father at his performance or at his
sports game. We know Jesus and the Father see everything we do and they
are proud of us no matter the results, but are looking only at our
hearts.

\textbf{Death in Adam, Life in Christ} \emph{12 Therefore, just as sin
came into the world through one man, and death through sin, and so death
spread to all men because all sinned---}

One man only sinned, and with it, though unbeknownst, except through his
nakedness, he started dying. So there was a mental change (he knew he
was naked), an emotional change (he was afraid), a physical change (his
cells started dying) and a spiritual change (he no longer had contact
with God). I am not sure if God's Holy Spirit was ever in Adam, but it
certainly was not after his sin. Now quite logically every person to be
born from Adam from that time on would at least have physical failures,
because of Adam's imperfection in his DNA, which may be coupled with
this mental state, since the mind is based on the brains which are in
turn based on DNA. So that could be explained away with by a change in
DNA that God caused as a consequence for eating this fruit. The fact
that this newborn child had never seen or experienced the full version
of God, but only heard a part of it, made it such that the child would
not have a perfect understanding or relationship with God. In the case
of a perfect relationship with God, the person' emotional state should
be perfect. There would be love and where there is love, there is no
fear. There would be no jealousy or denial or anything that leads up to
the destruction of our lives. However, what would happen with a person
had he been created next to Adam and Eve, or a person born from them
before the Fall? Would this person be exposed to sin and would he fall
for it? Would death be spread to him as well? The situation would be
exceptional. Unlike an innocent child in class among sinful children,
where this innocent child would learn bad things and be hurt in the
process, Adam was a fully grown man and knew what was bad and he would
never be bullied or become sad. What I mean to say by this, is that some
people start into bad habits as a result of being excluded from a group,
and in order to belong to a group they start picking up these habits.
This would not be this third person's case. He (or she) would not
bullied by Adam and get hurt, because he would not know these feelings
in a perfect world. However, Adam might have been able to seduce this
person, just like the Satan was able to seduce Adam. So the answer is,
yes, if there is only person in a group of no sinners, he will cause
others to sin. God would have removed Adam and Eve from this third
person in order to make sure that this third person stayed holy. But
death would not spread as if it were an airborne contagious disease.

\emph{13 for sin indeed was in the world before the law was given, but
sin is not counted where there is no law.}

\sout{This is a clear statement of the difference between sin and
transgression or trespass. Note that it says that sin is not counted
towards our figurative book which all records of all transgressions are
kept.} However, the \emph{natural} consequences of these transgressions
are present. If one person hits another, even unknowingly or not on
purpose, he might get a slap back. God did not order (or conduct) this
other guy to slap him back. It is like hitting a ball against a wall, at
one time it will come back at you. That is just a law of natural
physics. So death is a natural consequence of our transgressions. I am
not sure how to explain this. If Adam's cells were multiplying without
any cell dying out before it was replaced by another or getting a
shorter lifespan (i.e.~the telomere \#todo/opzoeken/engels decreases),
why would a transgression lead to Adam's cells suddenly do dying out?
The answer lies in the fact that the above description is wrong. Verse
13 says ``\emph{sin} is not counted'', but I misread it as
``transgression is not counted''. So what does verse 13 then mean? It
means that sin was present because people knew in their hearts what was
wrong. The fact that it says it is not counted, does not mean it is not
counted towards your record of sins, because we already know from Romans
3 that God judges according to a person's knowledge of the Law, both the
written law as well as the law in one's heart. What this verse then does
mean, we can see in verse 14. It says there that this sin, though not
counted, still resulted in death.

\emph{14 Yet death reigned from Adam to Moses, even over those whose
sinning was not like the transgression of Adam, who was a type of the
one who was to come.} \#biblestudy/theology/typology

This part makes it clear that no matter what size the sin is, death will
always be the result, even if the sin was not a great one as that of
Adam. On one hand Adam's sin was the greatest of all, it is that sin
that defied God's direct commandment, which was clear than any other
commandment ever given to anyone in history---even clearer than the Ten
Commandments that have been given to Moses on mount Carmel
\#todo/opzoeken deze berg? given over a period of forty day, even though
the Ten Commandments and all the laws that follow after Exodus 20 would
not take forty days to read. Perhaps God had repeated his laws many
times over due to its importance until Moses had memorized them verse by
verse, or perhaps God had told him many other instructions. But even
then, the instruction were not as clear as those received by Adam,
because God was hidden in a cloud or in some other form---and Adam's
instructions could not have been any simpler: do not eat from the fruit
of this tree. We know from this book of Romans that God judges according
to one's understanding of the Law. In many of the law given to Moses
there would be an understanding of less than a hundred percent. Even
murder and theft could be in a grey area, when you look at what happens
during a war. But the law given to Adam was perfectly understandable.
There was no situation of hunger or other need to take the fruit. There
was only a desire to become like God. On the other hand, Adam's sin
seemed small, just the eating of one thing that was not allowed. We
know, however, that it is not always the act that God detests, but
rather the intention with which it is done. The Sabbath as well requires
rest, but when the intention is to save people, God forgoes the sin.
Either way, any sin, no matter how big or small, will lead to death.

\emph{15 But the free gift is not like the trespass. For if many died
through one man's trespass, much more have the grace of God and the free
gift by the grace of that one man Jesus Christ abounded for many.}

The free gift God has given us is not like the sin of Adam, because it
does not entail that once the free gift had been distributed
\emph{everyone} would fall under it, like during the first trespass of
Adam and Eve when the whole of humanity fell under sin. There are a few
points, however, that changed for all humanity after Christ rose up from
the cross. People now are able to have the Holy Spirit and therefore
have the certainty that they will go to heaven. They will thus have the
Holy Spirit as long as they live, whereas they would not have this in
the Old Testament. The Holy Spirit is a warranty \#todo/opzoeken/engels
for us of eternal life, but also a way for us to have direct contact
with God and obtain His wisdom, strength and the power to do miracles.
These points, however, are only applicable for mankind when they choose
to follow God. Before Christ's resurrection they were not available to
anyone. So the difference is that where Adam causes everyone to fall
under his sin and we had no choice in that, Jesus' sacrifice is
\emph{available} to everyone, but only when they choose for it. In other
words, sin has taken our freedom away from us and we have no choice but
to sin, but under Christ's law we \emph{have} the right to \emph{choose}
again.

\emph{16 And the free gift is not like the result of that one man's sin.
For the judgment following one trespass brought condemnation, but the
free gift following many trespasses brought justification.}

We now come to a few difficult verses that require complex reasoning in
order to understand them. Paul is explaining the difference between
judgment without and with the free gift. Though this verse is not
literally written in this way, it seems like in the former case, a
single trespass brings a whole lot of condemnation. Only one trespass is
enough to condemn you for the rest of your life here on Earth and in the
afterlife. On the other hand, the free gift is so powerful that even
after many trespasses it can give you the one thing you need,
justification. One could conclude from that the free gift of God's grace
is much more powerful than condemnation, because it covers multiple
condemnations. If each trespass let you sit ten years in prison, then
ten trespasses would condemn to prison for a hundred years. A single
justification, however, would \#todo/opzoeken/engels opliften all of the
incarceration \#todo/opzoeken/engels correct? .

\emph{17 For if, because of one man's trespass, death reigned through
that one man, much more will those who receive the abundance of grace
and the free gift of righteousness reign in life through the one man
Jesus Christ.}

Here too, we must look at the effect of one man's trespass. The effect
of one trespass by one man is death reigning not only \emph{over} that
one man, but \emph{through} that man. This death spreads itself through
his thoughts and actions, first and foremost to his family---if he has
one---and then to his friends. We see this in influential people such as
Charles Manson, Ted Bundy and David Koresh who stirred up people into
cults, spreading lies about himself and in the latter case about the
Bible as well, so that others do what they want. This effect spreads
itself unendingly. Everyone born in such an environment is almost doomed
to accustom to these perverse thoughts. So are we, in this world, where
even a ``normal'' situation is completely sinful in the eyes of God. Now
for every man who in and through whom death reigns, which is
\emph{every} man, Jesus must provide His grace in to justify the person
so that he will not live under the Law anymore. We know that it is much
harder to build up than to break down. This is universal law. Building
up an object with Lego® pieces is much more difficult than breaking it
down. Jesus' gift is sufficient for every man. His gift is worth more
than whatever death destroys in a person's life. Now does death or
Jesus' gift reign more? Of course it is Jesus' gift. And as many lives
death can destroy, so many lives Jesus can not only restore, but also
give meaning and true fulfillment.

\emph{18 Therefore, as one trespass led to condemnation for all men, so
one act of righteousness leads to justification and life for all men. 19
For as by the one man's disobedience the many were made sinners, so by
the one man's obedience the many will be made righteous. 20 Now the law
came in to increase the trespass, but where sin increased, grace
abounded all the more, 21 so that, as sin reigned in death, grace also
might reign through righteousness leading to eternal life through Jesus
Christ our Lord.}

Verse 20 explains that when God says that He forgives \emph{all} sin the
scope of His forgiveness is limited to the sin committed, for one cannot
forgive more than the sin that has been committed---although the current
amount of sin present in this world still is a great deal when one
realized how much it is. Sin existed \emph{before} the Law, as we have
read above, through sinning against our conscience, but after the Law
came, it became clear to everyone who knew the Law what sin was, and now
every trespass committed against the conscience \emph{as well as} the
Law were counted as sin. This means that God's forgiveness is ever
increasing along with every sin committed. But there is a
\emph{hina}-clause, translated in English by the words ``so that''. That
is, the next line answers the question why grace abounds all the more.
Let us first analyze that verse. Sin reigns \emph{in} death, but grace
reigns \emph{through} righteousness. What does \emph{in} mean? Does it
mean that in the world of the dead, it is sin that reigns? Almost. This
world on Earth contains both spiritually dead and living people. One
could say the more royal blood one has, the more right he has to ascend
the throne. The more money one has, the more power. As for the spiritual
realm: the more sin, the more reign---in the world of the spiritually
dead. When one has bad friends, he will reign if his badness exceeds
that of his peers. He would be deemed more courageous and that is what
it is about. God, however, wants grace to reign in the realm of the
spiritually living people. The second part of this verse is more
difficult to understand. Indeed, it is grace that reigns, but grace
cannot reign without righteousness. It is like (a president) pardoning a
person who has not been indicted or is not incarcerated. Grace comes
after righteousness, after God's righteousness concludes that we are
guilty. That is why grace reigns \emph{through} righteousness and this
freedom that grace gives us, leads to eternal
life---again---\emph{through} Jesus Christ, for eternal life as well
cannot be obtained without Jesus Christ.

\#biblestudy/devotionals/romans

\hypertarget{romans-6-esv}{%
\section{Romans 6 (ESV)}\label{romans-6-esv}}

\textbf{Dead to Sin, Alive to God} \emph{1 What shall we say then? Are
we to continue in sin that grace may abound? 2 By no means! How can we
who died to sin still live in it?}

The image I have when reading this verse is that of an organ that has
died to our body. Our bodies naturally resist anything that is external,
even if it is human, but from another person. That is why organ
transplants often \#todo/opzoeken how often? fail. I do not interpret
verse 1 in such a way that I see sin as an organ that died and now has
been removed. Rather yet, I see us as the healthy organ in the body of
sin, and we have died to it and we have removed from it. But just like
in real life a part of healthy tissue is cut away along with the dead
tissue, so a part of this sinful body is attached to us, the dead organ.

\emph{3 Do you not know that all of us who have been baptized into
Christ Jesus were baptized into his death?}

Now the image continues, but now it is like the worldview of the
Christians but the opposite of it. I mean by that, there is a worldview
where people worship the Satan as their god, and their heaven is hell.
Some people fall of the path and end up in Christianity, never to return
to their path again. In Christianity of course it is the other way
around, where we say that some people are on the wrong side or path, and
some of those do end up on the good side. Those on the good side,
however, always stay on the good side and never fall off the path to the
wrong side. Those who have ``fallen'' are those who were on the bad side
and were \emph{trying} to reach the good side, but failed to do so. It
is like a cliff. On the slope it is still possible to fall, but once you
are on the cliff, you are safe. Now these worlds co-exist on Earth
simultaneously, let's say they are bad-Earth and good-Earth. Once a
person has died to bad-Earth, he will enter a semi heaven-like state,
although that is much overrated for no matter how good things get on
Earth, they will not even approach 1\% of what heaven is, although it
still is much better than what bad-Earth is. Suppose we were in heaven,
would we then long for good-Earth? No, we would not, because we would be
fully clean then, without any memory to remind us of good-Earth. But now
we are in good-Earth, we are still not fully clean yet, and we still
have traces of bad-Earth which remind us of our old life. And just as
illogical it is for the prostitute who was loved by the prophet Hosea
still ran away from him and returned to the business she did not want to
be in, in the same way it is illogical for us to return to our old realm
where there is no reward except death, and even in life itself there is
no reward except for pain---all the while while our heavenly reward is
right in front of us.

\emph{4 We were buried therefore with him by baptism into death, in
order that, just as Christ was raised from the dead by the glory of the
Father, we too might walk in newness of life.}

As referred to in Romans 5 we are now part of the spiritual realm of the
living. This is where sin is condemned, and where grace is reigning. Why
would we sin? It is utterly useless. It is counterproductive. Does it
make one feel better?---not in this realm. It makes one realize how
sinful he is. It does not bring any rewards. It makes one regress to the
place where he came from. At least in the realm of the dead to sin was
cool. But sinning in the realm of the living makes you an outsider.
There is no logic to sin in this realm except the love for oneself, and
it our \emph{self} that we need to die to. This is what Paul means when
he writes in verse 3 that we have been baptized in Jesus' death. As soon
as we die, we will have a new life---one in which we have completely
died to our sinful self. But right now we are not dead yet, we have only
been baptized in Jesus' death. We have chosen our path and this is on
Jesus' death. And just as time goes forward without a chance to go back,
so the only way we can move forward is on this path that Jesus paved for
us, in the realm of the living. Now it sounds like we were forced to
walk onto this path, but it is actually the other way around. We were
begging for this path, because the path of the dead, which leads to
hell, was the path we were walking on ever since we were born. That did
not satisfy us and in our pain we cried out to God to save us from all
the pain that is here on Earth.

\emph{5 For if we have been united with him in a death like his, we
shall certainly be united with him in a resurrection like his. 6 We know
that our old self was crucified with him in order that the body of sin
might be brought to nothing, so that we would no longer be enslaved to
sin. 7 For one who has died has been set free from sin.}

It is true that we have been set free from sin. We now have the
possibility to break with our addictions and bad habits. It requires
discipline and power but most of all a choice. We have this choice,
because God requires freedom, even after having made the choice to
follow God. We now need to experience what this choice entails and show
(not for Him but to ourselves) how we deal with the consequences of
being a Christian.

\emph{8 Now if we have died with Christ, we believe that we will also
live with him.}

Normally, if you die with someone, it does not mean you will live with
that person as well. Suppose you are in the army and you decide to die
together in battle with your brothers or in capital punishment by the
enemy. If any of you survive, however, it does not mean that the others
will as well. What this verse refers to is sharing of experiences, pain
and effort. Usually when several people start working on a project
together, those who succeed, will then also share their success with
their co-founders. See for example the stories of self-made billionaires
Bill Gates (Microsoft), Mark Zuckerberg (Facebook) and Jack Ma (from
Alibaba). Of course, in practice this principle does not always work out
and some people may be left out of the profits, which was shown in the
motion picture \emph{The Social Network} although it must be mentioned
that there is (a lack of) scrutiny concerning the authenticity of that
script. \#tags/movies But in Jesus' case, He rewards us. We die with Him
and \emph{when} He raises, for He surely will, He will bring us from the
dead as well and He will let us reign with Him.

\emph{9 We know that Christ, being raised from the dead, will never die
again; death no longer has dominion over him. 10 For the death he died
he died to sin, once for all, but the life he lives he lives to God.}

Jesus' resurrection is special in that sense that His death was only
once. Of course, this does not apply to others. Lazarus and others who
have been raised from the dead died again, because they rose with their
regular bodies. Jesus on the other hand was never man until His
incarnation. Now if His human body died, would Jesus then be died? Of
course not. His soul would ever be alive as it always had been. That in
itself, however, could not be called a resurrection, for a resurrection
requires someone to die first. Also, it was not a resurrection by a
third person, for then Jesus would have been healed just like Lazarus
had been, in a regular body. No, His resurrection was of a different
order. Jesus resurrected by His own power, limited to a purely human
nature. Did He fight the devil? He did not need to, the devil was
speechless before the throne of the Judge, God the Father, and could not
even use a single fact of what happened in Jesus' life to accuse him.
The Satan knew that Jesus was innocent and so did the Father. That would
have been the fastest court hearing in ``history'' (although there is no
concept of time in heaven) and directly after His death, Jesus would as
much as skip court, go to heaven, preach to the generation of Noah
(cf.~\#todo/opzoeken ) \#todo nog af te maken

All of this would have taken less than a second, because no time exists
outside of our realm, yet Jesus took three days before returning to
Earth. Why? \#todo/opzoeken But now, Jesus Christ would have a glorified
body---albeit with scars on at least His hands and perhaps His side as
well---and would never die again. Now how do we know He will never die
again? Except for the fact that He now has an imperishable glorified
body, it is the word ``dominion'' that explains it here. Death might
still exist, but it has no \emph{dominion} over Him. Why? The answer is
written in verse 10. Jesus died to sin. The word ``to'' can be read as
``because of'' or ``due to'', it should be interpreted as a indirect
object (dative) and ``sin'' is a noun here, not a verb, just like ``I
propose to you.'' Jesus could not die a second time to sin, because the
first time He died, He already died to all sin there ever was. Now if
anyone were to sin again, one would say that Jesus had not died yet for
that single sin, but that is not the case here. Jesus died for all
sins---of the past and the future. So now, He will never die anymore,
and the life He lives now, is to God. Whenever sin is still a factor, we
can never be sure what intention one has. When one gives money to
charity, is it to God that he gives, or partially to his own ego, to
gain compliments or stature, in other words, to sin? Jesus has died to
sin, all that was in Him has already died to sin, like every part that
is not gold, will melt in the furnace, and that which is not gold,
cannot burn. Now Jesus was pure and therefore---in our analogy---not
flammable, but because He took upon Him our sin, which is highly
inflammable, He burned and suffered, and not just a little like when you
would put a napkin on your hand and burn the napkin, but it is like He
were packed inside mountains and mountains of paper and He had to suffer
until all of that was burned. Is that the reason why it took three days
for Him until everything was burned? No.~The suffering is not expressed
in time nor is time a factor in suffering. Suffering in hell takes place
outside the realm of time, space and matter, but for Jesus, His
suffering was on the cross where human suffering was present during the
hours Jesus was on the cross and superhuman suffering occurred at the
very last moment when He spoke out His very last words: ``\emph{Eloi,
Eloi, lama sabachtani?}'' (Father, Father, why hast Thou forsaken Me?)
accompanied with a cry---that was the moment when all of history's,
present and future sin would all accumulate and put on Jesus Who then
paid the price for it.\\
Back to where we were. We now know that Jesus can never die again,
because Jesus paid for all sin. Could it then be that a sinner could
tell the devil that he does not have to pay for his sins, because Jesus
already paid for it? If this sinner were to be punished, the double
amount of punishment would have been distributed, once for the sinner
and once for Jesus, which is not just. Would the devil accept this
reasoning? Of course not! He is a liar and a proponent if not the
founder of injustice. It would be up to the court justice, which is God
the Father, to decide. He would listen, but He already has His answer
ready: ``Did you accept the mercy I gave You?''

\emph{11 So you also must consider yourselves dead to sin and alive to
God in Christ Jesus.}

Now the \emph{hina}-clause \#todo/opzoeken/greek used here follows from
the fact Jesus now lives His life to God. Why should we then do what
verse 11 dictates? Should it not say ``may'' instead ``must''?
Logically, one would interpret the context as follows. Jesus has paid
the price for all sin and died for that, opening by that action a
``portal''---if you will''---for us to enter the Kingdom of Heaven---not
yet Heaven itself, for that is where we will go to after our life, but
here on this world, on Earth, access to the \emph{Kingdom} of Heaven
(see notes on \ldots. \#todo/opzoeken eerder uitgelegd?), as explained
in verse 3 through the ``baptism into His death.'' Now if we are pass
through that portal into the realm of the living, where sin neither has
the effect of binding us or giving us pleasure, then there all will
strive to live for the good---as one can empirically see in the fact
that \ldots{} see Keep notes. \#todo nog af te maken

\emph{12 Let not sin therefore reign in your mortal body, to make you
obey its passions. 13 Do not present your members to sin as instruments
for unrighteousness, but present yourselves to God as those who have
been brought from death to life, and your members to God as instruments
for righteousness.}

Even though we---or our souls---are in this realm of the living, our
bodies are not and they are still mortal and fleshly. That is why we
still feel worldly feelings, such as hunger and sleep, but also more
than just the things we need, such as power and lust. Now our bodies are
pulling us towards the realm of the dead, but we cannot discard our
bodies yet. We should not ignore it either, for it is the way for us to
communicate with people from the realm of the dead, for our souls cannot
enter that---and that is a good thing! Our bodies are not only to
communicate, but also to exemplify. We show a good example of how the
Christian life is, along with all of its suffering, by showing how good
our God is. The members of our bodies, our arms and our mouth for
example, should therefore also be good examples of the Christian life.
They should be used for the good, to help people, not to fight or to
scold. Our bodies are still connected to our soul and can even make it
dirty by making it one with a prostitute for example, \#todo/opzoeken
waar? 1 Corinthians ? But Christians should have the power from Jesus to
refrein from these things (cf.~1 Corinthians 10:13). This is very
debatable however, as I see more and more leaders having fallen into
sexual sin, such as \ldots{} West (almost pastor of a megachurch) who as
a youth pastor had sex with two of his students, and John Crist, but
both of them repented of their sin. The latter did not actually had sex
and the former was still young and perhaps not a believer yet back then
(and maybe he still is not). So would any Christian, real Christians
that is, ever go as far to have unlawful sex with another person, even
when they have the power from Jesus Christ to control their bodily
urges? The answer is ``no''. Even Joseph Prince said in an article with
Dr.~Brown that such as person is not under grace, or in other words, not
saved.

\emph{14 For sin will have no dominion over you, since you are not under
law but under grace.}

This latter is what I have noticed in practice. Sin does not have a
dominion over me. In the past addictions and bad (social) habits were
something I was controlled by and could not change. Aggression and
hot-temperedness for example, were two of those habits. But now, though
I still sin in these fields, I feel that when I fall back into these
sins or habits, it is because I choose to do it or to allow it. God has
already freed me from the addiction, and put me in a safe position, but
I allow myself to be exposed to this sin.

\textbf{Slaves to Righteousness} \emph{15 What then? Are we to sin
because we are not under law but under grace? By no means! 16 Do you not
know that if you present yourselves to anyone as obedient slaves, you
are slaves of the one whom you obey, either of sin, which leads to
death, or of obedience, which leads to righteousness?}

Paul is using rhetoric here and posing the obvious question. Of course
none should sin on purpose when we are not under the law. This would not
satisfy us, as the previous verses have already concluded. However, Paul
uses another argument. When we sin, we are obedient to sin. The devil
did not force us to sin, but we willingly did it. When we do it in an
obedient way, we acknowledge that sin is our master, and having sin as
your master will lead to death. Christians, on the other hand, would not
willingly sin, but reluctantly, for we know what is good, but we do not
do it. This shows that we are slaves of obedience to God, for when there
is something good we need to do, we willingly do it. The wicked,
however, have trouble doing good things, as we can see in so many
depictions of antagonists, such as Ed Harris in \emph{National Treasure
2}, who did a good thing at the end, but really had to force himself to
do so. \#tags/movies

\emph{17 But thanks be to God, that you who were once slaves of sin have
become obedient from the heart to the standard of teaching to which you
were committed, 18 and, having been set free from sin, have become
slaves of righteousness. 19 I am speaking in human terms, because of
your natural limitations. For just as you once presented your members as
slaves to impurity and to lawlessness leading to more lawlessness, so
now present your members as slaves to righteousness leading to
sanctification.}

In verse 19 Paul describes how parts of our body were not only exposed
to sin, but also leading us to more lawlessness. A hand that is used in
violence to beat someone to steal money for example, is also feeding the
soul with power and greed for money. But now that our souls have been
cleansed and forgiven, we can use our hands for the good and thereby
feed our souls with good things. Helping people for example would teach
us endurance, perseverance and character, eventually leading us to ever
more sanctification.

\emph{20 For when you were slaves of sin, you were free in regard to
righteousness.}

I am not sure if I understand verse 20 completely, but it says that we
were free, but only in regard to righteousness. I think it means that we
were free within a certain scope, just like one is free in his
conscience to do all he wants, whereas after we have been set free from
sin, we are not. But what does it mean then that we were free in regard
to righteousness? Is it not that righteousness will always prevail, for
the those under the law and those under grace (albeit that in the latter
case justice will take place on Jesus of those on Christians)? I suppose
that is not the meaning. The meaning rather is that the only thing they
were free of was righteousness in that sense that they would not receive
it. Sinners do not get the righteousness they deserve. Why not? The crux
is in the definition of the word ``righteousness''. Basically it means
that the righteous people get their rewards for being righteous or good,
and the unrighteous will receive punishment for their deed, but here and
elsewhere in Romans ``righteousness'' refers to the part that is due to
the righteous people. So with that in mind---if it is true---it means
that the slaves of sin are free from rewards---which is not a good thing
at all. It is like telling an employee he can do all he wants, he is
free---but he is also free from his salary.

\emph{21 But what fruit were you getting at that time from the things of
which you are now ashamed? For the end of those things is death. 22 But
now that you have been set free from sin and have become slaves of God,
the fruit you get leads to sanctification and its end, eternal life.
::23 For the wages of sin is death, but the free gift of God is eternal
life in Christ Jesus our Lord.}:: \#biblestudy/memorization

The answer to this question is obvious: we were getting nothing from
that for which we are now ashamed. All my pride in the number of girls
has led me to a sex addiction for which I am ashamed now---I think that
is the main thing I am ashamed of. I am ashamed of the money I spent as
well. Instead of working hard at a job and saving money, and working
hard at my studies to quickly go through it, I slacked---not only
because of a girlfriend I had---but also for failing to see the
importance of money, the stewardship of it, that loaning puts me more
under slavery of it, and the indulgence and sloth it led to in not
cooking my meals myself, wasting time on watching \emph{manga} and a
plethora of other things. I was not free back then, but I am now. I
still have a few of those things mentioned above, and perhaps I have
even gained a few new things, but the difference is that I am ashamed of
it now. I know it is bad, but I cannot control myself, because I am weak
in the flesh. My spirit wants to do the good thing, but I too often
surrender to my flesh. So now, I sanctify my body and spirit by doing
the things I am not ashamed for, which are the things that are accepted
in Gods kingdom. All these ``wages'' from sin---which are in fact
debts---have been paid by Jesus Christ, but the Earthly wages I have
left are huge financial debts, which I need to pay off every month for
the coming twelve years---but is \emph{nothing} compared to what God has
saved me from.

\#biblestudy/devotionals/romans

\hypertarget{romans-7-esv}{%
\section{Romans 7 (ESV)}\label{romans-7-esv}}

\textbf{\emph{Released from the Law}} \emph{1 Or do you not know,
brothers---for I am speaking to those who know the law---that the law is
binding on a person only as long as he lives? 2 For a married woman is
bound by law to her husband while he lives, but if her husband dies she
is released from the law of marriage. 3 ::Accordingly, she will be
called an adulteress if she lives with another man while her husband is
alive.::} \#biblestudy/apologetics/divorce \emph{But if her husband
dies, she is free from that law, and if she marries another man she is
not an adulteress.} \emph{4 Likewise, my brothers, you also have died to
the law through the body of Christ, so that you may belong to another,
to him who has been raised from the dead, in order that we may bear
fruit for God. 5 For while we were living in the flesh, our sinful
passions, aroused by the law, were at work in our members to bear fruit
for death. 6 But now we are released from the law, having died to that
which held us captive, so that we serve in the new way of the Spirit and
not in the old way of the written code.}

This is a beautiful example of how we pass from one world into another,
where the second world has different rules and a different purpose.
Everything we lived for in the realm of the dead was meant for death.
None of the things we did, not even the good things, would be used for
life after death, because eternal life is only for those living in the
realm of the living. The dead will take their accomplishments with them
to hell. Also, all fruit born in the realm of the dead is for death,
that means even the children born. As pitiful as it may sound, these
children are raised up for death by parents who do not know any better.
But fortunately all of them still get the chance to repent and convert.

Since we have now died in the realm of the dead, does the Law then still
apply to us? Yes and no. The Law of God is something absolute, it is
good, whether in the realm of the living or dead, or in heaven or in
hell. What Paul says is just that it does not \emph{apply} anymore to
the dead. ``Apply'' in what sense? Does this mean we can do everything
we want? Also here the answer is yes and no. For those who have
physically died there is no marriage, so adultery does not apply to
them, in the same that rules concerning marriage only apply to human
beings, not to spirits or animals. Now we still have a human body, but
our spirits have been renewed. So to answer the first question, the law
does to us because we still have a body, however, it does not apply to
us in the sense that we do not get the punishment anymore, but Jesus
Christ does. As for the second question, we \emph{can} do everything we
want, but Christians do not want Jesus Christ to be punished and so we
do not want to do everything outside of the law. Note that we do not
only want to obey because we \emph{fear} of Jesus Christ's punishment,
but it is \emph{actually} something we do not want. It is like hurting
your own body, as we are part of Jesus Christ's body. Nobody wants to
hurt his own body. It is not like we actually do want to hurt our own
body, but we are afraid of pain. In the same way, we see Jesus Christ as
part of our body, as the head, and we want to obey him. As a further
answer to this question, as mentioned before, we \emph{want} to obey God
and do good, we \emph{want} to stop sinning, but we cannot, because our
flesh is weak.

\textbf{The Law and Sin} \emph{7 What then shall we say? That the law is
sin? By no means! Yet if it had not been for the law, I would not have
known sin. For I would not have known what it is to covet if the law had
not said, ``You shall not covet.'' 8 But sin, seizing an opportunity
through the commandment, produced in me all kinds of covetousness. For
apart from the law, sin lies dead. 9 I was once alive apart from the
law, but when the commandment came, sin came alive and I died. 10 The
very commandment that promised life proved to be death to me. 11 For
sin, seizing an opportunity through the commandment, deceived me and
through it killed me. 12 So the law is holy, and the commandment is holy
and righteous and good.}

``Sin itself seized the opportunity through the command.'' Sin itself is
not a living thing, able to think for itself. This means that as soon as
God created something good and holy, such as the law, sin came into
existing, defined by everything that is outside of the law. Now that is
about sin itself, but how about sin in us? It is true that humans long
to do which they may not do, just like the curiosity of children is
aroused when you tell them not to look in a certain cupboard. Yet how
would this \emph{produce} all kinds of covetousness? That could only be
the case when sin already was present in Paul. Coveting was something
that Paul already was guilty of, but before he knew of this law he did
not \emph{know} it was wrong. He knew what it was, that is the desiring
of something that does not belong to you, but what it really meant, what
impact it had, that it means you will sin towards God, that he did not
know. Was sin always existing then in us even before the law? I thought
it was. I thought that transgression was present, but that it was
counted as sin. However, of course man's lust for sin has always been
present and sin itself---not through the written law of Moses, but
through the law of the heart---was present as well. How can Paul then
say that apart from the law, sin lies dead? How can he say that he was
once alive? First of all, Paul never lived before the law, so he is not
talking about the Law of Moses coming into existence. Paul could be
talking talking about the moment he learned about the Law of Moses or
about the law written on our hearts. He would have learned about the
written Law most likely on his fifth, and realize what sin actually
meant, perhaps on his seventh. According to a psychologist children are
not capable of purposely lying before the age of around seven. Before
that it is something they know about but do not realize what they are
doing. So in that case, Paul was literally alive before the Law and just
like children, who are to die at a young age, will go to heaven, before
they realize they have sinned. Is that, however, really what he means
here? \#todo nog af te maken \#todo/opzoeken in \emph{MacSBNnkjv} Paul
could have meant that he thought he was alive through the Law, because
he hold it. However, once he realized that he did not hold by it at all
to Jesus' standards, he realized he doomed for death. That, however, is
not what I think this part means.

\emph{13 Did that which is good, then, bring death to me? By no means!
It was sin, producing death in me through what is good, in order that
sin might be shown to be sin, and through the commandment might become
sinful beyond measure. 14 For we know that the law is spiritual, but I
am of the flesh, sold under sin. 15 ::For I do not understand my own
actions. For I do not do what I want, but I do the very thing I hate. 16
Now if I do what I do not want, I agree with the law, that it is good.
17 So now it is no longer I who do it, but sin that dwells within me. 18
For I know that nothing good dwells in me, that is, in my flesh. For I
have the desire to do what is right, but not the ability to carry it
out. 19 For I do not do the good I want, but the evil I do not want is
what I keep on doing. 20 Now if I do what I do not want, it is no longer
I who do it, but sin that dwells within me.}:: \#biblestudy/memorization

God had to take this measure of creating the commandment so that sin
might show itself. Paul phrases this as ``in order that sin might be
shown to be sin''. Here the first ``sin'' refers to sin itself which is
not known to people to be sin. Therefore many people engage in it, for
they lust after it, without knowing that it is wrong. It is like the
period in China when opium was used when the Chinese did not know it was
bad for one's health. People kept on using it because it felt so good.
The law, however, made clear that opium was a sin (or in other words, so
that opium might be shown to be sin) and made an end to the use of it,
but arresting people for the possession and use of it. This does not
prevent us from \emph{liking} opium or in this case, sin. We still live
in our bodies made of flesh, and sin is on the same ``level'' as our
bodies. Our bodies long for this sin. The law, which is spiritual, is
put above the flesh. One needs to rise from the flesh and all of its
dependencies and longings, in order to enter the spiritual. It is like
animals who only like food, but who will never be interested in a god.
Even the anthropomorphic animals of Disney and Pixar never speak of
faith in a higher being but only about ``relatively'' shallow things
such as happiness, and sometimes loyalty and love as well, which are
less shallow, but still not on the level of spirituality. In the same
way, it is like talking to people about God and seeing a nonchalance and
superficiality about them of never even thinking about God, about the
possibility He might exist and what it would mean to them. They are not
very different from (anthropomorphic) animals in that sense. In fact, I
would state it is impossible for us to rise above ourselves and enter
the spiritual realm, for we are still living in this body and even our
spirit resides within this body. Our spirit, however, can be set free
and our bodies as well after death, after which we will be completely
separated from sin. For now, while our spirits are on the brink of the
two realms, we must deal with the fact that what our spirit does not
want, our body wants, and our mind and spirit should control the body.

\emph{21 So I find it to be a law that when I want to do right, evil
lies close at hand. 22 For I delight in the law of God, in my inner
being, 23 but I see in my members another law waging war against the law
of my mind and making me captive to the law of sin that dwells in my
members.}

These verses exactly state what I had in mind. This only confirms my
interpretation of this chapter. So what this part says is that our body
is like two people, each in a different realm and with different laws.
Our heavenly person wants to obey the law of God and make his body holy
as well, but he is pulled back by the earthly person who wants to who
has the same mission to bring back this heavenly person to Earth and
make him abide by the law of the flesh. Of course, in a formal debate
there is no chance that the earthly person could ever win--by all
definitions of what is good, heaven would win by a landslide--but in
practice however, the earthly person wins sometimes, partly because the
heavenly man had been an earthly man before and enjoyed all that is on
Earth. Biologically speaking one could say that he had an amount of
chemicals in his brains, the same kind of stuff that one gets when using
drugs, dopamine. That is the reason why so many drug addicts fall back
into their old addiction, because they want to feel this amount of
dopamine in their brains again, but nothing here on Earth gives them
that, except for a drug stronger than their previous one or perhaps
numbing drugs, so they at least do not feel the lack of dopamine. Real
happiness, however, does not come from dopamine only. It comes from love
and affection, respect and mutual interests. Love for people and to be
loved in return is the best feeling there is and it all originates from
God Who started loving us and so spread the fire.\\
If dopamine were the only thing that gives us happiness it would be like
saying that the orgasm is the only part of sex that is interesting, but
we all know that is not the case, for otherwise masturbation would be
all that people desire. However, just like masturbation, people often
and rather choose the easy way to be satisfied quickly instead of
choosing the more difficult and longer way that leads to a higher
result, because they either do not know what the higher result is---that
is, a feeling much more euphoric than dopamine could ever give---or they
have forgotten this feeling. This is exemplified in contemporary
(sexual) relationships. People either do not get married, because they
think it cannot get any better than the relationship they already have,
where they have a partner yet have freedom to leave anytime they want
(provided they do not have any children), or they are married and cheat
on their spouses---thinking that they can go back to their previous
state of being single and have sex with anyone---but now it feels
different---and it feels different for several reasons. The first reason
is that they have changed. They are not the young, single person they
were before their marriage. Their state has changed, better yet, even
\emph{officially} their status has changed from \emph{single} to
\emph{married}. And just like when one shifts from the earthly (deadly)
realm to the spiritual realm, along with one's taste and preferences, so
this change to being married influence's one's preferences as well. This
is not known to the person who cheats, however, as he only finds out
\emph{after} he has cheated on his spouse. He (or she) did not realize
that his body has now become one with another person and that whatever
he does to damage it, will also damage his ``other half'', that is his
spouse. The second reason is analogous to the biological and
psychological explanation. This person is looking for the same amount of
excitement with which he had sex as a single. Now, as a married person,
he probably has less sex, less intense and long, and less loudly because
of his children sleeping in the room next door. I am not talking about
most marriages or about my own experience, but I am basing this on the
image that is portrayed in movies, which has a core of truth in it. That
is why this person thinks that if only he could have sex like he used
to, he is going to have fun again. The result is, that is he is not
going to have the same feeling, and he will then think that is because
of the age of the woman he cheats with, for when he was young, his
girlfriends were young as well. And so his next step is to find younger
women to sleep with, but eventually that does not satisfy either. A
whole number of other things then pop up in his head as a possible cause
of this euphoria-anemia and---if he has the money for it---he will try
out everything he had when he was young, a nice car, cool stuff, losing
weight---in short, that is the mid-life crisis. This person, or man in
this case, does not realize that his happiness is a sum of multiple
things, but made up of love. Someone once wrote that being loved is the
highest form of happiness \#todo/opzoeken , and the man does not realize
that, and he will not realize it if he does not love his wife, for the
less he loves his wife, the less his wife loves him back.

\emph{24 Wretched man that I am! Who will deliver me from this body of
death? 25 Thanks be to God through Jesus Christ our Lord! So then, I
myself serve the law of God with my mind, but with my flesh I serve the
law of sin.}

How wretched this is for the man who feels this, but at the same time he
is blessed for feeling it. In fact everybody should feel this, but not
everybody reaches this realization of what Paul describes and which I
have tried to explain above. Indeed, it is a battle between the mind and
the flesh and when the flesh has won, it hardly remains a battle. That
is why many do not see this battle, because they have already been
defeated. The war, however, is not over yet, as this message continues
in Chapter 8 and should be read continuously with Chapter 7. There it
says that even though we sin, it will not be accounted to us through
Jesus' sacrifice.

\#biblestudy/devotionals/romans

\hypertarget{romans-8-esv}{%
\section{Romans 8 (ESV)}\label{romans-8-esv}}

\textbf{\emph{Life in the Spirit}} ::\emph{1 There is therefore now no
condemnation for those who are in Christ Jesus. 2 For the law of the
Spirit of life has set you free in Christ Jesus from the law of sin and
death.}:: \#biblestudy/memorization

These first two verses are in fact an answer to the conclusion in
chapter 7 and earlier. The conclusion was that we are sinful and that we
cannot do anything about that as long as we still have our fleshly body
which is connected to sin and keeps dragging the soul down. However,
this counter statement is that Jesus Christ frees us. It is that simple.
Note how the arguments to show that we are sinful comprise multiple
chapters, while the argument to show that we are saved is contained
within a mere two verses (and a few extra verses for explanation). It is
as if we need more explanation about our sinful state and each
counterargument of ours to show that we are not sinful, must be
attributed by Paul, so that no one has any excuse to say that he is not
sinful or that he did not know he was. Now this law of the Spirit only
applies to spirits as well, that is our spirits. Our bodies are still
under the law of sin, and the spirit is under the law of the spirit.
Only now that our spirit is under this law, and this did not used to be
so, for it used to be under the law of sin and death, we can say we are
free. Then does what happens with the body happen solely to the body?
No, it is not. For if our body sins, alas, that is a pity, but our
spirit stays pure, because we did not want it sin, but the flesh was
stronger. However, if we \emph{allow} it to sin---and \emph{we}
indicates our spirits here---then our spirits would be the ones sinning
along with our bodies. It is like parents trying to control their
children. Spiritually speaking, the parent is not responsible to God for
their children's sin except when he allows the children to sin. Now the
question arises, when the parent has done so, would that mean that this
can never be forgiven anymore? The body dies along with all its sin, but
what happens with the spirit? There are some who say that this
determines what reward you get in heaven and whether you will be a king
of one or of many cities. However, I find that not intuitive nor
logical. I believe in different kinds of rewards, but if the minimum
reward for \emph{every soul} is always one city or at least being an
inhabitant in a city in heaven, then it would not matter for us to sin
all we want, for life in heaven is infinitely better than on Earth, no
matter our position there, be it a cleaner of toilets. No, as mentioned
in Chapter 7 the term ``sin all we want'' does not apply as soon as we
have become Christians, because the amount of sin that we want is zero.
No Christian \emph{wants} to sin. So to counter the question ``when the
parent has done\ldots{}'' another question must be asked: ``Is it
\emph{possible} for a Christian to sin with his spirit?'' The answer is
\emph{no, it is not}. We have to remember that what Paul said in Romans
7:15-20, both at the beginning and at the conclusive end of this part is
still true: \emph{it is no longer I who do it, but sin that dwells
within me}. We, i.e.~our spirits, cannot sin anymore, only our bodies
can, and we must do all we can to stop our bodies from sinning because
that is not what we want.

\emph{3 For God has done what the law, weakened by the flesh, could not
do. By sending his own Son in the likeness of sinful flesh and for sin,
he condemned sin in the flesh, 4 in order that the
\textasciitilde righteous requirement of the law\textasciitilde{} might
be fulfilled in us, who walk not according to the flesh but according to
the Spirit.}

This is a very deep text, which can only be understood in the light of
all that has been said before in at least Chapters 7 and 8. The Law,
though it is good, is not perfectly powerful. It could not prevent
people from sinning, neither by fear nor clarity nor reward nor the
pointing to God. In that sense it was \emph{weakened} by (human) flesh,
for humans have their own will. Unlike in computers where a law is
always obeyed, unless a particle of dust intrudes the process by landing
on some part of the motherboard so that the electric current is
disrupted, but aside from that, it is always obeyed as well as the laws
of physics, humans disobey laws, by \emph{nature}, imputed in us by God.
This nature is called \emph{free will} and it will---by
definition---never be as perfect as God. There was only one way to let
God's plan succeed and that is by putting Jesus in this flesh. Jesus is
the only Person Who can control sin and command it to stay in the flesh
only and stay away from the spirit. Before Jesus, sin was something that
protruded both flesh and soul, but Jesus' Himself, or as the Bible says,
His Word, cuts through the flesh and soul, or as Hebrews 4:12 says:
\emph{the division of soul and spirit} which (most likely) means the
same thing here. In other words, it is Jesus Who split the spirit from
the flesh and lets sin abide in flesh only. No one and nothing else has
ever been able to achieve that. Now this (splitting) is the only way to
let the requirement of the Law \emph{of Moses} be fulfilled in us. Note
that this law is still good and righteous according to verse 4. By
splitting our being into two, a body and a spirit, Jesus could put sin
in the one, while the other could be released.

\emph{5 For those who live according to the flesh set their minds on the
things of the flesh, but those who live according to the Spirit set
their minds on the things of the Spirit. 6 For to set the mind on the
flesh is death, but to set the mind on the Spirit is life and peace. 7
For the mind that is set on the flesh \textasciitilde is hostile to
God,\textasciitilde{} for it does not submit to God's law; indeed, it
cannot. 8 Those who are in the flesh \textsubscript{cannot} please God.}

The mind that is set on the flesh is \emph{hostile} to God. Paul notes
what Jesus has said before and that is that no one can stay neutral. One
is either for Him or against Him. Even a nonbeliever who is ``looking''
for God is still hostile to God, for he still has his own arguments why
he should not believe in God and those arguments are dangerous weapons.
Even one who does not have any arguments for or against God, but simply
ignores God out of disinterest, has the dangerous characteristic of
nonchalance that could cost him his own life, but also that of others if
he were to influence others by it. Hostility is not in the eye of the
beholder, but it is a mere fact---are you aiding the advancement of the
Gospel or are you not? If you are not, you are only standing in the way.
If you are not susceptible for the Gospel, you are wasting the resources
of the kingdom by letting Christians speak to you while they could have
spoken to those who \emph{are} susceptible. One may think that it is the
Christian's choice for speaking to him, but that is not the issue here.
The question here is what is hostile to the kingdom of God, and that is
everyone who and everything that stands in the way.

\emph{9 You, however, are not in the flesh but in the Spirit, if in fact
the Spirit of God dwells in you. Anyone who does not have the Spirit of
Christ does not belong to him. 10 But if Christ is in you, although the
body is dead because of sin, the Spirit is life because of
righteousness. 11 If the Spirit of him who raised Jesus from the dead
dwells in you, he who raised Christ Jesus from the dead will also give
life to your mortal bodies through his Spirit who dwells in you.}

This implies that our mortal bodies do not even have life yet. Surely,
they can breathe, but that is not the definition of life according to
God. As a matter of fact we were still dead before Christ came to awaken
up, let us hear the Gospel and incited us to convert and repent. The
Holy Spirit, which is the spirit Paul refers to in these verses, is
powerful and can raise not only dead bodies to life---which is
easy---but it can bring life to our spirits as well---which is much
harder. The Holy Spirit can convict one of his own sins, He can let
those who are deaf hear, those with hardened hearts become soft and make
those who are fearful courageous.

\textbf{\emph{Heirs with Christ}} \emph{12 So then, brothers, we are
debtors, not to the flesh, to live according to the flesh.}

We are in great debt indeed to our God Who saved us from all this
wretchedness of what is us. We ought to live a good life with the bodies
we have received, through the our spirit which has now been renewed and
has been separated from our bodies, being pure and holy. We will never
be able to repay this debt. Still, God asks us merely to live as
examples to those who have not been saved yet.

\emph{13 For if you live according to the flesh you will die, but if by
the Spirit you put to death the deeds of the body, you will live.}

There are several conditions and consequences for us to obtain life in
the spirit, although these do not diminish the grace of Jesus Christ.
One of these things is that what do in the body must be put to death. We
cannot live by it, for everything our flesh gives, will only kill us
further. Does this mean then we can never take a holiday anymore? No,
because we still live in our bodies, our bodies need rest, so they are
allowed to rest, and God has provided us with a vast wealth of cultures
abroad which we can enjoy as well, all to the glory of His Name.
However, if we are going on a luxury holiday trip with no added benefits
for the Kingdom, then I doubt whether that is needed for or even allowed
in God's kingdom. \#todo/opzoeken

\emph{14 For all who are led by the Spirit of God are sons of God. 15
For you did not receive the spirit of slavery to fall back into fear,
but you have received the Spirit of adoption as sons, by whom we cry,
``Abba! Father!''}

The term \emph{sons of God} is used for kings or people of high stature,
for angels and for godly men. In this context it is logical that it is
used for the latter. Note how there are two spirits, one that causes
fear and one that causes us to cry out to our heavenly Father. In both
cases we are helpless, but only in one case we do not need to fear. In
the latter case we are free to sin, but we do not want to, in the former
case we are not free to stay away from sin and so we do it. These are
opposites, but no matter strange how this is, these two opposites are
incorporated into one body: the human who is also a Christian.

\emph{16 The Spirit himself bears witness with our spirit that we are
children of God, 17 and if children, then heirs---heirs of God and
fellow heirs with Christ, ::provided we suffer with him in order that we
may also be glorified with him.::}

To be heirs means we need to go through the same as Christ has. In many
cultures and stories a son cannot be the heir of his father's legacy or
throne unless he accomplishes a set of difficult feats by which he
proves he is worthy of the throne. Now we are not worthy of God's
throne, we never shall be, only Jesus is and He has proven this and
suffered for it. If we want to rule \emph{with} Him, but not instead of
Him, it is only logical that we suffer at least a little like He has.
Our test is here on this Earth, although I would hardly call it a test.
The test is---as Paul will explain in another book in the New
Testament---the life we are living now. If we can maintain the good life
we are to live, according to the Law of Moses (except for the ceremonial
law, which Christ has fulfilled), with here and there a mistake because
we are not perfect, but mostly to keep our faith until the end of our
lives, we have proven that our faith is strong enough. Our lives may be
difficult and it may be hard to keep the faith, but what I mean, is that
our faith is either real or it is not. There is nothing in between.
Eternal life has been given to us at the moment we convert. If the
repentance was real then the conversion is also real and then the test
will be passed, eventually. This does not mean we can slack off now. No,
it means that from the moment we convert and God credits us His
righteousness, He also knows how weak we are and how much of His grace
and power we need. He will provide us then with the power to pass this
test of life and whenever we fail He is gracious to give us another
chance and merciful for not inflicting the full punishment on us---or in
other words, sending us to hell directly.

\textbf{\emph{Future Glory}} \emph{::18 For I consider that the
sufferings of this present time are not worth comparing with the glory
that is to be revealed to us.::} \#biblestudy/memorization/todo \emph{19
For the creation waits with eager longing for the revealing of the sons
of God.}

How can one consider something with another thing that is \emph{to be
revealed}? It is not even known yet. Through verse 19 we may know what
it is that is going to be revealed---it is the sons of God. That should
be us, Christians, however, in a different form. We will finally be
revealed in a sinless body, with the ultimate freedom as God has
intended, not restricted by sickness or pain, bad thoughts and bad
intentions. We are waiting for this promise that is so unimaginable
great that nothing can compare with it. The most healthy, happy, rich
and spiritual person on Earth, combined together, could not even compare
to the least of the sons of God. For who could live up to a hundred
years old and still feel like a young man? Who could have ultimate
safety without Jesus as supreme ruler of the world, where righteousness
reigns? I tell you that both Democrats, Leftists (there is a distinction
here between libertarians and liberals) \#tags/politics and
Republicans---so that is, across the \emph{whole} spectrum or in other
words, everybody---do not have peace of mind with all the injustice in
this world. The Left, led by Nancy Pelosi, wanted to defund the police,
and she was arguing the opening of borders to all kinds of immigrants
(whether criminal or not), but she called the police when people were
peacefully protesting in her front yard. So she is willing to open the
borders of her country, but not the door of her house. She wants to
defund the police, but when she is in trouble she calls upon them. The
irony in there is not difficult to find. Even she in her perfect liberal
world, which America has almost become, does not feel safe, let alone
the ones on the Right. But here I am only speaking of the benefits we
will have in the Millennium, and I have not even started talking about
the New Earth yet!

\emph{20 For the creation was subjected to futility, not willingly, but
because of him who subjected it, in hope 21 that the creation itself
will be set free from its bondage to corruption and obtain the freedom
of the glory of the children of God.}

Did God subject creation to Himself in the \emph{hope} that He could
free it? Would that not be counterproductive? Or does God want the honor
of being the One Who set His children free? No, of course not! God
deserves all honor and all honor is already to Him. Human beings were
subjected to futility, to death that is, and not willingly indeed, but
because of the laws of God. So yes, it is because of God that people are
subjected, indirectly, but directly it is because of man's sinful
nature. God created the Law in the \emph{hope} that His children would
obtain freedom, as freedom like no one has known but only a child of God
has known, which up until man was created there were none. That too
sounds counterproductive. So God created a law so that man could be
free? Then why did God not just not create the Law at all? The answer
has been given in the previous chapters all along. God knew that because
of man's sinful nature a world without a law, which is not hypothetical
but actually was before Moses, would still not give man freedom. How
much freedom was there in those days? And so the only world where man
could have real freedom, like those of God's children, is not a world
where there is no law, and it is not a world where there is a Law where
no one can abide by the law, although the latter still is better than
the former. It is a world where people do not want to sin, or in other
words, where the law is in the people's hearts. This is not possible in
this world, even for people who have given their life to God, because we
are still living in this body. And then we get to the same conclusion as
we have formerly come to, the only way to reach this is to split the
body from the will. Hence, in heaven, where there is no body, but only
will or spirit, there is no sin, and later, on the new Earth, our
spirits will return to new bodies, with no sin. It is for this reason
that Jesus said, on the last night before he died, when the disciples
kept falling asleep in the garden of olives, that the mind is strong but
the body is weak.

\emph{22 For we know that the whole creation has been groaning together
in the pains of childbirth until now. 23 And not only the creation, but
we ourselves, who have the \textsubscript{firstfruits} of the Spirit,
groan inwardly as we wait eagerly for adoption as sons, the redemption
of our bodies.}

The firstfruits indicated here refers to the first fruit of a produce,
not to the first children or animals to be born, otherwise it would have
said firstborn. The firstborn, which are by definition the most
valuable, not because of the size or quality, but because one never
knows for sure whether he will have more children or offspring, is what
is offered to God. It is most valuable because it is the first
experience a parent has. One offers the first born animals and
symbolically offers his first born child as well. For a farmer with
multiple animals he has to do this multiple time, once for each female
animal, but for his own family, he needs to do this only once. This is
symbolically represented in the fact that we needed to sacrifice animals
multiple times, whether first born or not, at the temple, but God needs
to sacrifice His Son, His firstborn, only once (cf.~1 Peter 3:18) so
that the whole of mankind can be saved. Now the difference between the
firstborn and the firstfruits, except for the plurality and the kind of
organism, is that the firstfruits are generally speaking of the worst
quality, while with animals and humans the firstborn says nothing about
the quality of the rest. However, it is the fact that one has fruit at
all that gives one hope, no matter how small or sour the fruit. For
apple trees it may take three years before the first apples grow, and
they will be small then, but we know then that more apple will come the
next season. So we, who have received the firstfruits, yes, what we see
right now through the Holy Spirit, the miracles, both physically and
spiritually, are mere firstfruits and indicators of a greater hope that
is to come.

About adoption: We wait for adoption, this means that we are not adopted
yet. Is adoption the same as the redemption of our bodies? As far as I
understand this part of the Bible and other parts, it says we are
already children of God, waiting to be adopted---which is strange,
because usually you become a child \emph{after} adoption. Here, however,
God is not referring to adopting a child, because He has already done
that at the moment we gave our lives to Him, but it is \emph{we} who
still need to adopt a body, not the current one, for that leads to
death, but the new one, which we will receive on the New Earth (as far
as I know, I have not studied that part yet). But why does it say
``adoption as sons'' then? \#biblestudy/questions

\emph{24 For in this hope we were saved. ::Now hope that is seen is not
hope. For who hopes for what he sees? 25 But if we hope for what we do
not see, we wait for it with patience.::}

This hope, that verse 23 is speaking of, is what has saved us. We do not
have worldly certainty, something on paper---as if that will provide
certainty---but we have Someone's Word, which is more valid than any
other person's word. And so we hope for the day that our bodies will be
restored, but since we already know that it is going to happen, it is
more than hope and in practice it is merely waiting.

\emph{26 ::Likewise the Spirit helps us in our weakness. For we do not
know what to pray for as we ought, but the Spirit himself intercedes for
us with groanings too deep for words. 27 And he who searches hearts
knows what is the mind of the Spirit, because the Spirit intercedes for
the saints according to the will of God.::}
\#biblestudy/memorization/todo

It is clear from this part that the Spirit of God helps us to pray to
God. We do not pray correctly, we might pray for the things we want, but
not for what we \emph{ought} to want. But even if we do pray the right
things to God, it does not mean our hearts are set correctly. For
example we need to pray for forgiveness---that God forgives a
person---and so we do, but does that mean then that we actually want
this person to be forgiven? Does our heart really go out to this person
like Jesus' did? Most probably not, or better said, most definitely not,
for there is nothing good that dwells inside us (cf.~Romans 7:16). So
the Holy Spirit helps us with groaning. The second way the Holy Spirit
helps us, even if we do have groanings, it is when our groanings are too
deep for words and we do not know how to express these. Then too, the
Holy Spirit helps us. Thirdly, there are longings or other things, such
as grudges and shameful things, which we do not know about because they
are deeply hidden within our hearts, but the Spirit does know this and
will pray them to God to heal them, provide them and forgive them. So
even without our knowing we have been forgiven for the things we did not
even ask forgiveness for.

\emph{::28 And we know that for those who love God all things work
together for good, for those who are called according to his purpose.::}
\#biblestudy/theology/predestination/election \#biblestudy/memorization
\emph{::29 For those whom he foreknew he also predestined to be
conformed to the image of his Son, in order that he might be the
firstborn among many brothers. 30 And those whom he predestined he also
called, and those whom he called he also justified, and those whom he
justified he also glorified.::} \#biblestudy/memorization/todo

This verse is very well-known among many, fewer people, but still a lot,
know that this verse is indicative of God's works and that He uses
\emph{everything} for \emph{good}, yet \emph{only} for those who are
called according to His purpose. Now the question is whether the
``called'' in v.29 is the same as that in v.30. In fact, there is a
whole array of questions that have to do with predestination and free
will. In v.29 for example we see that those whom God foreknew He also
predestined \emph{to be conformed}. Whom did God foreknow? Did He select
these persons? By what standard did He select them---is it anything
related to merit? We do not have the answer, except that it was done out
of grace. A side question is whether people whom God did not predestine
could also be conformed to the image of his Son. The answer to that is
clearly no, \emph{for all have sinned and fall short} (Romans 3:23).

Now about the predestination to conformation. Does God mold us, like He
has said many a time in the Bible? If that is true, then where does that
leave free will? The answer to this is that God already knows beforehand
which persons are going to reject becoming like Jesus, even if God
touches their hearts. Some others, however, need this kickstart from God
coming to them and they are willing then to give up their lives in order
to conform to the image of Jesus Christ---that is those whom God
predestined and those who conformed---as they were willing anyway.
Another thing that comes to mind is the term ``in order that''
\#todo/opzoeken hina? It sounds strange to create other people, just so
that your only son can become the firstborn \emph{among many brothers}.
It is like one has a child, but in order to be able to call him ``the
oldest son'' you want a few more sons. On the other hand, I think this
verse should not be interpreted this way. It is rather that a father
wants his son to have brothers---he wants his firstborn to be
\emph{among} other brothers. No one wants his son to be alone without
any friends or brothers.

We also see that this verse does not indicate that many are called, but
only few will be glorified. In Matthew 22:14 Jesus says that many are
invited, but only few are chosen. The wording is different there. We
know from Romans that God has foreknown some people all along and let
them go through this process: foreknow -\textgreater{} predestine
-\textgreater{} call -\textgreater{} justify -\textgreater{} glorify. So
those who have been invited, to church, who have the heard Gospel
through someone or read about it, who have even been on the brink of
becoming a Christian, these were invited, but God did not know them.
Examples of these are the atheists(Footnote: Shermer does not ascribe to
atheism, he ascribes to things he is, not to what he is \emph{not}.
However, he did say he was figuratively knocking on doors to convince
people that there is no god.) Michael Shermer, founder of \emph{Sceptic}
magazine, focused on evolution and other ``sciences'', and New Testament
scholar Bart Ehrman \#tags/famouspeople/critics\#, both self-proclaimed
former ``born-again Christians'', they were invited, but God did not
know them. They did their best, perhaps, as both were active in their
ministry, and we cannot say whether they were just not willing to give
their whole lives to Christ, or whether it is that they could not give
their lives to Christ because God had not predestined them for that.
Either way, this is another topic, but God is righteousness and God can
choose whomever He wants. For the real truth is that \emph{no one} would
ever give their lives to Christ, because of our inherent evilness, which
God did not place there, unless God removes this evil from our hearts.

This leads us to the conclusion that everyone who has given his life to
Christ c.q. God the Father, for those who did not know Christ in the way
we know Him, such as Abraham and Moses, was touched by the Holy Spirit
before that moment. Those who were touched, were chosen by the Father to
be touched, for the Holy Spirit only does what the Father tells Him to
do. So everyone needs to be chosen first, before he can ever get to
Christ. The question was touched by the Holy Spirit, so that at least
\emph{we were able} to give the answer to accept Christ in our hearts,
would our own sinful will be able to overrule this moment and decide to
reject Christ anyway? The doctrine of \emph{irresistible grace} says
this is not possible \#biblestudy/theology/calvinism/irresistiblegrace.
Others might say that this would not be fair of God. 1. It would not be
fair to give people two options, but not being able to select the right
one 2. If they are given the ability to choose the right one, there
needs to be free will, so that it is the choice of man itself. 3.
\#todo/nogaftemaken

\textbf{God's Everlasting Love} \emph{31 What then shall we say to these
things? If God is for us, who can be against us? 32 He who did not spare
his own Son but gave him up for us all, how will he not also with him
graciously give us all things?}

A lot of people are against us in this world. Although the percentage is
less than we think and even this percentage is declining (see the book
``\emph{What's so great about Christianity?{}``} by Dinesh D'Souza
\#tags/famouspeople/christian \#biblestudy/devotionals/books ) it still
feels like a lot because of the megaphones they are controlling, such as
Twitter, Hollywood and the media. \#todo/nogaftemaken

\emph{33 Who shall bring any charge against God's elect? It is God who
justifies.}

If a father loves one particular child so much he would even sacrifice
his other child for it, who would dare then to bring any charge against
this child? Of course if the child has done wrong, then it is logical
that one does this. And so the devil does, he charges us with our sin,
he rightly convicts us of our sin, but he does it for a different
purpose. He does this so that we are punished and that we belong to him.
At the same time, however, the Holy Spirit convicts us of our sin as
well, yet for another reason, namely to lead us to the truth and to sin
no more. Our committed sin, every one, even the ones we are going to do
today, will be put on Jesus. So for this reason the Satan cannot charge
us anymore, if there is anyone he should charge it is Jesus Who bears
the sin but has already paid for it. So Satan can do nothing anymore
except charge those for whom the price has not been ``paid'' yet, that
is those who have not accepted Jesus. ``Paid'' in the sense that
although Jesus has paid the price already by His sufferings, He has not
given that ``money'' to the Satan yet.

\emph{34 Who is to condemn? Christ Jesus is the one who died---more than
that, who was raised---who is at the right hand of God, who indeed is
interceding for us.}

Note that ``who'' is the subject here, otherwise this would have said
``whom''. Who can condemn but God alone, and who alone can condemn but
the person who has been disadvantaged or who has paid the price. In some
cases a person or a company might pay for the damage and in return be
the one to condemn others. I cannot think of an existing example here,
but suppose someone is hurt by a deliberate manufacturing mistake
causing the lives of many people. A law firm could take up a class
action suit against the company on behalf of those people. It is not
exactly the same, because the law firm is not paying for the damage
except that it is paying its own fees up front. In all human related
cases God is the One Who is hurt, as well as another human being, but in
this case all three are the same. Jesus actually is the only example I
can think of who has paid for the damage in full, not with money but
with His own life, and now has every right to condemn in stead of the
Satan. And so sometimes He does not condemn, as in the case of the
adulterous woman (cf.~John 8:11).

\emph{35 Who shall separate us from the love of Christ? Shall
tribulation, or distress, or persecution, or famine, or nakedness, or
danger, or sword? 36 As it is written,} \emph{``For your sake we are
being killed all the day long;\emph{ }we are regarded as sheep to be
slaughtered.''} \emph{37 No, in all these things we are more than
conquerors through him who loved us. 38 ::For I am sure that neither
death nor life, nor angels nor rulers, nor things present nor things to
come, nor powers, 39 nor height nor depth, nor anything else in all
creation, will be able to separate us from the love of God in Christ
Jesus our Lord.::} \#biblestudy/memorization/todo

Europeans and Americans have many stories about a death that conquers
until the end. No one could separate Romeo and Juliet, not their
background, their family or their friends. Not even if the whole world
was against them. But that is all it was---until the end, which is
death. In death they were separated and so was their love. The Hunchback
of Notre Dame went a little further and loved Esmeralda even in the
grave and was willing to be buried alive with her. The Chinese take it
another step further and have many stories about love that conquers even
death. There was this story of a boy who loved a princess girl, who was
to be wed to other men. He protected her from them and even when they
were pushed to a cliff, they had rather choose death than to share their
love with another person---and so they jumped off the cliff. But later
when they went down to look for them, they could not find any trace.
Still some time later they found a flower they had never seen there
before, it was a rose. The girl had transformed into a beautiful flower,
but the boy had transformed into thorns to protect the girl. These
stories, however, are but myths and everyone knows that it is relatively
easy to separate people from each other's love. Wedding vows even
contain the line (or at least, they used to in the Netherlands) ``until
death do us part.'' Death is all it takes. But the love of Christ is so
strong that nothing, absolutely nothing, can separate us from it. All of
the mentioned difficulties in verse 35 have not succeeded in separating
people from the love of Christ. People have been killed and suffered in
various ways, but still they have hung on to faith in Christ. Even
things outside of this world have not been able to separate us from His
love. People who have been possessed by demons have been released
because God's love for us is stronger than the Satan's stronghold. So
how could anyone then willingly set himself free from God's love and run
back to the world? The explanation disbelievers of \emph{Once Saved
Always Saved} give, is that they can willingly give up God's love. But
how did they come to that point? It is through conviction of materials
they read, or pressure by others. If all of the above is not strong
enough to separate us, then how can a book full of mistakes on evolution
separate us from Christ? \#church/material The answer is: it cannot. It
must mean that these people like Michael Shermer and Bart Ehrman were
never Christians to begin with. \#biblestudy/devotionals/romans

\hypertarget{romans-9-esv}{%
\section{Romans 9 (ESV)}\label{romans-9-esv}}

\textbf{God's Sovereign Choice} \emph{1 I am speaking the truth in
Christ---I am not lying; my conscience bears me witness in the Holy
Spirit--- 2 that I have great sorrow and unceasing anguish in my heart.
3 For I could wish that I myself were accursed and cut off from Christ
for the sake of my brothers, my kinsmen according to the flesh. 4 They
are Israelites, and to them belong the adoption, the glory, the
covenants, the giving of the law, the worship, and the promises. 5 To
them belong the patriarchs, and from their race, according to the flesh,
is the Christ, who is God over all, blessed forever. Amen.}

I am repeating what Paul said, namely that he is not lying. He is
uttermost serious when he speaks of being cut off himself just so that
his Jewish brothers could know God. Of course, it is a small sacrifice,
one might think, to offer one life so that millions of others may know
God, but thinking it through, it is not such an easy choice to make.
First of all, consider that Paul does not personally know these millions
of Jews. Who would give up his life to save strangers? I'm sure there a
few who would do that, although they would be in the minority. Now
consider that many of those Jews wanted to kill you. Would you then be
willing to die so that they may be saved? The choice instantly becomes
harder and the number of people willing to do so have drastically
lowered. Perhaps very rational people who count the value of life in
numbers would come to the conclusions that millions of bad people are
still worth more than one good person---either that, or it would be a
very emotional person who would be willing to die, because he feels so
much love for them. Now second of all, these people are not dying. The
rational person who just counted the value of the lives not lost,
suddenly loses a lot of utility gained, because these people's lives
would not have been lost, physically anyway, and neither would Paul's.
\#church/material gebruik ook het voorbeeld van berekening van de waarde
van levens in nut en in geld voor epidemiologische doeleinden. Then
comes the question, what is Paul exchanging then? Paul gives us
something much more valuable than just his Earthly life. His life here
on Earth is only 70 or 80 years, but his life in heaven and on the New
Earth would be infinitely long. He would give up all of that to save a
bunch of strangers would tried to kill him, even at the time of writing
this letter. Paul has not even experienced this heavenly life, where the
goodness of one day would be like thousands here on Earth combined---and
still he is willing to give it all up for his ``brothers''. Perhaps Paul
argued that the utility of many persons is always more than of a single
person. A purely rational person would therefore always choose to give
his life if it could save the life of more than one person, given that
the sum of their utility is higher than he could enjoy. This means that
a rational who would live at least ten more years would not give his
life for two people who could live for five years at most. But that is
not something Paul has to worry about, because the summed utility of a
million Jews is always more than his' alone, even if he enjoys life with
God much more than any other person in history. Now the comparison for
non completely rational people, which all humans are, is made by the
trade-off between the utility enjoyed by Paul himself as a Christian and
the utility that Paul enjoys by seeing others become a Christian. Of
course it a negative utility to see other people hunger or suffer in any
kind of way. Christians, and most definitely Paul, would be willing to
give up some of their utility so that they could increase in theirs,
which at its core is a selfish thing, but not very surprisingly
considering our evil hearts \#todo/opzoeken (where does it say our
hearts are evil?) the negative utility it causes to see others suffer is
usually less than experiencing this negative utility ourselves. That is
why there are traitors, who do anything to survive. They might feel
grudge, but their will to survive and therefore not suffer themselves is
stronger. Consider for example the person who could speak Japanese in
the movie \emph{Ip Man}. \#tags/movies He was called a traitor because
he translated for the Japanese, simply so that he would have a job. But
it was through these translations that the Japanese could establish some
things in Chinese territory. Now of course the Japanese would have found
some other person to translate for him, if he would not be willing, and
perhaps he would have died as well for refusing. But it still is
choosing your own wellbeing over the suffering of others.

\sout{But at the same time this allows him to see what is given then to
these others.}

\emph{6 But it is not as though the word of God has failed. For not all
who are descended from Israel belong to Israel, 7 and not all are
children of Abraham because they are his offspring, but ``Through Isaac
shall your offspring be named.'' 8 This means that it is not the
children of the flesh who are the children of God, but the children of
the promise are counted as offspring.}

Even though all children of Abraham are biologically related to him,
they are not the offspring God referred to here. First of all, the
Ishmaelites are not part of the promise, but second of all neither are
all Jews.

\emph{9 For this is what the promise said: ``About this time next year I
will return, and Sarah shall have a son.'' 10 And not only so, but also
when Rebekah had conceived children by one man, our forefather Isaac, 11
though they were not yet born and had done nothing either good or
bad---in order that God's purpose of election might continue, not
because of works but because of him who calls--- 12 she was told, ``The
older will serve the younger.'' 13 As it is written, ``Jacob I loved,
but Esau I hated.''}

God already had purpose and elected some, God chosen us before we were
in the womb, and this example of Esau and Jacob proves this. Jacob was
chosen by God and Esau was not.
\#biblestudy/theology/predestination/election

\emph{14 What shall we say then? Is there injustice on God's part? By no
means! 15 For he says to Moses, ``I will have mercy on whom I have
mercy, and I will have compassion on whom I have compassion.'' 16 So
then it depends not on human will or exertion, but on God, who has
mercy. 17 For the Scripture says to Pharaoh, ``For this very purpose I
have raised you up, that I might show my power in you, and that my name
might be proclaimed in all the earth.'' 18 So then he has mercy on
whomever he wills, and he hardens whomever he wills.}

I know that many theologians have discussed this issue of
predetermination and many have raised arguments against election
\#biblestudy/theology/predestination/election. But nothing more can be
said than what is clearly written in these verses. God has mercy on
whomever He wills and according to His view this is fair and righteous,
for God would never do anything that is not righteous. Even when God
hardens a heart---and we know He hardened Pharaoh's heart after he
rejected God's request three times and did not even love his people
enough to let them suffer no more than three plagues---and even if He
does this directly and not give a person three chances like He did with
Pharaoh, then still it is righteous when God does it, for no one would
have been susceptible to God's love unless God would have softened him
and no one would have experience His love unless God had mercy on them
and showed the love. Once more, God first needs to have mercy on someone
and \emph{show} His love, then the result is that this person cannot
feel it, because we are hardened by nature. Then God needs to have mercy
again on this person and let this person realize it by softening his
heart. We can see this during a so-called \emph{act of God}, such as a
thunderstorm or tsunami. Some die and some survive, the latter are those
on whom God had mercy, if you will, for the sake of argument. Now of
these who survive, some will thank God, while others will not and merely
think it is coincidence. The former are those whose hearts have been
softened, let us say that the latter are those who have not been
hardened, but they were hard by nature. Let us also say that everyone is
equally hard by nature. Now except for letting them survive, God could
be even more merciful to them and let a boat full of supplies float by
right in front of their nose, while they are residing on the roof of a
building, finding shelter for the storm. Those who are neutral then,
might feel \emph{something} of God, although they might doubt the
coincidence, but others will still stick to their argument that
everything is controlled by random events, and so is this boat. These
people have most likely been hardened. Now I believe God is fair
(according to our definition) and has given people like Stephen Hawking
and Christopher Hitchens \#tags/famouspeople plenty of chances, but even
if He had not, it would still be fair (according to God's own
definition).

\emph{19 You will say to me then, ``Why does he still find fault? For
who can resist his will?'' 20 But who are you, O man, to answer back to
God? Will what is molded say to its molder, ``Why have you made me like
this?'' 21 Has the potter no right over the clay, to make out of the
same lump one vessel for honorable use and another for dishonorable
use?}

The first and foremost thing we have to realize is that we may not
answer back to God---not in this way! Questioning God is different from
asking God a question. Answering Him in this way is criticizing His
plan. Who are we to talk in such a way, to propose suggestions to God if
God is the One Who created us and our intelligence? The capacity with
which we think was put into our minds and everything we are able to
think of has been thought of by God since the beginning---but since God
has no beginning it means He has always known this. We are talking back
at God because we think the way He treats people is unfair. It is not.
First of all, because God decides on what the definition of ``fairness''
is. It is we who do not understand the concept of fairness, just like I
previously explained the example of a few people on death row
\#todo/opzoeken waar? and a president pardons them. But even that
example or the induced and deduced definition of fairness thereof do not
encompass the true meaning of fairness. Just like the law ``thou shalt
not kill'' is more comprehensive than just about the killing of persons.
Phinehas killed a person and was praised for it; Dennis Prager writes in
his book \emph{The Rational Bible: Exodus}
\#biblestudy/devotionals/books that murdering is killing with (bad)
intent. But it most likely conveys many more definitions and
explanation, perhaps more than all the smartest lawyers in this world
together could ever comprehend. Now second of all, according to God's
definition it is fair, but we cannot see it. Those who are rational
define this as ``all should play by the same rules.'' But let us take
the example of comparing one student with another and evaluating whether
each had a fair chance of getting into university. However, can one
measure the amount of intelligence each had, the concentration,
resources (such as time and money), emotional support (from their
friends and family), physical comfort and other (meta)physical and
psychological aspects? God alone knows whether something is \emph{truly}
fair or not.

\emph{22 What if God, desiring to show his wrath and to make known his
power, has endured with much patience vessels of wrath prepared for
destruction, 23 in order to make known the riches of his glory for
vessels of mercy, which he has prepared beforehand for glory---}

In the previous verse we have read that the potter may decide whether he
creates a pot for honorable or dishonorable use. Now the same goes for
God and if He created someone like Hitler or Mao \#tags/famouspeople,
people who dishonor God, then God is still righteous in having created
them, for the responsibility of their choices cannot be put on God. By
the way, can God create anything that does not give Him honor? No, He
cannot. Everything He makes is beautiful by His standard, whether it is
a handicapped child or a future dictator like Stalin or Kim Yong-Un.
When they were born they were beautiful, however, their jealousy,
paranoia and everything in their upbringing that contributed to their
ruthless dictatorship is what has made them ugly and dishonorable to
God's creation. Now these verses go even further. When you do not read
it carefully, it will say something like: ``What if God has created
vessels of wrath, just so that He can destroy it in order to show His
power?'' Now that would sound cruel or masochistic, but still, who are
we to answer back to God? It would be like us creating clay puppets and
destroying them. No one would criticize us for an act like that. As a
side note: in a previous devotional on {[}Isaiah
55:9{]}(bear://x-callback-url/open-note?id=37D75FC3-088F-47CE-8A55-EF6E4BD17438-57499-0004FF7E7D6A7CD2I
spoke of a professor who debated against William Lane Craig who used the
example of God creating sentient beings and then destroying them and
that that would be harsh. However, he did not take into consideration
the greatness of God and that the power distance in the relationship
between Him and humans is far greater than between humans and sentient
robots. Even the power distance between humans and clay jars is
insufficient to depict the infinite power distance between God and
humans. As mentioned in that piece, we did not create clay, nor did we
create the elements (atoms) I had written that we humans are not even
able to create that what clay consists of, let alone the particles
thereof (molecules, atoms and quarks) and the powers between them in the
quantum realm. Back to our main point. If you do not read it carefully,
you will think of a masochistic God, but the former paragraph already
explained that masochism does not apply to an infinitely almighty God,
there is no word for His behavior for nothing ever has been like this on
Earth. If you do read it carefully, however, you will see that God has
also made vessels of mercy so that He could show how great His glory is.
In other words, God created pots to be destroyed and pots that would be
left intact out of mercy. If God were only to create pots to be left
intact, His mercy would have less meaning (to human understanding). It
is like miracles happening every day and everywhere. This miracle would
still be as great as it is, but people would not appreciate or
understand how special it is. Hence God created vessels that are to be
destroyed so that His glory may be known even more through His act of
mercy. This, however, is still not what is exactly being said. We need
to take an even closer look at this verse. First of all, there is a
``what if'' at the beginning, so it does not say that God has done this
for this reason, but it is only a supposition. Second of all, what it
says, is that ``God endures\ldots{} in order to make known\ldots{} His
glory''. God \emph{desires} to show His wrath. Remember that this is not
a masochistic trait, for everything from God is good. His wrath is not
shown on those who do not deserve it. It would be like a war hero
killing soldiers and their general and dictator in his wrath, because
this dictator was killing their own people, just like the American army
killed \href{https://en.wikipedia.org/wiki/Qasem_Soleimani}{Qasem
Soleimani} in Iraq in 2020. So God desires to show His wrath, but does
not do that, because He wants to show His glory through the mercy He
gives to some people. The fact that God is patient alone is enough to
show His glory, considering how badly He wants to show His power, but He
even glorifies certain people, who do not deserve it, which is all the
more greater. So back to the ``What if''-statement. What if this were
God's intention? Do we have any argument then to answer back to God? No,
we do not. Each of the above reasons provides sufficient argument to
keep our mouths shut. \#church/material

\emph{24 even us whom he has called, not from the Jews only but also
from the Gentiles? 25 As indeed he says in Hosea,} \emph{``Those who
were not my people I will call `my people,'\emph{ }and her who was not
beloved I will call `beloved.'\,''} \emph{26 ``And in the very place
where it was said to them, `You are not my people,'\emph{ }there they
will be called sons of the living God.'''}

God is so merciful that some of the vessels he created for mercy are
even made for people like us, the Gentiles. He will turn this world
upside down and call those whom He hated, \emph{His people}. It is like
the poor becoming the rich, not the middle class becoming rich. God uses
extremes, not only as examples but He actually does this to show the
greatness of His power. The world thinks the firstborn should receive a
double portion, but God has always used the young one, such as Joseph
and David, to inherit the double portion and carry out God's will. He
let the handsome and strong ones, like Saul and David's son \ldots{}
\#todo/opzoeken with the long hair perish. So it is in this world, we
look up to the handsome and strong ones, the rich and powerful, but it
is they who will perish. Of course, there are some exceptions of movie
stars who are all of this and still live a relatively moral life,
perhaps even better than we do, such as Keanu Reeves and Jay Chou (who
has become a Christian now), but generally God raises the weak ones who
depend on Him to depend on Him even more. Now why would God want to
raise the weak so that they could depend on Him? Could He not have left
them weak so that they would always depend on Him? The answer is that
God wants them to enjoy life, but with restraints. A life without money
or materials is not always fun, God gives us riches so that we can enjoy
them, but when we enjoy them too much, we will forget about God and
others and only think of ourselves. That is why God raises the poor to
become rich, but they still need God's help to overcome their own
problems such as emotional insecurity and a weak conscience. He raises
the weak to become strong, but they still need God to maintain their
strength, for just like Samson they are easily prone to all kinds of
seductions, such as women, narcissism of their own body. He raises the
insecure to make them mentally strong and confident, but they are still
prone to (new) addictions, by replacing their old ones with other ones.
Codependency in drugs, can be replaced by controlling others. In short,
we need to realize we are weak and that we will always stay weak. We
need God to show us we are so dependent of Him that without Him we can
do nothing, but \emph{with} Him we are more than conquerors!

\emph{27 And Isaiah cries out concerning Israel: ``Though the number of
the sons of Israel be as the sand of the sea, only a remnant of them
will be saved, 28 for the Lord will carry out his sentence upon the
earth fully and without delay.'' 29 And as Isaiah predicted,} \emph{``If
the Lord of hosts had not left us offspring,\emph{ }we would have been
like Sodom\emph{ }and become like Gomorrah.''}

The former verses contrast this one. In the former ones it says that
even the Gentiles, who were not beloved, will now be beloved---and even
named as such by God---while the Israelites, who were beloved and were
as many as the sea, only a remnant of them will be saved---and even this
remnant is mercy of God, for if He had not given His mercy, the Jews
would have been like Sodom, that is, nothing. If we look at what has
happened to the Israelites, if the research done is correct, we see a
few tribes have ended in Africa and in India. They are far from
prosperous, technologically advanced or even morally advanced. It is
true that God has only left a remnant, Judah, I think, and perhaps
Benjamin as well, because it was located in the same area and this tribe
(perhaps) was also transported to Babylon. That God left an offspring is
not referring to \ldots{} Not only has God \#todo nog af te maken, see
Evernote.

\textbf{Israel's Unbelief} \emph{30 What shall we say, then? That
Gentiles who did not pursue righteousness have attained it, that is, a
righteousness that is by faith; 31 but that Israel who pursued a law
that would lead to righteousness did not succeed in reaching that law.
32 Why? Because they did not pursue it by faith, but as if it were based
on works. They have stumbled over the stumbling stone, 33 as it is
written,} \emph{``Behold, I am laying in Zion a stone of stumbling, and
a rock of offense;\emph{ }and whoever believes in him will not be put to
shame.''}

We see here that God (or Paul) defines two terms, \emph{righteousness}
and \emph{righteousness by faith}. Those who try to attain righteousness
without faith, will not succeed. In fact, it is exactly because of all
the works they have done, that they failed, because they put their faith
in those things. So, as was obvious from everything above, what matters
is where or who you put your faith in.

Now what God has done is more than just creating a crossways where one
chooses to go the path with or the path without faith, each with their
respective destinations. If it were that simple, some would have gone
that way and at the end of their road c.q. lives, they would have found
out that they went the wrong way. But it would be too late for them to
turn around by then. No, those who walk this path without faith are
offended by those who walk the path with faith. They are angry, like one
who stumbles over a rock and injures his little toe and falls flat on
his face. Why would these people care about what path others take? In
the movie \emph{The Case for Christ} Lee Strobel is also very frustrated
up to the point of angry with his wife for choosing the path of Christ.
Not because she is leaving him behind---because he could simply have
followed her---but for some unknown reason. He does not believe that God
is real, but why does he have to impose it so badly on others. If he
wants to convince others of his religion, which is atheism, just like
Christians want to convince others, that is understandable. But why
would he get angry over such a thing? It is exactly because of this
stumbling stone. Everything written in verse 33 is for those who do not
believe in him, that is pain, agony and shame.

\#biblestudy/devotionals/romans


\end{document}

%%%%%%%%%%%%%%%%%%%%%%%%%%%%%%%%%%%%%%%%%%%%%%%
	\begin{thebibliography}{17}
	\vspace{10.5cm}
		\bibitem{RPI} Academic and Research Computing. \textit{Text Formatting with \LaTeX\ A Tutorial}. NY, April 2007.\\
				\url{http://www.rpi.edu/dept/arc/docs/latex/latex-intro.pdf}
		\bibitem{is.skills} Leslie Lamport. \textit{\LaTeX\ for Beginners}.  fifth edition, Document Reference: 3722-2014, March 2014.\\
				\url{http://www.docs.is.ed.ac.uk/skills/documents/3722/3722-2014.pdf}
		\bibitem{Tobias Oetiker} Tobias Oetiker, Hubert Partl, Irene Hyna, and Elisabeth Schlegl. \textit{The Not So Short Introduction to \LaTeX\ }. Version 6.3, March 1994.\\
				\url{https://tobi.oetiker.ch/lshort/lshort.pdf}
		\bibitem{Helmut Kopka} Helmut Kopka and Patrick W. Daly. \textit{A Guide to \LaTeX\ and Electronic Publishing}. Addison-Wesley, fourth edition, May 2003.\\
				\url{https://www2.mps.mpg.de/homes/daly/GTL/gtl_20030512.pdf}
		\bibitem{Philip Hirschhorn} Philip Hirschhorn. \textit{Using the exam document class}. Wellesley College, second edition, MA, November 2017.\\
				\url{http://www-math.mit.edu/~psh/exam/examdoc.pdf}
		\bibitem{web1} \textit{When should I use \cs{input} vs \cs{include}}.\\
				\url{https://tex.stackexchange.com/questions/246/when-should-i-use-input-vs-include}		
		\bibitem{Overleaf} Overleaf. \textit{Inserting Images}.\\
				\url{https://www.overleaf.com/learn/latex/Inserting_Images}
		\bibitem{Rice} Rice University. \textit{\LaTeX\ Mathematical Symbols}.\\
			\url{https://www.caam.rice.edu/~heinken/latex/symbols.pdf}
		\bibitem{Overleaf1} Overleaf. \textit{Integrals, sum and limits}.\\
			\url{https://www.overleaf.com/learn/latex/Integrals,_sums_and_limits}
		\bibitem{BU} Boston University. \textit{\LaTeX\ Command Summary}. December 1994.\\
			\url{https://www.bu.edu/math/files/2013/08/LongTeX1.pdf}
		\bibitem{Overleaf2} Overleaf. \textit{Subscripts and superscripts}.\\
			\url{https://www.overleaf.com/learn/latex/Subscripts_and_superscripts}
		\bibitem{Disquis} Disquis. \textit{\LaTeX\ Color}. Addison-Wesley, second edition, Reading, MA, 1994.\\
				\url{http://latexcolor.com/}
		\bibitem{David Woods} David Woods.\textit{Useful \LaTeX\ Commands}.\\
				\url{https://www.scss.tcd.ie/~dwoods/1617/CS1LL2/HT/wk1/commands.pdf}
		\bibitem{Overleaf3} Overleaf. \textit{Spacing in Math Mode}.\\
				\url{https://www.overleaf.com/learn/latex/Spacing_in_math_mode}
		\bibitem{Overleaf4} Overleaf. \textit{Fractions and Binomials}.\\
				\url{https://www.overleaf.com/learn/latex/Fractions_and_Binomials}
		\bibitem{Overleaf5} Overleaf. \textit{Environments}.\\
				\url{https://www.overleaf.com/learn/latex/Environments}
		\bibitem{Overleaf6} Overleaf. \textit{Margin Notes}.\\
				\url{https://www.overleaf.com/learn/latex/Margin_notes}
		\bibitem{Math} Art of Problem Solving. \textit{LaTeX:Commands}\\
				\url{https://artofproblemsolving.com/wiki/index.php/LaTeX:Commands}
		\bibitem{LaTeX} Latex-Project.\\
				\url{https://www.latex-project.org/about/}
		\bibitem{texstackexchange} Tex Stack Exchange. customizing part style with Tikz.\\
				\url{https://tex.stackexchange.com/questions/159551/customizing-part-style-with-tikz}
		\bibitem{texstackexchange2}	Tex Stack Exchange. How to change chapter/section style in tufte-book?.\\
				\url{https://tex.stackexchange.com/questions/83057/how-to-change-chapter-section-style-in-tufte-book?noredirect=1&lq=1}
	\end{thebibliography}
\addtocounter{section}{14}
\addcontentsline{toc}{section}{\protect\numberline{\thesection}~~~ References}
%%%%%%%%%%%%%%%%%%%%%%%%%%%%%%%%%%%%%%%%%%%%%%%%%%%%%%%%%%%%%%%%%%
